\ifdefined\ishandout
  \documentclass[handout,landscape]{beamer} 
\else
  \documentclass[landscape]{beamer}
\fi

%\hypersetup{pdfpagemode=FullScreen} %Enabling this option will cause the slides to go full-screen on opening

\mode<handout>
{
  \usepackage{pgf}
  \usepackage{pgfpages}

\pgfpagesdeclarelayout{6 on 1 boxed}
{
  \edef\pgfpageoptionheight{\the\paperheight} 
  \edef\pgfpageoptionwidth{\the\paperwidth}
  \edef\pgfpageoptionborder{0pt}
}
{
  \pgfpagesphysicalpageoptions
  {%
    logical pages=6,%
    physical height=\pgfpageoptionheight,%
    physical width=\pgfpageoptionwidth%
  }
  \pgfpageslogicalpageoptions{1}
  {%
    border code=\pgfsetlinewidth{2pt}\pgfstroke,%
    border shrink=\pgfpageoptionborder,%
    resized width=.5\pgfphysicalwidth,%
    resized height=.5\pgfphysicalheight,%
    center=\pgfpoint{.25\pgfphysicalwidth}{.833\pgfphysicalheight}%
  }%
  \pgfpageslogicalpageoptions{2}
  {%
    border code=\pgfsetlinewidth{2pt}\pgfstroke,%
    border shrink=\pgfpageoptionborder,%
    resized width=.5\pgfphysicalwidth,%
    resized height=.5\pgfphysicalheight,%
    center=\pgfpoint{.75\pgfphysicalwidth}{.833\pgfphysicalheight}%
  }%
  \pgfpageslogicalpageoptions{3}
  {%
    border code=\pgfsetlinewidth{2pt}\pgfstroke,%
    border shrink=\pgfpageoptionborder,%
    resized width=.5\pgfphysicalwidth,%
    resized height=.5\pgfphysicalheight,%
    center=\pgfpoint{.25\pgfphysicalwidth}{.5\pgfphysicalheight}%
  }%
  \pgfpageslogicalpageoptions{4}
  {%
    border code=\pgfsetlinewidth{2pt}\pgfstroke,%
    border shrink=\pgfpageoptionborder,%
    resized width=.5\pgfphysicalwidth,%
    resized height=.5\pgfphysicalheight,%
    center=\pgfpoint{.75\pgfphysicalwidth}{.5\pgfphysicalheight}%
  }%
  \pgfpageslogicalpageoptions{5}
  {%
    border code=\pgfsetlinewidth{2pt}\pgfstroke,%
    border shrink=\pgfpageoptionborder,%
    resized width=.5\pgfphysicalwidth,%
    resized height=.5\pgfphysicalheight,%
    center=\pgfpoint{.25\pgfphysicalwidth}{.167\pgfphysicalheight}%
  }%
  \pgfpageslogicalpageoptions{6}
  {%
    border code=\pgfsetlinewidth{2pt}\pgfstroke,%
    border shrink=\pgfpageoptionborder,%
    resized width=.5\pgfphysicalwidth,%
    resized height=.5\pgfphysicalheight,%
    center=\pgfpoint{.75\pgfphysicalwidth}{.167\pgfphysicalheight}%
  }%
}


  \pgfpagesuselayout{6 on 1 boxed}[letterpaper, border shrink=5mm]
  \nofiles
}

\usepackage{listings}
\usepackage{multimedia}
\usepackage[normalem]{ulem}
\usepackage{ifthen}

\usetheme{Warsaw} 
\usecolortheme{seahorse}
\useoutertheme{infolines} 

\setbeamertemplate{blocks}[rounded][shadow=true] 

\author{Joe Fields}
\title{Introduction to Proof} 

\date{Lecture 26 (GIAM \S 5.2) \newline formulas for sums and products}
\institute[SCSU]{ {\tt fieldsj1@southernct.edu} }


\newlength{\cwidth}
\newcommand{\cents}{\settowidth{\cwidth}{c}%
\divide\cwidth by2
\advance\cwidth by-.1pt
c\kern-\cwidth
\vrule width .1pt depth.2ex height1.2ex
\kern\cwidth}

\newcommand{\sageprompt}{ {\tt sage$>$} }
\newcommand{\tab}{\rule{20pt}{0pt}}
\newcommand{\blnk}{\rule{1.5pt}{0pt}\rule{.4pt}{1.2pt}\rule{9pt}{.4pt}\rule{.4pt}{1.2pt}\rule{1.5pt}{0pt}}
\newcommand{\suchthat}{\; \rule[-3pt]{.25pt}{13pt} \;}
\newcommand{\divides}{\!\mid\!}
\newcommand{\tdiv}{\; \mbox{div} \;}
\newcommand{\restrict}[2]{#1 \,\rule[-4pt]{.125pt}{14pt}_{\,#2}}
\newcommand{\lcm}[2]{\mbox{lcm} (#1, #2)}
\renewcommand{\gcd}[2]{\mbox{gcd} (#1, #2)}
\newcommand{\Naturals}{{\mathbb N}}
\newcommand{\Integers}{{\mathbb Z}}
\newcommand{\Znoneg}{{\mathbb Z}^{\mbox{\tiny noneg}}}
\newcommand{\Enoneg}{{\mathbb E}^{\mbox{\tiny noneg}}}
\newcommand{\Qnoneg}{{\mathbb Q}^{\mbox{\tiny noneg}}}
\newcommand{\Rnoneg}{{\mathbb R}^{\mbox{\tiny noneg}}}
\newcommand{\Rationals}{{\mathbb Q}}
\newcommand{\Reals}{{\mathbb R}}
\newcommand{\Complexes}{{\mathbb C}}
%\newcommand{\F2}{{\mathbb F}_{2}}
\newcommand{\relQ}{\mbox{\textsf Q}}
\newcommand{\relR}{\mbox{\textsf R}}
\newcommand{\nrelR}{\mbox{\raisebox{1pt}{$\not$}\rule{1pt}{0pt}{\textsf R}}}
\newcommand{\relS}{\mbox{\textsf S}}
\newcommand{\relA}{\mbox{\textsf A}}
\newcommand{\Dom}[1]{\mbox{Dom}(#1)}
\newcommand{\Cod}[1]{\mbox{Cod}(#1)}
\newcommand{\Rng}[1]{\mbox{Rng}(#1)}

\DeclareMathOperator\caret{\raisebox{1ex}{$\scriptstyle\wedge$}}

\newtheorem*{defi}{Definition}
\newtheorem*{exer}{Exercise}
\newtheorem{thm}{Theorem}[section]
\newtheorem*{thm*}{Theorem}
\newtheorem{lem}[thm]{Lemma}
\newtheorem{cor}{Corollary}
\newtheorem{conj}{Conjecture}

\renewenvironment{proof}%
{\begin{quote} \emph{Proof:} }%
{\rule{0pt}{0pt} \newline \rule{0pt}{15pt} \hfill Q.E.D. \end{quote}}


\newcommand{\vs}{\rule{0pt}{12pt}}
\newcommand{\notimplies}{\;\not\!\!\!\implies}
\newcommand{\dx}{\,\mbox{d}x}

\AtBeginSection[]
{
 \begin{frame}{Table of Contents} 
  \tableofcontents[currentsection]
 \end{frame}
}

%%%% SAVE %%%%
%{ %magic to get a full screen image...
%\setbeamertemplate{navigation symbols}{}  % hide navigation buttons 
%\setbeamertemplate{background canvas}{\centerline{\includegraphics 
%	[height=\paperheight]{Cantor_4.jpeg}}}
%\begin{frame}[plain]
%\rule{0pt}{0pt}
%\end{frame} 
%} %end of magic


\begin{document}

\begin{frame}[plain]
  \titlepage
\end{frame}

\section{motivation}

\begin{frame}{Gauss in elementary school}
\begin{itemize}
\item Was it a punishment? \pause
\item The class was tasked with adding all of the numbers from 1 to 100. \pause
\item This was 1784, so imagine the tools involved. \pause
\item Pairing 1 with 100, 2 with 99 and so on.\pause
\item Only works because 100 is even.\pause
\item Write the sum twice: \pause
\begin{tabular}{ccccccccc}
$1$ & $+$ & $2$ & $+$ & $3$ & $+$ & \textellipsis & $+$ & $n$ \\
$n$ & + & $n-1$ & + & $n-2$ & + & \textellipsis & + & $1$ \\
\end{tabular} \pause
\item There are $n$ columns each totalling $n+1$, and that's the double of our sum, so \pause
\[  \sum_{k=1}^n k \; = \; \frac{1}{2} n(n+1). \]
\end{itemize}
\end{frame}

\begin{frame}{Riemann sums}
\begin{itemize}
\item Linearity of integrals:
\[ \int a\cdot f(x) + b \cdot g(x) \dx \; = \;  a \int f(x) \dx \, +  \, b \int g(x) \dx \] \pause
\item We can get the integral of any polynomial provided we know the integral of the powers of $x$. \pause
\item The generalization of Gauss's trick gives \newline
\[ \sum_{j=1}^n j \; = \; \frac{n (n+1)}{2} \] \pause
\item That gives us the integral of $x$. \pause
\item Desmos \pause
\item We need formulas for the sum of squares, cubes and so on.
\end{itemize}
\end{frame}

\begin{frame}{formulas for Riemann sums}
\begin{itemize}
\item First powers:
\[ \sum_{k=1}^n k \; = \; \frac{1}{2} n(n+1) \] \pause
\item Squares:
\[ \sum_{k=1}^n k^2 \; = \; \frac{1}{6} n(n+1)(2n+1) \] \pause
\item Cubes:
\[ \sum_{k=1}^n k^3 \; = \; \frac{1}{4} n^2(n+1)^2 \] \pause
\item A nice memory aid is that the last one is the square of first one. 
\end{itemize}
\end{frame}

\begin{frame}{Bernoulli}
\begin{itemize}
\item Jacob Bernoulli may have taken this too far\textellipsis \pause
\item He developed a table of formulas for the sums of powers up through the 10th powers.\pause
\item He also showed how similar formulas could be developed for even higher powers using ``Bernoulli numbers". \pause 
\item This is the 10th power formula: \newline
\hspace{-1in}$\frac{1}{66} n(n+1)(2n+1)(n^2+n-1)(3n^6+9n^5+2n^4-11n^3+3n^2+10n-5)$
\pause
\item ``With the help of this table, it took me less than half of a quarter of an hour to find that the tenth powers of the first 1000 numbers being added together will yield the sum 91,409,924,241,424,243,424,241,924,242,500.'' \pause
\item That's a pretty fancy computation to complete in 7.5 minutes -- by hand!
\end{itemize}
\end{frame}

\section{two approaches}

\begin{frame}{use the IH along the way}
\begin{itemize}
\item There are two general strategies for completing the inductive step. \pause
\item You can take a sum whose upper limit is $k+1$ and seperate it into two parts: \pause
\[ \sum_{j=1}^{k+1} f(j) \; = \; f(k+1) \; + \; \sum_{j=1}^{k} f(j) \] \pause
\item That's the sum from stage $k$ plus a final term. \pause
\item The inductive hypothesis applies to the sum from stage $k$. \pause
\item In this approach you're generally working with expressions not equalities.
\end{itemize}
\end{frame}

\begin{frame}{start with the IH}
\begin{itemize}
\item An alternative approach is to just begin with the inductive hypothesis. \pause
\item Then figure out what to add to both sides to wind up with the inductive conclusion. \pause
\item This will usually end up using the same algebraic steps as the previous method.\pause \newline
(Just organized differently.) \pause
\item Either approach is fine.  It's a matter of taste. \pause
\item In this approach you're generally working with equalities rather than expressions.
\end{itemize}
\end{frame}

\section{examples}

\begin{frame}{Gauss's formula}
\begin{thm*} 
$$ \forall n \in \Naturals^+, \; \sum_{j=1}^{n} j \; = \; \frac{n(n+1)}{2} $$.
\end{thm*}
{\em Proof:} We proceed by induction on $n$.

\noindent {\bf Basis: }  Both the sum and the formula evaluate to 1 when $n=1$:

\[ \sum_{j=1}^{1} j \; = \; 1, \]

and 

\[ \left. \frac{n(n+1)}{2} \right|_{n=1} \; = \; 1. \]

\end{frame}

\begin{frame}{continued}

\noindent {\bf Inductive step: }  Consider the sum

\[ \sum_{j=1}^{k+1} j. \]

We can separate out the last term of this sum.

\[ = (k+1) + \sum_{j=1}^{k} j \]

Next, we can use the inductive hypothesis to replace the sum (the part 
that goes from 1 to $k$) with a formula.  So 

\end{frame}

\begin{frame}{continued}

\begin{align*}
\uncover<1->{   \sum_{j=1}^{k+1} j \; }
\uncover<2->{   = & \; (k+1) + \sum_{j=1}^{k} j \\ }
\uncover<3->{  = & \; (k+1) + \frac{k(k+1)}{2} \\ }
\uncover<4->{  = & \; \frac{2(k+1)}{2} + \frac{k(k+1)}{2} \\ }
\uncover<5->{  = & \; \frac{2(k+1) + k(k+1)}{2} \\ }
\uncover<6->{  = & \; \frac{(k+1) \cdot (k+2)}{2}. \\ }
\end{align*}
\uncover<7->{\hfill {\em Q.E.D.} }

\end{frame}

\begin{frame}{once more with feeling}
\begin{itemize}
\item Let's use the alternative approach to prove this a second time. \pause
\item The base case is the same, so we'll jump straight to the inductive step. \pause
\end{itemize}

\noindent {\bf Inductive step: }
The inductive hypothesis is 

\[ \sum_{j=1}^{k} j  \; = \; \frac{k(k+1)}{2}. \]

\pause

Adding $k+1$ to both sides of this yields

\[ (k+1) \, + \, \sum_{j=1}^{k} j  \; = \; (k+1) \, + \, \frac{k(k+1)}{2}. \]

\end{frame}

\begin{frame}{continued}
Thus 
\begin{align*}
\uncover<1->{ \sum_{j=1}^{k+1} j  \; &= (k+1) \, + \, \frac{k(k+1)}{2} \\}
\uncover<2->{ &= \; \frac{2(k+1)}{2} \, + \, \frac{k(k+1)}{2}. \\}
\uncover<3->{                        &= \; \frac{(k+1)(k+2)}{2}. \\}
\end{align*}
\uncover<4->{\hfill {\em Q.E.D.} }

\end{frame}

\begin{frame}{what to add?}
\begin{itemize}
\item A particularly pernicious beginner's error is to get stuck in the habit of adding $k+1$ to both sides. \pause
\item Don't do that! \pause
\item That's what we added in the last proof, but that's basically the only time we'll do that. \pause
\item The thing you need to add is the $k+1$-st term of the sum.  Apply whatever the formula for the summand is to $k+1$.
\end{itemize}
\end{frame}

\begin{frame}{sum of cubes}
\begin{thm*} 
$$ \forall n \in \Naturals^+, \; \sum_{j=1}^{n} j^3 \; = \; \frac{n^2(n+1)^2}{4} $$.
\end{thm*}
{\em Proof:} We proceed by induction on $n$.

\noindent {\bf Basis: }  Both the sum and the formula evaluate to 1 when $n=1$:

\[ \sum_{j=1}^{1} j^3 \; = \; 1, \]

and 

\[ \left. \frac{n^2(n+1)^2}{4} \right|_{n=1} \; = \; 1. \]

\end{frame}

\begin{frame}{continued}
\noindent {\bf Inductive step: }
The inductive hypothesis is 

\[ \sum_{j=1}^{k} j^3  \; = \; \frac{k^2(k+1)^2}{4}. \]

\pause

We want to show that 

\[ \sum_{j=1}^{k+1} j^3  \; = \; \frac{(k+1)^2(k+2)^2}{4}. \]
\end{frame}

\begin{frame}{continued}
Adding $(k+1)^3$ to both sides of the inductive hypothesis gives

\[ (k+1)^3 \, + \, \sum_{j=1}^{k} j^3  \;  = \; (k+1)^3 \, + \, \frac{k^2(k+1)^2}{4}.\]

Thus,
\begin{align*}
\uncover<2->{ \sum_{j=1}^{k+1} j^3  \;  & = \; (k+1)^3 \, + \, \frac{k^2(k+1)^2}{4} \\ }
\uncover<3->{  & = \; \frac{4(k+1)^3 \, + \, k^2(k+1)^2}{4} \\ }
\uncover<4->{  & = \; \frac{(k+1)^2 \cdot (4(k+1) \, + \, k^2)}{4} \\} 
\uncover<5->{  & = \; \frac{(k+1)^2 \cdot (k^2 + 4k + 4) }{4} \\ }
\uncover<6->{  & = \; \frac{(k+1)^2 \cdot (k+2)^2 }{4} \\ }
\end{align*}
\uncover<7->{\hfill {\em Q.E.D.} }

\end{frame}

\begin{frame}
\begin{itemize}
\item Execise 1 in \S 5.1 had you investigating the truth of 

\[ \sum_{j=0}^n 4j+1 = 2n^2 + 3n + 1. \] \pause

\item Let's prove it now using mathematical induction.
\end{itemize}
\end{frame}

\begin{frame}{proof}
\begin{thm*}
\[ \forall n \in \Naturals,\; \sum_{j=0}^n 4j+1 \, = \, 2n^2 + 3n + 1. \]
\end{thm*}
\pause

{\em Proof:} We proceed by induction on $n$. \pause

\noindent {\bf Basis: }  Both the sum and the formula evaluate to 1 when $n=0$:

\[ \sum_{j=0}^{0} 4j+1 \; = \; 1, \]

\pause

and 

\[ \left. 2n^2+3n+1 \right|_{n=0} \; = \; 1. \]
\end{frame}

\begin{frame}{continued}
\noindent {\bf Inductive step: }
The inductive hypothesis is 

\[ \sum_{j=0}^{k} 4j+1  \; = \; 2k^2+3k+1. \]

\pause

We want to show that 

\[ \sum_{j=0}^{k+1} 4j+1  \; = \; 2(k+1)^2+3(k+1)+1 \; = \; 2k^2 + 7k + 6 . \]

\end{frame}

\begin{frame}{continued}
Adding $4(k+1)+1$ to both sides of the inductive hypothesis gives 

\[ (4(k+1)+1) \, + \, \sum_{j=0}^{k} 4j+1  \; = \; (4(k+1)+1) \, + \, 2k^2+3k+1. \]

\pause 

Thus,
\begin{align*}
\uncover<2->{ \sum_{j=0}^{k+1} 4j+1 \; &= \; (4(k+1)+1) \, + \, (2k^2+3k+1) \\}
\uncover<3->{ &= \; (4k+5) \, + \, (2k^2+3k+1) \\}
\uncover<4->{ &= \; 2k^2+7k+6 \\}
\end{align*}

\uncover<5->{\hfill {\em Q.E.D.} }
\end{frame}

\begin{frame}{a product}
\begin{itemize}
\item Formulas for products work pretty much the same as formulas for sums. \pause
\item The only difference is that we'll multiply by the $k+1$-st term rather than adding it. \pause
\item Here's a nice example we can practice on:\pause
\end{itemize}

\begin{thm*}
\[ \forall n \in \Naturals^+, \; \prod_{j=1}^n \frac{j}{j+1} \; = \; \frac{1}{n+1}. \]
\end{thm*} 
\end{frame}

\begin{frame}{proof}

{\em Proof:} We proceed by induction on $n$. \pause

\noindent {\bf Basis: }
When $n=1$, we have

\[ \prod_{j=1}^n \frac{j}{j+1} \; = \; \frac{1}{2}, \]

\pause

and 

\[ \left. \frac{1}{n+1} \right|_{n=1} \; = \; \frac{1}{2}. \]

\end{frame}

\begin{frame}{continued}
\noindent {\bf Inductive step: }
The inductive hypothesis is 

\[ \prod_{j=1}^k \frac{j}{j+1} \; = \; \frac{1}{k+1}. \]

\pause

We want to show that 

\[  \prod_{j=1}^{k+1} \frac{j}{j+1} \; = \; \frac{1}{k+2}.\]

\end{frame}

\begin{frame}{continued}

Consider the product

\[ \prod_{j=1}^{k+1} \frac{j}{j+1}. \]

\pause

Note that 

\[ \prod_{j=1}^{k+1} \frac{j}{j+1} \quad = \quad \left( \frac{k+1}{k+2} \right) \; \cdot \; \prod_{j=1}^k \frac{j}{j+1}. \]

\pause

Using the inductive hypothesis we get 
\begin{align*}
\uncover<3->{ \prod_{j=1}^{k+1} \frac{j}{j+1} \; & = \; \frac{k+1}{k+2} \cdot \frac{1}{k+1} \\ }
\uncover<4->{ & = \; \frac{1}{k+2} \\}
\end{align*}
\uncover<5->{\noindent as desired.
\pause
\hfill {\em Q.E.D.} }
\end{frame}

\section{concluding remarks}

\begin{frame}{final thoughts}
\begin{itemize}
\item Be careful to seperate out the computations you do in the base case. \pause
\item You don't have to do a WWTS sentence (but it certainly doesn't hurt!).  \pause \newline
At least write the inductive conclusion down somewhere on scrap paper. \pause
\item Try organizing proofs in both of the ways we've illustrated. \pause \newline
  Some proofs will lend themselves better to one or the other style, so be able to do both. \pause \newline
 (It's okay to pick a favorite.) \pause
\item Practice! \pause There's no way to really get the ``zen'' of doing an inductive proof without doing lots of inductive proofs.

\end{itemize}
\end{frame}

\end{document}
