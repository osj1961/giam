%\documentclass[handout,landscape]{beamer}
\documentclass[landscape]{beamer}
%\hypersetup{pdfpagemode=FullScreen}
\mode<handout>
{
  \usepackage{pgf}
  \usepackage{pgfpages}

\pgfpagesdeclarelayout{6 on 1 boxed}
{
  \edef\pgfpageoptionheight{\the\paperheight} 
  \edef\pgfpageoptionwidth{\the\paperwidth}
  \edef\pgfpageoptionborder{0pt}
}
{
  \pgfpagesphysicalpageoptions
  {%
    logical pages=6,%
    physical height=\pgfpageoptionheight,%
    physical width=\pgfpageoptionwidth%
  }
  \pgfpageslogicalpageoptions{1}
  {%
    border code=\pgfsetlinewidth{2pt}\pgfstroke,%
    border shrink=\pgfpageoptionborder,%
    resized width=.5\pgfphysicalwidth,%
    resized height=.5\pgfphysicalheight,%
    center=\pgfpoint{.25\pgfphysicalwidth}{.833\pgfphysicalheight}%
  }%
  \pgfpageslogicalpageoptions{2}
  {%
    border code=\pgfsetlinewidth{2pt}\pgfstroke,%
    border shrink=\pgfpageoptionborder,%
    resized width=.5\pgfphysicalwidth,%
    resized height=.5\pgfphysicalheight,%
    center=\pgfpoint{.75\pgfphysicalwidth}{.833\pgfphysicalheight}%
  }%
  \pgfpageslogicalpageoptions{3}
  {%
    border code=\pgfsetlinewidth{2pt}\pgfstroke,%
    border shrink=\pgfpageoptionborder,%
    resized width=.5\pgfphysicalwidth,%
    resized height=.5\pgfphysicalheight,%
    center=\pgfpoint{.25\pgfphysicalwidth}{.5\pgfphysicalheight}%
  }%
  \pgfpageslogicalpageoptions{4}
  {%
    border code=\pgfsetlinewidth{2pt}\pgfstroke,%
    border shrink=\pgfpageoptionborder,%
    resized width=.5\pgfphysicalwidth,%
    resized height=.5\pgfphysicalheight,%
    center=\pgfpoint{.75\pgfphysicalwidth}{.5\pgfphysicalheight}%
  }%
  \pgfpageslogicalpageoptions{5}
  {%
    border code=\pgfsetlinewidth{2pt}\pgfstroke,%
    border shrink=\pgfpageoptionborder,%
    resized width=.5\pgfphysicalwidth,%
    resized height=.5\pgfphysicalheight,%
    center=\pgfpoint{.25\pgfphysicalwidth}{.167\pgfphysicalheight}%
  }%
  \pgfpageslogicalpageoptions{6}
  {%
    border code=\pgfsetlinewidth{2pt}\pgfstroke,%
    border shrink=\pgfpageoptionborder,%
    resized width=.5\pgfphysicalwidth,%
    resized height=.5\pgfphysicalheight,%
    center=\pgfpoint{.75\pgfphysicalwidth}{.167\pgfphysicalheight}%
  }%
}


  \pgfpagesuselayout{6 on 1 boxed}[letterpaper, border shrink=5mm]
  \nofiles
}

\usepackage{listings}
%\lstset{language=TeX}
\usepackage{multimedia}
\usepackage[normalem]{ulem}
\usepackage{amssymb}

%\usecolortheme[named=Purple]{structure} 
%\usetheme{Copenhagen}
\usetheme{Warsaw} 
\usecolortheme{seahorse}
\useoutertheme{infolines} 
%\usetheme[height=7mm]{Rochester} 
%\setbeamertemplate{items}[ball] 
\setbeamertemplate{blocks}[rounded][shadow=true] 
%\setbeamertemplate{navigation symbols}{} 
\author{Joe Fields}
\title{Introduction to Proof} 
%\subtitle{}
\date{Lecture 14 (GIAM \S 3.1)}
\institute[SCSU]{ {\tt fieldsj1@southernct.edu} }

\newcommand{\versionNum}{$3.2$\ }

\newboolean{InTextBook}
\setboolean{InTextBook}{false}
\newboolean{InWorkBook}
\setboolean{InWorkBook}{false}
\newboolean{InHints}
\setboolean{InHints}{false}

%When this boolean is true (beginning in Section 5.1) we will use the convention
% that $0 \in \Naturals$.  If it is false we will continue to count $1$ as the smallest
%natural number (thus making Giuseppe Peano spin in his grave...)
 
\newboolean{ZeroInNaturals}

%This boolean is used to distinguish the version where we use $\sim$ rather than $\lnot$

\newboolean{LNotIsSim}

%The values of the last two booleans are set in ``switches.tex''

%\input{switches}

\let\savedlnot\lnot
\ifthenelse{\boolean{LNotIsSim}}{\renewcommand{\lnot}{\sim} }{}

%This command puts different amounts of space depending on whether we are
% in the text, the workbook or the hints & solutions manual. 
\newcommand{\twsvspace}[3]{%
 \ifthenelse{\boolean{InTextBook} }{\vspace{#1}}{%
  \ifthenelse{\boolean{InWorkBook} }{\vspace{#2}}{%
   \ifthenelse{\boolean{InHints} }{\vspace{#3}}{} %
   }%
  }%
 }


\newcommand{\wbvfill}{\ifthenelse{\boolean{InWorkBook}}{\vfill}{}}
\newcommand{\wbitemsep}{\ifthenelse{\boolean{InWorkBook} }{\rule[-24pt]{0pt}{60pt}}{}}
\newcommand{\textbookpagebreak}{\ifthenelse{\boolean{InTextBook}}{\newpage}{}}
\newcommand{\workbookpagebreak}{\ifthenelse{\boolean{InWorkBook}}{\newpage}{}}
\newcommand{\hintspagebreak}{\ifthenelse{\boolean{InHints}}{\newpage}{}}

\newcommand{\hint}[1]{\ifthenelse{\boolean{InHints}}{ {\par \hspace{12pt} \color[rgb]{0,0,1} #1 } }{}}
\newcommand{\inlinehint}[1]{\ifthenelse{\boolean{InHints}}{ { \color[rgb]{0,0,1} #1 } }{}}

%\newlength{\cwidth}
%\newcommand{\cents}{\settowidth{\cwidth}{c}%
%\divide\cwidth by2
%\advance\cwidth by-.1pt
%c\kern-\cwidth
%\vrule width .1pt depth.2ex height1.2ex
%\kern 3\cwidth}
\newcommand{\cents}{\textcent\kern 5pt}

\newcommand{\sageprompt}{ {\tt sage$>$} }
\newcommand{\tab}{\rule{20pt}{0pt}}
\newcommand{\blnk}{\rule{1.5pt}{0pt}\rule{.4pt}{1.2pt}\rule{9pt}{.4pt}\rule{.4pt}{1.2pt}\rule{1.5pt}{0pt}}
\newcommand{\suchthat}{\; \rule[-3pt]{.5pt}{13pt} \;}
\newcommand{\divides}{\!\mid\!}
\newcommand{\tdiv}{\; \mbox{div} \;}
\newcommand{\restrict}[2]{#1 \,\rule[-4pt]{.25pt}{14pt}_{\,#2}}
\newcommand{\lcm}[2]{\mbox{lcm} (#1, #2)}
\renewcommand{\gcd}[2]{\mbox{gcd} (#1, #2)}
\newcommand{\Naturals}{{\mathbb N}}
\newcommand{\Integers}{{\mathbb Z}}
\newcommand{\Znoneg}{{\mathbb Z}^{\mbox{\tiny noneg}}}
\ifthenelse{\boolean{ZeroInNaturals}}{%
  \newcommand{\Zplus}{{\mathbb Z}^+} }{%
  \newcommand{\Zplus}{{\mathbb N}} }
\newcommand{\Enoneg}{{\mathbb E}^{\mbox{\tiny noneg}}}
\newcommand{\Qnoneg}{{\mathbb Q}^{\mbox{\tiny noneg}}}
\newcommand{\Rnoneg}{{\mathbb R}^{\mbox{\tiny noneg}}}
\newcommand{\Rationals}{{\mathbb Q}}
\newcommand{\Reals}{{\mathbb R}}
\newcommand{\Complexes}{{\mathbb C}}
%\newcommand{\F2}{{\mathbb F}_{2}}
\newcommand{\relQ}{\mbox{\textsf Q}}
\newcommand{\relR}{\mbox{\textsf R}}
\newcommand{\nrelR}{\mbox{\raisebox{1pt}{$\not$}\rule{1pt}{0pt}{\textsf R}}}
\newcommand{\relS}{\mbox{\textsf S}}
\newcommand{\relA}{\mbox{\textsf A}}
\newcommand{\Dom}[1]{\mbox{Dom}(#1)}
\newcommand{\Cod}[1]{\mbox{Cod}(#1)}
\newcommand{\Rng}[1]{\mbox{Rng}(#1)}

\DeclareMathOperator\caret{\raisebox{1ex}{$\scriptstyle\wedge$}}

\newtheorem*{defi}{Definition}
\newtheorem*{exer}{Exercise}
\newtheorem{thm}{Theorem}[section]
\newtheorem*{thm*}{Theorem}
\newtheorem{lem}[thm]{Lemma}
\newtheorem*{lem*}{Lemma}
\newtheorem{cor}{Corollary}
\newtheorem{conj}{Conjecture}

\renewenvironment{proof}%
{\begin{quote} \emph{Proof:} }%
{\rule{0pt}{0pt} \newline \rule{0pt}{15pt} \hfill Q.E.D. \end{quote}}


\newcommand{\vs}{\rule{0pt}{12pt}}

\def\mycommand{\setlength{\abovedisplayskip}{4pt}%
\setlength{\belowdisplayskip}{-12pt}%
\setlength{\abovedisplayshortskip}{0pt}%
\setlength{\belowdisplayshortskip}{0pt}}

\let\oldselectfont\selectfont
\def\selectfont{\oldselectfont\mycommand}

\mycommand

\AtBeginSection[]
{
 \begin{frame}{Table of Contents} 
  \tableofcontents[currentsection]
 \end{frame}
}

%%%% SAVE %%%%
%{ %magic to get a full screen image...
%\setbeamertemplate{navigation symbols}{}  % hide navigation buttons 
%\setbeamertemplate{background canvas}{\centerline{\includegraphics 
%	[height=\paperheight]{Cantor_4.jpeg}}}
%\begin{frame}[plain]
%\rule{0pt}{0pt}
%\end{frame} 
%} %end of magic


\begin{document}

\begin{frame}[plain]
  \titlepage
\end{frame}

\section{exploring a motivating example}

\begin{frame}{the product of 4 consecutive numbers}
\begin{itemize}
\item The product of 4 consecutive natural numbers is always one less than a square. \pause
\item Explore! \pause
\item Process: what hypotheses can we use? \pause
\item Try reframing the statement as a UCS. \pause (Universal Conditional Sentence) \pause
\item The antecedant is your hypothesis! \pause (at least it's one of them) 
\end{itemize}
\end{frame}

\begin{frame}{refining the statement}
\begin{itemize}
\item We want to refine the statement of the theorem until it's precise enough (and simple enough) that we can prove it. \pause
\[ \forall a,b,c,d \in \Integers, (\mbox{a,b,c,d  consecutive}) 
\implies \exists k \in \Integers, a{\cdot}b{\cdot}c{\cdot}d = k^2 -1 
\]
\pause
\item Are all of those variables really necessary? \pause
\[ \forall a \in \Integers, \exists k \in \Integers, a(a+1)(a+2)(a+3) = k^2 - 1. \]
\end{itemize}
\end{frame}

\begin{frame}{can we prove it now?}
\begin{itemize}
\item Notice that our final formulation is about finding (there exists) $k$ that works for a particular $a$. \pause
\item See page 121. \pause
\item Well, that came out of left field!
\end{itemize}
\end{frame}

\begin{frame}{what the what?}
\begin{itemize}
\item There're at least two possible routes to discovering that $k=a^2+3a+1$ will do the trick. \pause
\item Guess and check. \pause 
\item Algebra. \pause 
\item Both of these give us opportunities for expanding your analytical toolkits.
\end{itemize}
\end{frame}

\begin{frame}{Algebra}
\begin{itemize}
\item One impediment is that we need to find the product of $a$, $a+1$, $a+2$ and $a+3$. \pause
\item You can use FOIL to multiply these in pairs, but you're left with a binomial times a trinomial. \pause
\item Often people are taught to ``just grind through the distributive law'' to form such products. \pause
\item The table method is easier: \pause

\begin{tabular}{c|ccc}
\rule[-6pt]{0pt}{20pt} & \rule{12pt}{0pt} $a^2$ \rule{12pt}{0pt}  & \rule{12pt}{0pt} $5a$ \rule{12pt}{0pt} & \rule{12pt}{0pt} $6$ \rule{12pt}{0pt} \\ \hline
\rule[-6pt]{0pt}{24pt} $a^2$ & \uncover<7->{ $a^4$ }& \uncover<8->{ $5a^3$ } & \uncover<9->{ $6a^2$ }\\
\rule[-6pt]{0pt}{24pt} $a$ & \uncover<10->{ $a^3$ } & \uncover<11->{ $5a^2$ } & \uncover<12->{ $6a$ }\\
\end{tabular}

\uncover<13->{
\item  Notice that the like terms appear on diagonals of the table. \pause
}

\uncover<14->{
\[ a^4 \; + \; 6a^3 \; + \; 11a^2 \; + \; 6a \]
}

\end{itemize}
\end{frame}

\begin{frame}{square root of a polynomial?}
\begin{itemize}
\item So we need to figure out a polynomial whose square is 
\[ a^4 \; + \; 6a^3 \; + \; 11a^2 \; + \; 6a \; + \; 1. \] \pause
\item Notice that it will have to be a quadratic. \pause
\item The leading term will have to be $a^2$. \pause
\item The constant term will be 1 (or $-1$). \pause
\item Those considerations mean it will be of the form $x^2 + mx + 1$. \pause
\item Another chance to use a table! \pause

\vspace{.2in}

\begin{tabular}{c|ccc}
\rule[-6pt]{0pt}{20pt} & \rule{12pt}{0pt} $a^2$ \rule{12pt}{0pt}  & \rule{12pt}{0pt} $ma$ \rule{12pt}{0pt} & \rule{12pt}{0pt} $1$ \rule{12pt}{0pt} \\ \hline
\rule[-6pt]{0pt}{24pt} $a^2$ & \uncover<7->{ $a^4$ }& \uncover<8->{ $ma^3$ } & \uncover<9->{ $a^2$ }\\
\rule[-6pt]{0pt}{24pt} $ma$ & \uncover<10->{ $ma^3$ } & \uncover<11->{ $m^2a^2$ } & \uncover<12->{ $ma$ }\\
\rule[-6pt]{0pt}{24pt} $1$ & \uncover<13->{ $a^2$ } & \uncover<14->{ $ma$ } & \uncover<15->{ $1$ }\\
\end{tabular}

\uncover<16->{
\[ a^4 + 2ma^3 + (m^2+2)a^2 + (2m)a + 1 \]
}
\end{itemize}
\end{frame}

\begin{frame}{guess and check}
\begin{itemize}
\item By experiment we can find the first several values of the sequence in question. \pause

\begin{tabular}{c|c|c}
\rule[-6pt]{0pt}{20pt} \rule{12pt}{0pt} $a$ \rule{12pt}{0pt} & $a(a+1)(a+2)(a+3)+1$ & \rule{12pt}{0pt} $k$ \rule{12pt}{0pt} \\ \hline
1 & 25 & 5 \\
2 & 121 & 11 \\
3 & 361 & 19 \\
4 & 841 & 29 \\
$\vdots$ & $\vdots$ & $\vdots$ \\
\end{tabular}
\pause
\item A common way to analyze a sequence is to form a ``difference table.'' \pause \newline
Write the terms of the sequence in a row -- and under them, write the differences between consecutive terms. \pause

\vspace{.2in}

\begin{tabular}{cccccccc}
\rule[-6pt]{0pt}{20pt} 5 &   & 11 &   & 19 &    & 29 & \\
\rule[-6pt]{0pt}{20pt}   & 6 &    & 8 &    & 10 &    & \\
\end{tabular}

\end{itemize}
\end{frame}

\begin{frame}{vive la difference!}
\begin{itemize}

\item If the pattern doesn't become apparent, write the differences between the differences in a 3rd row. \pause

\vspace{.1in}

\begin{tabular}{cccccccc}
\rule[-6pt]{0pt}{20pt} 5 &   & 11 &   & 19 &    & 29 & \\
\rule[-6pt]{0pt}{20pt}   & 6 &    & 8 &    & 10 &    & \\
\rule[-6pt]{0pt}{20pt}   &   &  2 &   &  2 &    & 2? & \\
\end{tabular}
\pause

\vspace{.1in}

\item Each row in a difference table is analogous to a derivative in Calculus. \pause

\item The first row is $f(x)$, the second row is similar to $f'(x)$, the 3rd row is analogous to $f''(x)$, {\em et cetera}.\pause
\item We seem to be looking at an $f(x)$ whose 2nd derivative is the constant 2. \pause
\item This is a clue that the given sequence is related to $x^2$. \pause

\end{itemize}
\end{frame}


\begin{frame}{more difference}
\begin{itemize}
\item By thinking about squares we see that the sequence can be rewritten as a square plus something small.\pause

\begin{tabular}{cccccccc}
\rule[-6pt]{0pt}{20pt} \uncover<2>{5} \uncover<3->{4+1} &   & \uncover<2>{11} \uncover<3->{9+2}&   & \uncover<2>{19}\uncover<3->{16+3} &    & \uncover<2>{29}\uncover<3->{25+4} & \\
\end{tabular}
\pause

\pause

\item Can you see that this is $(a+1)^2 + a$ ?

\end{itemize}
\end{frame}

\section{direct proofs}

\begin{frame}{}
\begin{itemize}
\item First: Proofs require good, precise definitions of the terms involved. \pause
\item See table 3.1 in GIAM.  There is a link in the video description to a handout.\pause
\item Convert the statements of theorems to UCS's whenever possible. \pause This helps you clarify what the hypotheses are. \pause
\item The method we use for direct proofs is known as ``generalizing from the generic particular.''

\end{itemize}
\end{frame}

\begin{frame}{g.f.t.g.p.}
\begin{itemize}
\item Suppose you are trying to prove  $\displaystyle \forall x \in U, \; P(x)\, \implies \, Q(x)$. \pause
\item If, for any element $y \in U$, the statement $P(y)$ is false, the conditional $P(y)\, \implies \, Q(y)$ is vacuously true. \pause So we don't need to worry about those\textellipsis \pause
\item We concentrate instead on particular elements of $U$ that make $P$ true! \pause But we don't want to make any other presumptions. \pause That's the ``generic particular.'' \pause
\item See the outline on page 125.
\end{itemize}
\end{frame}

\begin{frame}{what goes on in the dots}
\begin{itemize}
\item Often the interior of a direct proof will have the following character: \pause
\item You'll use the definition of whatever property the ``generic particular'' has, so that you can re-express that property in more familiar terms. \pause
\item You'll work with this version of the property until\textellipsis \pause
\item Finally you use a definition again to convert back to the language of the desired conclusion.
\end{itemize}
\end{frame}

\begin{frame}{a trivial example}
\begin{itemize}
\item Let's prove that the sum of two odd numbers is even. \pause

\vfill

\begin{quote}
{\em Proof:} Suppose that $x$ and $y$ are particular, but arbitarily chosen odd numbers.  By the definition of odd, it follows that there are integers $m$ and $n$ such that $x=2m+1$ and $y=2n+1$.  Thus, $x+y \; = \; (2m+1) + (2n+1) \; = \; 2(m+n+1)$.  Since $m+n+1$ is an integer, it follows (by the definition of even) that $x+y$ is even.

\hspace{\fill} Q.E.D. 
\end{quote}

\vfill

\end{itemize}
\end{frame}

\begin{frame}{a proof about floor}
\begin{itemize}
\item We previously saw that this is {\bf not} true: \pause

\[ \forall x, y \in \Reals, \; \lfloor x + y \rfloor \; = \; \lfloor x \rfloor \; + \; \lfloor y \rfloor. \] \pause

\item For example if $x=y=1.9$ we get $\displaystyle \lfloor x + y \rfloor \; = \; 3$, but  $\displaystyle \lfloor x \rfloor \; + \; \lfloor y \rfloor \; = \; 2$. \pause

\item However, when one of the number is an integer it works! \pause

\begin{thm}
\[ \forall x \in \Reals, \, \forall n \in \Integers, \, 
\lfloor x + n \rfloor = \lfloor x \rfloor + n \]
\end{thm}
\pause
\item Let's trace through the proof.  It appears on page 128. 
\end{itemize}
\end{frame}

\section{Advice}

\begin{frame}{some tips}
\begin{itemize}
\item Page 128 in GIAM
\end{itemize}
\end{frame}

\begin{frame}{tools}
\begin{itemize}
\item What tools are we allowed to use? \pause
\begin{itemize}
\item Axioms \pause
\item Definitions \pause
\item Previous results \pause
\end{itemize}
\item For now, the definitions handout has all the definitions you'll need. \pause \newline
You should make every effort to memorize these! \pause \newline
(And in your future mathematical work, make it a practice to memorize the definitions.) \pause
\item Axioms \pause
\begin{itemize}
\item Euclid's ``common notions'' \pause
\item Closure axioms \pause
\item Peano axioms \pause \newline
(You'll seldom need to appeal to these, but you should know they exist.)
\end{itemize}
\end{itemize}
\end{frame}

\begin{frame}{closure}
\begin{itemize}
\item We say ``a set is closed'' under some operation, when the result of the operation always winds up back in the set. \pause
\item For instance $\Rationals$ is closed under multiplication because the product of two rational numbers is again a rational number. \pause
\item Another example, $\Reals^\ast$ is closed under division because the one problematic real number ($0$) is not in the set, and the quotient of two non-zero real numbers is a non-zero real number. \pause
\item Some closure statements are taken to be axioms, others are provable so there is no need to treat them as axioms. 
\end{itemize}
\end{frame}

\begin{frame}{more on closure}
\begin{itemize}
\item For example we'll take as an axiom that $\Naturals$ is closed under $+$, with that as a starting point, one could prove that $\Naturals$ is closed under $\cdot$ (multiplication).  \pause While we won't ask you to do the proof, you can now regard the statement that ``the product of two natural numbers is a natural number'' as a previous result. \pause
\item Closure of $\Integers$ under both addition and multiplicaton are also taken as axiomatic. \pause
\item You're asked to prove that $\Rationals$ is closed under both addition and multiplicaton in the exercises for this section. \pause \newline
You can use the closure axioms for $\Integers$ while doing these proofs. \pause \newline
And afterwards you can treat them as ``previous results''!
\end{itemize}
\end{frame}

\begin{frame}{a closure compendium}
\begin{itemize}
\item The natural numbers ($\Naturals$) are closed under $+$ and $\cdot$ (but not $-$ or $\div$) \pause
\item The integers ($\Integers$) are closed under $+$, $\cdot$ and $-$ (but not under $\div$) \pause
\item The rationals ($\Rationals$) are closed under $+$, $-$, $\cdot$ and $\div$ (with a single well-known exception).\pause
\item The reals ($\Reals$) are closed under $+$, $-$, $\cdot$ and $\div$ \pause (again, except for division by $0$.)\pause
\item The complex numbers ($\Complexes$) are closed under $+$, $-$, $\cdot$ and $\div$ \pause (yeah, yeah, except $0$.)\pause
\item We take the closure statements for $\Naturals$, $\Integers$ and $\Reals$ as axioms, and the statements for $\Rationals$ and $\Complexes$ as things that we must prove (using the statements that are axioms.)
\end{itemize}
\end{frame}

\end{document}
