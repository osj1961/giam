%\documentclass[handout,landscape]{beamer}
\documentclass[landscape]{beamer}
%\hypersetup{pdfpagemode=FullScreen}
\mode<handout>
{
  \usepackage{pgf}
  \usepackage{pgfpages}

\pgfpagesdeclarelayout{6 on 1 boxed}
{
  \edef\pgfpageoptionheight{\the\paperheight} 
  \edef\pgfpageoptionwidth{\the\paperwidth}
  \edef\pgfpageoptionborder{0pt}
}
{
  \pgfpagesphysicalpageoptions
  {%
    logical pages=6,%
    physical height=\pgfpageoptionheight,%
    physical width=\pgfpageoptionwidth%
  }
  \pgfpageslogicalpageoptions{1}
  {%
    border code=\pgfsetlinewidth{2pt}\pgfstroke,%
    border shrink=\pgfpageoptionborder,%
    resized width=.5\pgfphysicalwidth,%
    resized height=.5\pgfphysicalheight,%
    center=\pgfpoint{.25\pgfphysicalwidth}{.833\pgfphysicalheight}%
  }%
  \pgfpageslogicalpageoptions{2}
  {%
    border code=\pgfsetlinewidth{2pt}\pgfstroke,%
    border shrink=\pgfpageoptionborder,%
    resized width=.5\pgfphysicalwidth,%
    resized height=.5\pgfphysicalheight,%
    center=\pgfpoint{.75\pgfphysicalwidth}{.833\pgfphysicalheight}%
  }%
  \pgfpageslogicalpageoptions{3}
  {%
    border code=\pgfsetlinewidth{2pt}\pgfstroke,%
    border shrink=\pgfpageoptionborder,%
    resized width=.5\pgfphysicalwidth,%
    resized height=.5\pgfphysicalheight,%
    center=\pgfpoint{.25\pgfphysicalwidth}{.5\pgfphysicalheight}%
  }%
  \pgfpageslogicalpageoptions{4}
  {%
    border code=\pgfsetlinewidth{2pt}\pgfstroke,%
    border shrink=\pgfpageoptionborder,%
    resized width=.5\pgfphysicalwidth,%
    resized height=.5\pgfphysicalheight,%
    center=\pgfpoint{.75\pgfphysicalwidth}{.5\pgfphysicalheight}%
  }%
  \pgfpageslogicalpageoptions{5}
  {%
    border code=\pgfsetlinewidth{2pt}\pgfstroke,%
    border shrink=\pgfpageoptionborder,%
    resized width=.5\pgfphysicalwidth,%
    resized height=.5\pgfphysicalheight,%
    center=\pgfpoint{.25\pgfphysicalwidth}{.167\pgfphysicalheight}%
  }%
  \pgfpageslogicalpageoptions{6}
  {%
    border code=\pgfsetlinewidth{2pt}\pgfstroke,%
    border shrink=\pgfpageoptionborder,%
    resized width=.5\pgfphysicalwidth,%
    resized height=.5\pgfphysicalheight,%
    center=\pgfpoint{.75\pgfphysicalwidth}{.167\pgfphysicalheight}%
  }%
}


  \pgfpagesuselayout{6 on 1 boxed}[letterpaper, border shrink=5mm]
  \nofiles
}

\usepackage{listings}
%\lstset{language=TeX}
\usepackage{multimedia}
\usepackage[normalem]{ulem}
\usepackage{amssymb}

%\usecolortheme[named=Purple]{structure} 
%\usetheme{Copenhagen}
\usetheme{Warsaw} 
\usecolortheme{seahorse}
\useoutertheme{infolines} 
%\usetheme[height=7mm]{Rochester} 
%\setbeamertemplate{items}[ball] 
\setbeamertemplate{blocks}[rounded][shadow=true] 
%\setbeamertemplate{navigation symbols}{} 
\author{Joe Fields}
\title{Introduction to Proof} 
%\subtitle{}
\date{Lecture 13 (GIAM \S 2.6 \& 2.7)}
\institute[SCSU]{ {\tt fieldsj1@southernct.edu} }


\newlength{\cwidth}
\newcommand{\cents}{\settowidth{\cwidth}{c}%
\divide\cwidth by2
\advance\cwidth by-.1pt
c\kern-\cwidth
\vrule width .1pt depth.2ex height1.2ex
\kern\cwidth}

\newcommand{\sageprompt}{ {\tt sage$>$} }
\newcommand{\tab}{\rule{20pt}{0pt}}
\newcommand{\blnk}{\rule{1.5pt}{0pt}\rule{.4pt}{1.2pt}\rule{9pt}{.4pt}\rule{.4pt}{1.2pt}\rule{1.5pt}{0pt}}
\newcommand{\suchthat}{\; \rule[-3pt]{.25pt}{13pt} \;}
\newcommand{\divides}{\!\mid\!}
\newcommand{\tdiv}{\; \mbox{div} \;}
\newcommand{\restrict}[2]{#1 \,\rule[-4pt]{.125pt}{14pt}_{\,#2}}
\newcommand{\lcm}[2]{\mbox{lcm} (#1, #2)}
\renewcommand{\gcd}[2]{\mbox{gcd} (#1, #2)}
\newcommand{\Naturals}{{\mathbb N}}
\newcommand{\Integers}{{\mathbb Z}}
\newcommand{\Znoneg}{{\mathbb Z}^{\mbox{\tiny noneg}}}
\newcommand{\Enoneg}{{\mathbb E}^{\mbox{\tiny noneg}}}
\newcommand{\Qnoneg}{{\mathbb Q}^{\mbox{\tiny noneg}}}
\newcommand{\Rnoneg}{{\mathbb R}^{\mbox{\tiny noneg}}}
\newcommand{\Rationals}{{\mathbb Q}}
\newcommand{\Reals}{{\mathbb R}}
\newcommand{\Complexes}{{\mathbb C}}
%\newcommand{\F2}{{\mathbb F}_{2}}
\newcommand{\relQ}{\mbox{\textsf Q}}
\newcommand{\relR}{\mbox{\textsf R}}
\newcommand{\nrelR}{\mbox{\raisebox{1pt}{$\not$}\rule{1pt}{0pt}{\textsf R}}}
\newcommand{\relS}{\mbox{\textsf S}}
\newcommand{\relA}{\mbox{\textsf A}}
\newcommand{\Dom}[1]{\mbox{Dom}(#1)}
\newcommand{\Cod}[1]{\mbox{Cod}(#1)}
\newcommand{\Rng}[1]{\mbox{Rng}(#1)}

\DeclareMathOperator\caret{\raisebox{1ex}{$\scriptstyle\wedge$}}

\newtheorem*{defi}{Definition}
\newtheorem*{exer}{Exercise}
\newtheorem{thm}{Theorem}[section]
\newtheorem*{thm*}{Theorem}
\newtheorem{lem}[thm]{Lemma}
\newtheorem{cor}{Corollary}
\newtheorem{conj}{Conjecture}

\renewenvironment{proof}%
{\begin{quote} \emph{Proof:} }%
{\rule{0pt}{0pt} \newline \rule{0pt}{15pt} \hfill Q.E.D. \end{quote}}


\newcommand{\vs}{\rule{0pt}{12pt}}

\AtBeginSection[]
{
 \begin{frame}{Table of Contents} 
  \tableofcontents[currentsection]
 \end{frame}
}

%%%% SAVE %%%%
%{ %magic to get a full screen image...
%\setbeamertemplate{navigation symbols}{}  % hide navigation buttons 
%\setbeamertemplate{background canvas}{\centerline{\includegraphics 
%	[height=\paperheight]{Cantor_4.jpeg}}}
%\begin{frame}[plain]
%\rule{0pt}{0pt}
%\end{frame} 
%} %end of magic


\begin{document}

\begin{frame}[plain]
  \titlepage
\end{frame}

\section{Arguments}

\begin{frame}{deduction vs induction}
\begin{itemize}
\item Induction is the more natural process. \pause
\item It's about seeing patterns and coming to the conclusion that those patterns will continue. \pause
\item ``I've eaten those red berries many times and I've always been sick afterwards\textellipsis \pause \newline
I shouldn't eat the red berries.'' \pause
\item Deduction is less natural, we need to learn to apply it. \pause
\item Deduction is about chaining logical consequences together to reach a conclusion. \pause
\item ``The badger has sharp claws and teeth.  Sharp claws and teeth won't feel good on my face. If I poke the badger with this stick\textellipsis \pause \newline
I shouldn't poke the badger.''
\end{itemize}
\end{frame}

\begin{frame}{is this the right place for an argument?}
\begin{itemize}
\item Monty Python's Argument Clinic sketch. \pause \href{https://youtu.be/xpAvcGcEc0k}{https://youtu.be/xpAvcGcEc0k} \pause
\item ``a connected series of statements designed to establish a definite proposition'' \pause
\item More or less the correct definition of an argument (in Logic and Philosophy, not in Sports)
\end{itemize}
\end{frame}

\begin{frame}{arguing convincingly}
\begin{itemize}
\item A proof is nothing more than a convincing formal argument. \pause
\item An argument is just a collection of statements. \pause (3 kinds) \pause
\item Hypotheses\pause -- statements that we treat as definitely true (at least for the current situation). \pause
\item Deductions\pause -- statements that are deduced from statements that come before them (hypotheses and earlier deductions) \pause
\item Conclusion\pause -- the conclusion is the whole point of the argument. It is the ``definite proposition'' that the ``connected series of statements'' was designed to ``establish.'' \pause
\item The conclusion is also a deduction -- the final one.
\end{itemize}
\end{frame}

\section{rules of inference}

\begin{frame}{inference}
\begin{itemize}
\item Inference is the process of making a deduction \pause (so ``infer'' and ``deduce'' are synonyms). \pause
\item The rules of inference are a small collection of mini arguments that give us a formal scheme for inferring something from earlier statements. \pause
\item They're the glue that holds our ``connected series of statements'' together.
\end{itemize}
\end{frame}

\begin{frame}{Some almost trivial arguments}
\begin{itemize}
\item Conjunctive simplification \pause 
\newline 

\begin{center}
\begin{tabular}{cl}
 & $A \land B$ \\ \hline
$\therefore$ & $A$ \\
\end{tabular}
\end{center}
\pause 

\item The format we use for a rule of inference is like that. A bunch of hypotheses (one to a line) appear above a horizontal line, the symbol $\therefore$ (which stands for the word ``therefore'') and the conclusion are below the horizontal line. \pause
\item Conjuntive simplication has just one hypothesis (but often there are two or more) \pause
\item Disjunctive addition also has a single hypothesis. \pause \newline

\begin{center}
\begin{tabular}{cl}
 & $A$ \\ \hline
$\therefore$ & $A \lor B$ \\
\end{tabular}
\end{center}
\pause

\item These two are pretty lame. \pause But you still need them from time to time!

\end{itemize}
\end{frame}

\section{syllogisms}

\begin{frame}{somewhat less trivially...}
\begin{itemize}
\item A {\em syllogism} is a mini argument with two hypotheses. \pause
\item Many of the rules of inference are syllogisms. \pause
\item We previously mentioned the syllogistic form:

\begin{center}
\begin{tabular}{cl}
 & All men are mortal \\ 
 & Socrates is a man \\ \hline
$\therefore$ & Socrates is a mortal \\
\end{tabular}
\end{center}

\end{itemize}
\end{frame}

\begin{frame}{modus ponens}
\begin{itemize}
\item Modus ponens is the generalization of that argument about Socrates. \pause
\item Think of it as:

\begin{center}
\begin{tabular}{cl}
 & If $x$ is a human, then $x$ is mortal. \\
 & Socrates is human. \\ \hline
$\therefore$ & Socrates is mortal.\\
\end{tabular}
\end{center}
\pause

\item Made completely abstract, it looks like: \pause

\begin{center}
\begin{tabular}{cl}
 & $A \implies B$ \\
 & $A$ \\ \hline
$\therefore$ & $B$\\
\end{tabular}
\end{center}

\end{itemize}
\end{frame}

\begin{frame}{terminology}
\begin{itemize}
\item The conditional in modus ponens is known as the {\em major premise}. \pause
\item The other hypothesis is called the {\em minor premise}. \pause
\item There are variants that have to do with quantifiers \pause
\item Modus Ponens, \pause Universal Modus Ponens, \pause the Particular Form of Universal Modus Ponens. \pause
\item The Socrates argument is actually P.F.U.M.P.
\end{itemize}
\end{frame}

\begin{frame}{UMP and PFUMP}
\begin{itemize}
\item The particular form of universal modus ponens is: \pause \newline

\begin{center}
\begin{tabular}{cl}
 & $ \forall x \in U, \; A(x) \implies B(x)$ \\
 & $A(p)$ (where $p$ is a particular element of $U$) \\ \hline
$\therefore$ & $B(p)$\\
\end{tabular}
\end{center}

\pause

\item Full on {\em Universal} modus ponens is somewhat less common because the hypotheses are quite strong. \pause

\begin{center}
\begin{tabular}{cl}
 & $ \forall x \in U, \; A(x) \implies B(x)$ \\
 & $ \forall x \in U, \; A(x)$  \\ \hline
$\therefore$ & $\forall x \in U, \; B(x)$\\
\end{tabular}
\end{center}

\end{itemize}
\end{frame}

\begin{frame}{trust but verify}
\begin{itemize}
\item There are two ways we can verify a rule of inference. \pause
\item Using a truth table. \pause (we'll come back to this\textellipsis) \pause
\item Showing that it is equivalent to a tautology. \pause
\item For example modus ponens can be re-expressed as: \pause \newline
$ ( (A \implies B) \land A ) \implies B $. \pause
\item is that tautological?
\end{itemize}
\end{frame}

\begin{frame}{a truth table approach}
\begin{itemize}
\item Construct a truth table with columns for {\em all} of the statements in the argument you're analyzing. \pause
\item Mark the rows where all of the hypotheses are true. \pause (We call these the {\em critical rows}.) \pause
\item Verify that the conclusion is true in all of the critical rows. \pause
\item We'll use this approach on our next syllogism.
\end{itemize}
\end{frame}

\begin{frame}{modus tollens}
\begin{itemize}
\item {\em Modus tollens} is sometimes called the method of denial. \pause \newline
(Similarly, {\em modus ponens} is sometimes called the method of affirming.) \pause
\item It has the same major premise as {\em modus ponens}. \pause
\item Here's the particular form of universal modus tollens: \pause \newline

\begin{center}
\begin{tabular}{cl}
 & $ \forall x \in U, \; A(x) \implies B(x)$ \\
 & $\lnot B(p)$  \\ \hline
$\therefore$ & $ \lnot A(p)$\\
\end{tabular}
\end{center}

\end{itemize}
\end{frame}

\begin{frame}{a truth table approach}
\begin{itemize}
\item Let's verify!
\end{itemize}
\end{frame}

\begin{frame}{a whirlwind tour}
\begin{itemize}
\item There's a link to a handout in the video description. \pause
\item Let's look briefly at each of the rules of inference. \pause
\item Table 2.3 in GIAM.
\end{itemize}
\end{frame}

\section{common errors}

\begin{frame}{so common}
\begin{itemize}
\item There are a couple of errors that are made so frequently that logicians have names for them. \pause
\item These have to do with messing up a {\em modus ponens} or {\em modus tollens} style argument. \pause
\item The errors are named after the change that would need to be made to the major premise to get {\em modus ponens}. \pause
\item The converse error is:

\begin{center}
\begin{tabular}{cl}
 & $A \implies B$ \\
 & $B$ \\ \hline
$\therefore$ & $A$\\
\end{tabular}
\end{center}

\pause

\item ``All fish live in water, dolphins live in water, therefore dolphins are fish.''
\end{itemize}
\end{frame}

\begin{frame}{the inverse error}
\begin{itemize}
\item Essentially, a person who has made the converse error has mistaken a conditional for its converse. \pause
\item If one confuses a statement for its inverse, you get the inverse error: \pause

\begin{center}
\begin{tabular}{cl}
 & $A \implies B$ \\
 & $\lnot A$ \\ \hline
$\therefore$ & $\lnot B$\\
\end{tabular}
\end{center}

\pause
\item ``If you install solar panels your electric bill will go down.  Jim did not install solar panels therefore his electric bill did not go down.'' \pause

\item Notice that if we replace the major premise with its inverse we get {\em modus ponens} \pause (in a thin disguise). \pause

\begin{center}
\begin{tabular}{cl}
 & $\lnot A \implies \lnot B$ \\
 & $\lnot A$ \\ \hline
$\therefore$ & $\lnot B$\\
\end{tabular}
\end{center}

 
\end{itemize}
\end{frame}
\end{document}
