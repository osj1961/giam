\ifdefined\ishandout
  \documentclass[handout,landscape]{beamer} 
\else
  \documentclass[landscape]{beamer}
\fi

%\hypersetup{pdfpagemode=FullScreen} %Enabling this option will cause the slides to go full-screen on opening

\mode<handout>
{
  \usepackage{pgf}
  \usepackage{pgfpages}

\pgfpagesdeclarelayout{6 on 1 boxed}
{
  \edef\pgfpageoptionheight{\the\paperheight} 
  \edef\pgfpageoptionwidth{\the\paperwidth}
  \edef\pgfpageoptionborder{0pt}
}
{
  \pgfpagesphysicalpageoptions
  {%
    logical pages=6,%
    physical height=\pgfpageoptionheight,%
    physical width=\pgfpageoptionwidth%
  }
  \pgfpageslogicalpageoptions{1}
  {%
    border code=\pgfsetlinewidth{2pt}\pgfstroke,%
    border shrink=\pgfpageoptionborder,%
    resized width=.5\pgfphysicalwidth,%
    resized height=.5\pgfphysicalheight,%
    center=\pgfpoint{.25\pgfphysicalwidth}{.833\pgfphysicalheight}%
  }%
  \pgfpageslogicalpageoptions{2}
  {%
    border code=\pgfsetlinewidth{2pt}\pgfstroke,%
    border shrink=\pgfpageoptionborder,%
    resized width=.5\pgfphysicalwidth,%
    resized height=.5\pgfphysicalheight,%
    center=\pgfpoint{.75\pgfphysicalwidth}{.833\pgfphysicalheight}%
  }%
  \pgfpageslogicalpageoptions{3}
  {%
    border code=\pgfsetlinewidth{2pt}\pgfstroke,%
    border shrink=\pgfpageoptionborder,%
    resized width=.5\pgfphysicalwidth,%
    resized height=.5\pgfphysicalheight,%
    center=\pgfpoint{.25\pgfphysicalwidth}{.5\pgfphysicalheight}%
  }%
  \pgfpageslogicalpageoptions{4}
  {%
    border code=\pgfsetlinewidth{2pt}\pgfstroke,%
    border shrink=\pgfpageoptionborder,%
    resized width=.5\pgfphysicalwidth,%
    resized height=.5\pgfphysicalheight,%
    center=\pgfpoint{.75\pgfphysicalwidth}{.5\pgfphysicalheight}%
  }%
  \pgfpageslogicalpageoptions{5}
  {%
    border code=\pgfsetlinewidth{2pt}\pgfstroke,%
    border shrink=\pgfpageoptionborder,%
    resized width=.5\pgfphysicalwidth,%
    resized height=.5\pgfphysicalheight,%
    center=\pgfpoint{.25\pgfphysicalwidth}{.167\pgfphysicalheight}%
  }%
  \pgfpageslogicalpageoptions{6}
  {%
    border code=\pgfsetlinewidth{2pt}\pgfstroke,%
    border shrink=\pgfpageoptionborder,%
    resized width=.5\pgfphysicalwidth,%
    resized height=.5\pgfphysicalheight,%
    center=\pgfpoint{.75\pgfphysicalwidth}{.167\pgfphysicalheight}%
  }%
}


  \pgfpagesuselayout{6 on 1 boxed}[letterpaper, border shrink=5mm]
  \nofiles
}

\usepackage{listings}
\usepackage{multimedia}
\usepackage[normalem]{ulem}
\usepackage{ifthen}

\usetheme{Warsaw} 
\usecolortheme{seahorse}
\useoutertheme{infolines} 

\setbeamertemplate{blocks}[rounded][shadow=true] 

\author{Joe Fields}
\title{Introduction to Proof} 

\date{Lecture 22 (GIAM \S 4.3) \newline set operations}
\institute[SCSU]{ {\tt fieldsj1@southernct.edu} }


\newlength{\cwidth}
\newcommand{\cents}{\settowidth{\cwidth}{c}%
\divide\cwidth by2
\advance\cwidth by-.1pt
c\kern-\cwidth
\vrule width .1pt depth.2ex height1.2ex
\kern\cwidth}

\newcommand{\sageprompt}{ {\tt sage$>$} }
\newcommand{\tab}{\rule{20pt}{0pt}}
\newcommand{\blnk}{\rule{1.5pt}{0pt}\rule{.4pt}{1.2pt}\rule{9pt}{.4pt}\rule{.4pt}{1.2pt}\rule{1.5pt}{0pt}}
\newcommand{\suchthat}{\; \rule[-3pt]{.25pt}{13pt} \;}
\newcommand{\divides}{\!\mid\!}
\newcommand{\tdiv}{\; \mbox{div} \;}
\newcommand{\restrict}[2]{#1 \,\rule[-4pt]{.125pt}{14pt}_{\,#2}}
\newcommand{\lcm}[2]{\mbox{lcm} (#1, #2)}
\renewcommand{\gcd}[2]{\mbox{gcd} (#1, #2)}
\newcommand{\Naturals}{{\mathbb N}}
\newcommand{\Integers}{{\mathbb Z}}
\newcommand{\Znoneg}{{\mathbb Z}^{\mbox{\tiny noneg}}}
\newcommand{\Enoneg}{{\mathbb E}^{\mbox{\tiny noneg}}}
\newcommand{\Qnoneg}{{\mathbb Q}^{\mbox{\tiny noneg}}}
\newcommand{\Rnoneg}{{\mathbb R}^{\mbox{\tiny noneg}}}
\newcommand{\Rationals}{{\mathbb Q}}
\newcommand{\Reals}{{\mathbb R}}
\newcommand{\Complexes}{{\mathbb C}}
%\newcommand{\F2}{{\mathbb F}_{2}}
\newcommand{\relQ}{\mbox{\textsf Q}}
\newcommand{\relR}{\mbox{\textsf R}}
\newcommand{\nrelR}{\mbox{\raisebox{1pt}{$\not$}\rule{1pt}{0pt}{\textsf R}}}
\newcommand{\relS}{\mbox{\textsf S}}
\newcommand{\relA}{\mbox{\textsf A}}
\newcommand{\Dom}[1]{\mbox{Dom}(#1)}
\newcommand{\Cod}[1]{\mbox{Cod}(#1)}
\newcommand{\Rng}[1]{\mbox{Rng}(#1)}

\DeclareMathOperator\caret{\raisebox{1ex}{$\scriptstyle\wedge$}}

\newtheorem*{defi}{Definition}
\newtheorem*{exer}{Exercise}
\newtheorem{thm}{Theorem}[section]
\newtheorem*{thm*}{Theorem}
\newtheorem{lem}[thm]{Lemma}
\newtheorem{cor}{Corollary}
\newtheorem{conj}{Conjecture}

\renewenvironment{proof}%
{\begin{quote} \emph{Proof:} }%
{\rule{0pt}{0pt} \newline \rule{0pt}{15pt} \hfill Q.E.D. \end{quote}}


\newcommand{\vs}{\rule{0pt}{12pt}}
\newcommand{\notimplies}{\;\not\!\!\!\implies}

\AtBeginSection[]
{
 \begin{frame}{Table of Contents} 
  \tableofcontents[currentsection]
 \end{frame}
}

%%%% SAVE %%%%
%{ %magic to get a full screen image...
%\setbeamertemplate{navigation symbols}{}  % hide navigation buttons 
%\setbeamertemplate{background canvas}{\centerline{\includegraphics 
%	[height=\paperheight]{Cantor_4.jpeg}}}
%\begin{frame}[plain]
%\rule{0pt}{0pt}
%\end{frame} 
%} %end of magic


\begin{document}

\begin{frame}[plain]
  \titlepage
\end{frame}

\section{union, intersection and complementation}

\begin{frame}{definitions}
\begin{itemize}
\item Given two sets, if we glom all the elements together into one big set that's the union. \pause
\item If we look only at the elements that they have in common, that's the intersection. \pause
\item The symbols for union and intersection ($\cup$ and $\cap$) are simply rounded versions of $\vee$ and $\wedge$. \pause
\item This shows us the correspondence: \pause \newline
 {\em union} ($\cup$) corresponds with  {\em or} ($\vee$) \pause \newline
and  {\em intersection} ($\cap$) corresponds with  {\em and} ($\wedge$). 
\end{itemize}
\end{frame}

\begin{frame}{an example}
\begin{itemize}
\item Example: Let the universe be $U = \{0,1,2, \ldots 10 \}$.  Define $A = \{x \in U \suchthat\; x \; \mbox{is prime} \}$ and let $B = \{x \in U \suchthat\; 3 \divides x \}$. \pause
\item So $A = \{2, 3, 5, 7\}$ \pause and $B = \{0,3,6,9\}$. \pause
\item Then $A \cup B \; = \; \{ 0, 2, 3, 5, 6, 7, 9 \}$ \pause \hspace{.2in} and $A \cap B \; = \; \{ 3 \}$. \pause
\item Notice that $A \cup B$ consists of those elements of $U$ that are either prime OR a multiple of 3. \pause
\item Also, $A \cap B$ consists of those elements of $U$ that are  prime AND a multiple of 3.
\end{itemize}
\end{frame}

\begin{frame}{a 2nd example}
\begin{itemize}
\item Let's keep the universal set $U = \{0,1,2, \ldots 10 \}$. \pause
\item Let $A = \{x \in U \suchthat\; x \; \mbox{is a perfect square} \}$ and let $B = \{x \in U \suchthat\; x \; \mbox{is even} \}$. \pause
\item So $A = \{0,1,4,9\}$ \pause and $B = \{0,2,4,6,8,10\}$. \pause
\item Then $A \cup B \; = \; \{ 0, 1, 2, 4, 6, 8, 9, 10 \}$ \pause \hspace{.2in} and $A \cap B \; = \; \{ 0, 4 \}$. \pause
\item Notice that $A \cup B$ consists of those elements of $U$ that are either a square OR even. \pause
\item Also, $A \cap B$ consists of those elements of $U$ that are a square AND even.
\end{itemize}
\end{frame}

\begin{frame}{formal definitions}
\begin{itemize}
\item Suppose that $A$ is a set with membership criterion $P(x)$, \pause \newline
so $A = \{ x \in U \suchthat P(x)\}$. \pause
\item Similarly, suppose that $B$ is a set with membership criterion $Q(x)$, \pause \newline
so $B = \{ x \in U \suchthat Q(x)\}$. \pause
\item We define $A \cap B$ via the logical conjunction: \pause \newline
$A \cap B = \{ x \in U \suchthat P(x) \, \land\, Q(x)\}$. \pause
\item In a similar fashion, we define $A \cup B$ via the logical disjunction: \pause \newline
$A \cup B = \{ x \in U \suchthat P(x) \, \lor\, Q(x)\}$. \pause
\item Since $P(x)$ is literally identical in meaning to ``$x \in A$'' and likewise, $Q(x) = $ ``$x \in B$'', \pause 
\item \rule{0pt}{18pt}$A \cap B = \{ x \in U \suchthat\; x \in A \, \land \, x \in B \}$ and \pause \rule{0pt}{18pt}$A \cup B = \{ x \in U \suchthat \; x \in A \, \lor \, x \in B \}$.
\end{itemize}
\end{frame}

\begin{frame}{big versions}
\begin{itemize}
\item Because of the way $\cup$ and $\cap$ are defined it's almost immediate that they satisfy the same rules (associative, commutative, distributive, etc) as $\lor$ and $\land$. \pause
\item Because of associativity, there's no ambiguity in doing the union or intersection of lots of sets all at once. \pause
\item So we can use ``big'' versions of $\cup$ and $\cap$ and a notation like sigma notation for sums. \pause
\item As an example, \newline
\[ \bigcup_{n=2}^{\infty} [1/n, 1] \] \pause
is the union of an infinite number of intervals. \pause
\item Just for fun, guess if $0$ is in the previous union.
\end{itemize}
\end{frame}

\begin{frame}{what about not?}
\begin{itemize}
\item We've seen the set theoretic versions of disjunction and conjunction. \pause
\item What about logical negation? \pause
\item Suppose that $A$ is a set with membership criterion $P(x)$. \pause 
\item What set has membership criterion $\lnot P(x)$? \pause 
\end{itemize}
\end{frame}

\begin{frame}{example}
\begin{itemize}
\item Example: Let the universe be $U = \{0,1,2, \ldots 10 \}$.  Define $A = \{x \in U \suchthat x \; \mbox{is prime} \}$. \pause
\item So, in this example $P(x)$ is ``$x$ is prime.'' \pause \newline
Which means that $\lnot P(x)$ would be ``$x$ is not prime.'' \pause
\item What set is that? \pause It consists of all the things in $U$ that are {\em not} in $A$! \pause
\item We call such a set the {\em complement} of $A$ and denote it by $\overline{A}$. \pause \newline
(Other common notations for the complement are $A^c$ and $A'$.) \pause
\item Here's the formal definition: \pause \newline
Given a fixed universal set $U$ and a set $A \subseteq U$, with membership criterion $M_A(x)$, \pause
\[ \overline{A} \; = \; \{ x \in U \suchthat\; \lnot M_A(x) \}. \] \pause
\item Or, \hspace{.3in} $\displaystyle \overline{A} \; = \; \{ x \in U \suchthat\; x \notin A \}$.
\end{itemize}
\end{frame}

\section{equal sets - set identities}

\begin{frame}{when are two sets equal}
\begin{itemize}
\item We'll delve deeper into this question in the next lecture (particularly into proving that sets are equal to one another.) \pause
\item But, the idea of $=$ for sets is essentially the same thing as $\iff$ \newline (or $\cong$) for statments. \pause
\item Two sets are equal if their membership criteria are logically equivalent. \pause
\item It may be more direct to think about their elements.  If $A=B$, then any element of $A$ must also be an element of $B$, and vice versa: any element of $B$ must also be an element of $A$. \pause 
\item The phrase ``any element of $A$ must also be an element of $B$'' is the same thing as $A \subseteq B$. \pause \newline
similarly (by the vice versa part) we have $B \subseteq A$. \pause
\item Recapping: \pause $A = B$ is equivalent to $A \subseteq B \; \land \; B \subseteq A$.
\end{itemize}
\end{frame}

\begin{frame}{set identities}
\begin{itemize}
\item There is a link to a handout in the video description. \pause
\item It may seem rather redundant -- every logical equivalence from Logic has a Set Theoretic identity that corresponds to it. \pause
\item The proofs of these are fairly tedious.  They all follow the same pattern: \pause 
\item Convert a set theoretic expression into a logical one -- apply one of the rules from Chapter 2 -- convert back to a set theoretic form. \pause
\item In today's activity you'll practice a few of these. \pause
\item For now, let's just examine the handout.

\end{itemize}
\end{frame}

\section{other operations}

\begin{frame}{relative difference}
\begin{itemize}
\item Everything in $A$ that's not in $B$. \pause Denoted by $A \setminus B$.\pause
\item Definition: $A \setminus B \; = \; A \cap \overline{B}$.\pause
\item Like ordinary subtraction this is {\em not} a commutative operation!\pause
\item $A \setminus B \; \neq \; B \setminus A$.
\end{itemize}
\end{frame}

\begin{frame}{symmetric difference}
\begin{itemize}
\item We can create a commutative operation by forming the union of both relative differences. \pause
\item $A \triangle B \; = \; (A \setminus B) \, \cup \, (B \setminus A)$. \pause
\item This is known as the symmetric difference.
\item Exercise on page 183 in GIAM
\end{itemize}
\end{frame}

\section{3-d solid modelling}

\begin{frame}{}
\begin{itemize}
\item OpenSCAD
\item Thingiverse
\item Some physical models
\item Henry Segerman
\end{itemize}
\end{frame}



\end{document}
