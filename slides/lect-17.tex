\documentclass[handout,landscape]{beamer}
%\documentclass[landscape]{beamer}
%\hypersetup{pdfpagemode=FullScreen}
\mode<handout>
{
  \usepackage{pgf}
  \usepackage{pgfpages}

\pgfpagesdeclarelayout{6 on 1 boxed}
{
  \edef\pgfpageoptionheight{\the\paperheight} 
  \edef\pgfpageoptionwidth{\the\paperwidth}
  \edef\pgfpageoptionborder{0pt}
}
{
  \pgfpagesphysicalpageoptions
  {%
    logical pages=6,%
    physical height=\pgfpageoptionheight,%
    physical width=\pgfpageoptionwidth%
  }
  \pgfpageslogicalpageoptions{1}
  {%
    border code=\pgfsetlinewidth{2pt}\pgfstroke,%
    border shrink=\pgfpageoptionborder,%
    resized width=.5\pgfphysicalwidth,%
    resized height=.5\pgfphysicalheight,%
    center=\pgfpoint{.25\pgfphysicalwidth}{.833\pgfphysicalheight}%
  }%
  \pgfpageslogicalpageoptions{2}
  {%
    border code=\pgfsetlinewidth{2pt}\pgfstroke,%
    border shrink=\pgfpageoptionborder,%
    resized width=.5\pgfphysicalwidth,%
    resized height=.5\pgfphysicalheight,%
    center=\pgfpoint{.75\pgfphysicalwidth}{.833\pgfphysicalheight}%
  }%
  \pgfpageslogicalpageoptions{3}
  {%
    border code=\pgfsetlinewidth{2pt}\pgfstroke,%
    border shrink=\pgfpageoptionborder,%
    resized width=.5\pgfphysicalwidth,%
    resized height=.5\pgfphysicalheight,%
    center=\pgfpoint{.25\pgfphysicalwidth}{.5\pgfphysicalheight}%
  }%
  \pgfpageslogicalpageoptions{4}
  {%
    border code=\pgfsetlinewidth{2pt}\pgfstroke,%
    border shrink=\pgfpageoptionborder,%
    resized width=.5\pgfphysicalwidth,%
    resized height=.5\pgfphysicalheight,%
    center=\pgfpoint{.75\pgfphysicalwidth}{.5\pgfphysicalheight}%
  }%
  \pgfpageslogicalpageoptions{5}
  {%
    border code=\pgfsetlinewidth{2pt}\pgfstroke,%
    border shrink=\pgfpageoptionborder,%
    resized width=.5\pgfphysicalwidth,%
    resized height=.5\pgfphysicalheight,%
    center=\pgfpoint{.25\pgfphysicalwidth}{.167\pgfphysicalheight}%
  }%
  \pgfpageslogicalpageoptions{6}
  {%
    border code=\pgfsetlinewidth{2pt}\pgfstroke,%
    border shrink=\pgfpageoptionborder,%
    resized width=.5\pgfphysicalwidth,%
    resized height=.5\pgfphysicalheight,%
    center=\pgfpoint{.75\pgfphysicalwidth}{.167\pgfphysicalheight}%
  }%
}


  \pgfpagesuselayout{6 on 1 boxed}[letterpaper, border shrink=5mm]
  \nofiles
}

\usepackage{listings}
%\lstset{language=TeX}
\usepackage{multimedia}
\usepackage[normalem]{ulem}
\usepackage{amssymb}

\usepackage{ifthen}

%\usecolortheme[named=Purple]{structure} 
%\usetheme{Copenhagen}
\usetheme{Warsaw} 
\usecolortheme{seahorse}
\useoutertheme{infolines} 
%\usetheme[height=7mm]{Rochester} 
%\setbeamertemplate{items}[ball] 
\setbeamertemplate{blocks}[rounded][shadow=true] 
%\setbeamertemplate{navigation symbols}{} 
\author{Joe Fields}
\title{Introduction to Proof} 
%\subtitle{}
\date{Lecture 17 (GIAM \S 3.4)}
\institute[SCSU]{ {\tt fieldsj1@southernct.edu} }

\newcommand{\versionNum}{$3.2$\ }

\newboolean{InTextBook}
\setboolean{InTextBook}{false}
\newboolean{InWorkBook}
\setboolean{InWorkBook}{false}
\newboolean{InHints}
\setboolean{InHints}{false}

%When this boolean is true (beginning in Section 5.1) we will use the convention
% that $0 \in \Naturals$.  If it is false we will continue to count $1$ as the smallest
%natural number (thus making Giuseppe Peano spin in his grave...)
 
\newboolean{ZeroInNaturals}

%This boolean is used to distinguish the version where we use $\sim$ rather than $\lnot$

\newboolean{LNotIsSim}

%The values of the last two booleans are set in ``switches.tex''

%\input{switches}

\let\savedlnot\lnot
\ifthenelse{\boolean{LNotIsSim}}{\renewcommand{\lnot}{\sim} }{}

%This command puts different amounts of space depending on whether we are
% in the text, the workbook or the hints & solutions manual. 
\newcommand{\twsvspace}[3]{%
 \ifthenelse{\boolean{InTextBook} }{\vspace{#1}}{%
  \ifthenelse{\boolean{InWorkBook} }{\vspace{#2}}{%
   \ifthenelse{\boolean{InHints} }{\vspace{#3}}{} %
   }%
  }%
 }


\newcommand{\wbvfill}{\ifthenelse{\boolean{InWorkBook}}{\vfill}{}}
\newcommand{\wbitemsep}{\ifthenelse{\boolean{InWorkBook} }{\rule[-24pt]{0pt}{60pt}}{}}
\newcommand{\textbookpagebreak}{\ifthenelse{\boolean{InTextBook}}{\newpage}{}}
\newcommand{\workbookpagebreak}{\ifthenelse{\boolean{InWorkBook}}{\newpage}{}}
\newcommand{\hintspagebreak}{\ifthenelse{\boolean{InHints}}{\newpage}{}}

\newcommand{\hint}[1]{\ifthenelse{\boolean{InHints}}{ {\par \hspace{12pt} \color[rgb]{0,0,1} #1 } }{}}
\newcommand{\inlinehint}[1]{\ifthenelse{\boolean{InHints}}{ { \color[rgb]{0,0,1} #1 } }{}}

%\newlength{\cwidth}
%\newcommand{\cents}{\settowidth{\cwidth}{c}%
%\divide\cwidth by2
%\advance\cwidth by-.1pt
%c\kern-\cwidth
%\vrule width .1pt depth.2ex height1.2ex
%\kern 3\cwidth}
\newcommand{\cents}{\textcent\kern 5pt}

\newcommand{\sageprompt}{ {\tt sage$>$} }
\newcommand{\tab}{\rule{20pt}{0pt}}
\newcommand{\blnk}{\rule{1.5pt}{0pt}\rule{.4pt}{1.2pt}\rule{9pt}{.4pt}\rule{.4pt}{1.2pt}\rule{1.5pt}{0pt}}
\newcommand{\suchthat}{\; \rule[-3pt]{.5pt}{13pt} \;}
\newcommand{\divides}{\!\mid\!}
\newcommand{\tdiv}{\; \mbox{div} \;}
\newcommand{\restrict}[2]{#1 \,\rule[-4pt]{.25pt}{14pt}_{\,#2}}
\newcommand{\lcm}[2]{\mbox{lcm} (#1, #2)}
\renewcommand{\gcd}[2]{\mbox{gcd} (#1, #2)}
\newcommand{\Naturals}{{\mathbb N}}
\newcommand{\Integers}{{\mathbb Z}}
\newcommand{\Znoneg}{{\mathbb Z}^{\mbox{\tiny noneg}}}
\ifthenelse{\boolean{ZeroInNaturals}}{%
  \newcommand{\Zplus}{{\mathbb Z}^+} }{%
  \newcommand{\Zplus}{{\mathbb N}} }
\newcommand{\Enoneg}{{\mathbb E}^{\mbox{\tiny noneg}}}
\newcommand{\Qnoneg}{{\mathbb Q}^{\mbox{\tiny noneg}}}
\newcommand{\Rnoneg}{{\mathbb R}^{\mbox{\tiny noneg}}}
\newcommand{\Rationals}{{\mathbb Q}}
\newcommand{\Reals}{{\mathbb R}}
\newcommand{\Complexes}{{\mathbb C}}
%\newcommand{\F2}{{\mathbb F}_{2}}
\newcommand{\relQ}{\mbox{\textsf Q}}
\newcommand{\relR}{\mbox{\textsf R}}
\newcommand{\nrelR}{\mbox{\raisebox{1pt}{$\not$}\rule{1pt}{0pt}{\textsf R}}}
\newcommand{\relS}{\mbox{\textsf S}}
\newcommand{\relA}{\mbox{\textsf A}}
\newcommand{\Dom}[1]{\mbox{Dom}(#1)}
\newcommand{\Cod}[1]{\mbox{Cod}(#1)}
\newcommand{\Rng}[1]{\mbox{Rng}(#1)}

\DeclareMathOperator\caret{\raisebox{1ex}{$\scriptstyle\wedge$}}

\newtheorem*{defi}{Definition}
\newtheorem*{exer}{Exercise}
\newtheorem{thm}{Theorem}[section]
\newtheorem*{thm*}{Theorem}
\newtheorem{lem}[thm]{Lemma}
\newtheorem*{lem*}{Lemma}
\newtheorem{cor}{Corollary}
\newtheorem{conj}{Conjecture}

\renewenvironment{proof}%
{\begin{quote} \emph{Proof:} }%
{\rule{0pt}{0pt} \newline \rule{0pt}{15pt} \hfill Q.E.D. \end{quote}}


\newcommand{\vs}{\rule{0pt}{12pt}}
\newcommand{\notimplies}{\;\not\!\!\!\implies}

%\def\mycommand{\setlength{\abovedisplayskip}{-12pt}%
%\setlength{\belowdisplayskip}{-12pt}%
%\setlength{\abovedisplayshortskip}{0pt}%
%\setlength{\belowdisplayshortskip}{0pt}}

%\let\oldselectfont\selectfont
%\def\selectfont{\oldselectfont\mycommand}

%\mycommand

\AtBeginSection[]
{
 \begin{frame}{Table of Contents} 
  \tableofcontents[currentsection]
 \end{frame}
}

%%%% SAVE %%%%
%{ %magic to get a full screen image...
%\setbeamertemplate{navigation symbols}{}  % hide navigation buttons 
%\setbeamertemplate{background canvas}{\centerline{\includegraphics 
%	[height=\paperheight]{Cantor_4.jpeg}}}
%\begin{frame}[plain]
%\rule{0pt}{0pt}
%\end{frame} 
%} %end of magic


\begin{document}

\begin{frame}[plain]
  \titlepage
\end{frame}

\section{distinguishing true from false}

\begin{frame}{intro}
\begin{itemize}
\item The need to develop a good ``lie detector.'' \pause
\item We don't want to waste our time trying to prove false statements! \pause
\item You should feel free to experiment -- try the statement out with small values for the variables. \pause
\item Use a CAS if working by hand is too difficult.
\end{itemize}
\end{frame}

\begin{frame}{a wrong distr.\ law}
\begin{itemize}
\item $\forall a,b,c \in \Reals, \; (b+c)^a \, \only<1-2>{=} \only<3->{\neq} \, b^a + c^a$ \pause
\item Unfortunately, this is not true. \pause \pause Unfortunately, that's not true either. \pause
\item Here is a table of smallish values for $a$, $b$, and $c$ which lends evidence that the statement above actually is true: \pause

\begin{tabular}{ccc|c|c|c}
\rule{6pt}{0pt} $a$ \rule{6pt}{0pt} & \rule{6pt}{0pt} $b$ \rule{6pt}{0pt} & \rule{6pt}{0pt} $c$ \rule{6pt}{0pt} & \rule{6pt}{0pt} $(b+c)^a$ \rule{6pt}{0pt} & \rule{6pt}{0pt} $b^a + c^a$ \rule{6pt}{0pt} & \rule{6pt}{0pt} equal? \rule{6pt}{0pt} \\ \hline
\rule[-4pt]{0pt}{20pt} $1$ & $1$ & $2$ & $3$ & $3$ & $\checkmark$ \\
\rule[-4pt]{0pt}{20pt} $1$ & $3$ & $4$ & $7$ & $7$ & $\checkmark$ \\
\rule[-4pt]{0pt}{20pt} $2$ & $0$ & $3$ & $9$ & $9$ & $\checkmark$ \\
\rule[-4pt]{0pt}{20pt} $5$ & $2$ & $0$ & $32$ & $32$ & $\checkmark$ \\
\end{tabular}
\pause

\item Try to find some values of $a$, $b$ and $c$ that {\em do} provide a counterexample.

\end{itemize}
\end{frame}

\begin{frame}{another false ``theorem''}
\begin{itemize}
\item $\forall a,b,c \in \Integers, \; a \mid bc \; \only<1-2>{\implies} \only<3->{\,\notimplies} \; ( a \mid b \, \lor \, a \mid c )$ \pause
\item Also, not true.\pause \pause
\item Another pile of misleading evidence: \pause

\begin{center}
\begin{tabular}{ccc|c|c}
\rule[-4pt]{0pt}{20pt}\rule{6pt}{0pt}$a$\rule{6pt}{0pt} & \rule{6pt}{0pt}$b$\rule{6pt}{0pt} & \rule{6pt}{0pt}$c$\rule{6pt}{0pt} & \rule{6pt}{0pt}$a \divides bc$ ?\rule{6pt}{0pt} & \rule{6pt}{0pt}$ a \divides b \, \lor \, a \divides c $ ? \rule{6pt}{0pt}\\ \hline
\rule[-4pt]{0pt}{20pt}2 & 7 & 6 & yes & yes \\  
\rule[-4pt]{0pt}{20pt}2 & 4 & 5 & yes & yes \\  
\rule[-4pt]{0pt}{20pt}3 & 12 & 11 & yes & yes \\
\rule[-4pt]{0pt}{20pt}3 & 5 & 15 & yes & yes \\
\rule[-4pt]{0pt}{20pt}5 & 4 & 15 & yes & yes \\
\rule[-4pt]{0pt}{20pt}5 & 10 & 3 & yes & yes \\
\rule[-4pt]{0pt}{20pt}7 & 2 & 14 & yes & yes \\
\end{tabular}
\end{center}
\end{itemize}
\end{frame}

\begin{frame}{why is he trying to lead us astray?}
\begin{itemize}
\item $\forall n \in \Integers^+, \; n^2 - 79n + 1601 \; \mbox{is prime} $ \pause
\item You guessed it\textellipsis \pause
\item Let's use a computer this time.
\end{itemize}
\end{frame}

\section{disproofs}

\begin{frame}{philosophy}
\begin{itemize}
\item Disproving a statement $X$ is the same thing as {\em proving} $\lnot X$.\pause
\item To disprove a universally quantified statement \pause 
\[ \forall x, \; P(x). \] \pause
We need to prove the negation, which is existentially quantified, \pause
\[ \exists x, \; \lnot P(x). \] \pause
\item And {\em vice versa}. \pause
\item So disproving an existential is just a regular old proof. \pause
\item And disproving a universal is just about finding an example \pause \newline
(Because of the role it's playing we call it a {\em counterexample}.)
\end{itemize}
\end{frame}

\begin{frame}{example}
\begin{itemize}
\item Remember how we checked out whether $p\# + 1$ was always prime on Sage? \pause
\item Disprove the following ``theorem''
\begin{thm*}
\[ \forall p \in \mbox{Primes}, \; p\# + 1 \; \mbox{is prime.} \]
\end{thm*}
\pause

\item This statement is false.  A counterexample is $p=13$, since $13\# + 1 \; = \; 30031$ and $30031 \, = \, 59\cdot 509$ is composite. 
\end{itemize}
\end{frame}

\begin{frame}{another example}
\begin{itemize}
\item A sum is said to be {\em alternating} if the signs of successive terms are always opposite. \pause
\item For example $1-4+9-16+25-36$ is an alternating sum. \pause
\item Let $A(n)$ be the alternating sum of factorials, starting with $n!$ and descending to $\pm 1!$\pause
\begin{thm*}
\[ \forall n \in \Integers, \; n>2 \; \implies \; A(n) \; \mbox{is prime} \]
\end{thm*}
\pause
\item First let's explore in sage. \pause
\item Here's the formal disproof: \pause \newline
This statement is false.  A counterexample is $n=9$, where the alternating sum is $9!-8!+7!- \cdots +1! \; = \; 326981$ which factors as $79\cdot 4139$.
\end{itemize}
\end{frame}

\begin{frame}{A final exploration}
\begin{itemize}
\item The following two statements give alternative possibilities since they are negations. \pause
\begin{itemize}
\item The sum of any two irrational numbers is irrational. \pause
\item There are two irrational numbers whose sum is rational. \pause
\end{itemize}
\item Hint: If a number is irrational, isn't its negative also irrational?  \pause That's actually a pretty huge hint.\pause
\item Okay, when you put it that way\textellipsis

\end{itemize}
\end{frame}


\begin{frame}{A final exploration}
\begin{itemize}

\item The sum of any two irrational numbers is irrational. \pause \newline
This statement is false since $\sqrt{3}$ and $-\sqrt{3}$ are both irrational, but their sum is $0$ which is clearly rational. \pause

\begin{thm*}
There are two irrational numbers whose sum is rational. 
\end{thm*}
\pause 
For example, $1+\sqrt{2}$ and $1-\sqrt{2}$ are one such pair. 
\end{itemize}
\end{frame}


\begin{frame}{A final word}
\begin{itemize}

\item Caution:  You very seldom prove things by providing an example! \pause 
\item That only worked in the last example because the statement was existentially quantified. \pause
\item Disproof by counterexample happens all the time! \pause 
\item Proof by example is so uncommon that it is often used as a criticism.

\end{itemize}
\end{frame}



\end{document}
