\ifdefined\ishandout
  \documentclass[handout,landscape]{beamer} 
\else
  \documentclass[landscape]{beamer}
\fi

%\hypersetup{pdfpagemode=FullScreen} %Enabling this option will cause the slides to go full-screen on opening

\mode<handout>
{
  \usepackage{pgf}
  \usepackage{pgfpages}

\pgfpagesdeclarelayout{6 on 1 boxed}
{
  \edef\pgfpageoptionheight{\the\paperheight} 
  \edef\pgfpageoptionwidth{\the\paperwidth}
  \edef\pgfpageoptionborder{0pt}
}
{
  \pgfpagesphysicalpageoptions
  {%
    logical pages=6,%
    physical height=\pgfpageoptionheight,%
    physical width=\pgfpageoptionwidth%
  }
  \pgfpageslogicalpageoptions{1}
  {%
    border code=\pgfsetlinewidth{2pt}\pgfstroke,%
    border shrink=\pgfpageoptionborder,%
    resized width=.5\pgfphysicalwidth,%
    resized height=.5\pgfphysicalheight,%
    center=\pgfpoint{.25\pgfphysicalwidth}{.833\pgfphysicalheight}%
  }%
  \pgfpageslogicalpageoptions{2}
  {%
    border code=\pgfsetlinewidth{2pt}\pgfstroke,%
    border shrink=\pgfpageoptionborder,%
    resized width=.5\pgfphysicalwidth,%
    resized height=.5\pgfphysicalheight,%
    center=\pgfpoint{.75\pgfphysicalwidth}{.833\pgfphysicalheight}%
  }%
  \pgfpageslogicalpageoptions{3}
  {%
    border code=\pgfsetlinewidth{2pt}\pgfstroke,%
    border shrink=\pgfpageoptionborder,%
    resized width=.5\pgfphysicalwidth,%
    resized height=.5\pgfphysicalheight,%
    center=\pgfpoint{.25\pgfphysicalwidth}{.5\pgfphysicalheight}%
  }%
  \pgfpageslogicalpageoptions{4}
  {%
    border code=\pgfsetlinewidth{2pt}\pgfstroke,%
    border shrink=\pgfpageoptionborder,%
    resized width=.5\pgfphysicalwidth,%
    resized height=.5\pgfphysicalheight,%
    center=\pgfpoint{.75\pgfphysicalwidth}{.5\pgfphysicalheight}%
  }%
  \pgfpageslogicalpageoptions{5}
  {%
    border code=\pgfsetlinewidth{2pt}\pgfstroke,%
    border shrink=\pgfpageoptionborder,%
    resized width=.5\pgfphysicalwidth,%
    resized height=.5\pgfphysicalheight,%
    center=\pgfpoint{.25\pgfphysicalwidth}{.167\pgfphysicalheight}%
  }%
  \pgfpageslogicalpageoptions{6}
  {%
    border code=\pgfsetlinewidth{2pt}\pgfstroke,%
    border shrink=\pgfpageoptionborder,%
    resized width=.5\pgfphysicalwidth,%
    resized height=.5\pgfphysicalheight,%
    center=\pgfpoint{.75\pgfphysicalwidth}{.167\pgfphysicalheight}%
  }%
}


  \pgfpagesuselayout{6 on 1 boxed}[letterpaper, border shrink=5mm]
  \nofiles
}

\usepackage{listings}
\usepackage{multimedia}
\usepackage[normalem]{ulem}
\usepackage{ifthen}

\usetheme{Warsaw} 
\usecolortheme{seahorse}
\useoutertheme{infolines} 

\setbeamertemplate{blocks}[rounded][shadow=true] 

\author{Joe Fields}
\title{Introduction to Proof} 

\date{Lecture 21 (GIAM \S 4.2) \newline containment}
\institute[SCSU]{ {\tt fieldsj1@southernct.edu} }


\newlength{\cwidth}
\newcommand{\cents}{\settowidth{\cwidth}{c}%
\divide\cwidth by2
\advance\cwidth by-.1pt
c\kern-\cwidth
\vrule width .1pt depth.2ex height1.2ex
\kern\cwidth}

\newcommand{\sageprompt}{ {\tt sage$>$} }
\newcommand{\tab}{\rule{20pt}{0pt}}
\newcommand{\blnk}{\rule{1.5pt}{0pt}\rule{.4pt}{1.2pt}\rule{9pt}{.4pt}\rule{.4pt}{1.2pt}\rule{1.5pt}{0pt}}
\newcommand{\suchthat}{\; \rule[-3pt]{.25pt}{13pt} \;}
\newcommand{\divides}{\!\mid\!}
\newcommand{\tdiv}{\; \mbox{div} \;}
\newcommand{\restrict}[2]{#1 \,\rule[-4pt]{.125pt}{14pt}_{\,#2}}
\newcommand{\lcm}[2]{\mbox{lcm} (#1, #2)}
\renewcommand{\gcd}[2]{\mbox{gcd} (#1, #2)}
\newcommand{\Naturals}{{\mathbb N}}
\newcommand{\Integers}{{\mathbb Z}}
\newcommand{\Znoneg}{{\mathbb Z}^{\mbox{\tiny noneg}}}
\newcommand{\Enoneg}{{\mathbb E}^{\mbox{\tiny noneg}}}
\newcommand{\Qnoneg}{{\mathbb Q}^{\mbox{\tiny noneg}}}
\newcommand{\Rnoneg}{{\mathbb R}^{\mbox{\tiny noneg}}}
\newcommand{\Rationals}{{\mathbb Q}}
\newcommand{\Reals}{{\mathbb R}}
\newcommand{\Complexes}{{\mathbb C}}
%\newcommand{\F2}{{\mathbb F}_{2}}
\newcommand{\relQ}{\mbox{\textsf Q}}
\newcommand{\relR}{\mbox{\textsf R}}
\newcommand{\nrelR}{\mbox{\raisebox{1pt}{$\not$}\rule{1pt}{0pt}{\textsf R}}}
\newcommand{\relS}{\mbox{\textsf S}}
\newcommand{\relA}{\mbox{\textsf A}}
\newcommand{\Dom}[1]{\mbox{Dom}(#1)}
\newcommand{\Cod}[1]{\mbox{Cod}(#1)}
\newcommand{\Rng}[1]{\mbox{Rng}(#1)}

\DeclareMathOperator\caret{\raisebox{1ex}{$\scriptstyle\wedge$}}

\newtheorem*{defi}{Definition}
\newtheorem*{exer}{Exercise}
\newtheorem{thm}{Theorem}[section]
\newtheorem*{thm*}{Theorem}
\newtheorem{lem}[thm]{Lemma}
\newtheorem{cor}{Corollary}
\newtheorem{conj}{Conjecture}

\renewenvironment{proof}%
{\begin{quote} \emph{Proof:} }%
{\rule{0pt}{0pt} \newline \rule{0pt}{15pt} \hfill Q.E.D. \end{quote}}


\newcommand{\vs}{\rule{0pt}{12pt}}
\newcommand{\notimplies}{\;\not\!\!\!\implies}

\AtBeginSection[]
{
 \begin{frame}{Table of Contents} 
  \tableofcontents[currentsection]
 \end{frame}
}

%%%% SAVE %%%%
%{ %magic to get a full screen image...
%\setbeamertemplate{navigation symbols}{}  % hide navigation buttons 
%\setbeamertemplate{background canvas}{\centerline{\includegraphics 
%	[height=\paperheight]{Cantor_4.jpeg}}}
%\begin{frame}[plain]
%\rule{0pt}{0pt}
%\end{frame} 
%} %end of magic


\begin{document}

\begin{frame}[plain]
  \titlepage
\end{frame}

\section{element vs subset}

\begin{frame}{}
\begin{itemize}
\item Sometimes sets contain other sets. \pause
\item It's pretty clear that we shouldn't have a set that contained itself.\pause
\item Top-level commas are the ones inside the outer french braces, but not inside any other braces.\pause
\item Here's the power set of $\{1,2\}$ with the top-level commas indicated: \pause

\setlength{\tabcolsep}{0pt}
\begin{tabular}{clclclc}
$\left\{ \emptyset \right.$ & , & $\{1\}$ & , & $\{2\}$ & , & $\left. \{1,2\} \right\}$ \\
 & $\uparrow$ & & $\uparrow$ & & $\uparrow$ & \\
\end{tabular}
\setlength{\tabcolsep}{6pt}
\pause
\item The things in between the top-level commas are the {\em elements} of the set.
\end{itemize}
\end{frame}

\begin{frame}{an example}
\begin{itemize}
\item Use the ``$\in$'' sign for elements. \pause
\item Suppose $A = \{ 1, 2, \{3, 4\}\}$. \pause
\item It's correct to write $1 \in A$. \pause
\item It's also correct to write $\{3, 4\} \in A$. \pause
\item But it's wrong to write $3 \in A$. \pause (It's not in $A$, it's in one of $A$'s elements!) 
\end{itemize}
\end{frame}

\begin{frame}{inclusion vs membership}
\begin{itemize}
\item The ``$\in$'' sign denotes a relation called ``membership.'' \pause
\item Another way something can be in a set is called ``containment.'' \pause
\item A set $A$ is contained in a set $B$ if all of $A$'s elements are members of $B$. \pause
\item The symbol for ``is contained in'' is $\subseteq$. \pause
\item The things on either side of $\subseteq$ must both be sets. \pause \newline
$3 \in \Naturals$ is a sensible mathematical statement, \pause \newline
$3 \subseteq \Naturals$ is nonsensical. \pause
\item Exercise on page 172.
\end{itemize}
\end{frame}

\begin{frame}{variations on a theme}
\begin{itemize}
\item The symbol $\subseteq$ is a rounded version of $\leq$. \pause
\item Hungry alligators. \pause
\item If $A$ is definitely completely inside of $B$ we write $A \subset B$. \pause
\item These symbols are reversible.
\item If $A \supseteq B$ we say $A$ is a {\em superset} of $B$.
\end{itemize}
\end{frame}

\section{containment and implications}

\begin{frame}{more on the logic-set correspondence}
\begin{itemize}
\item Consider the numbers between 10 and 30 that are $1\mod 4$: \pause \newline
\[ A = \{13, 17, 21, 25, 29\}. \] \pause
\item The membership criterion for $A$ is $M_A = $ ``$10 \leq x \leq 30 \; \land \; x \mod 4 = 1  $''
\item Next consider the set of all odd numbers in the same range: \pause \newline
\[ B = \{11, 13, 15, 17, 19, 21, 23, 25, 27, 29 \}. \] \pause
\item Which set is contained in the other? \pause
\item Clearly, $A \subseteq B$. \pause
\item How are their corresponding membership criteria related? \pause
\item $M_A \implies M_B$ \hspace{.1in} or \hspace{.1in} $M_B \implies M_A$ ? \pause
\item ???
\end{itemize}
\end{frame}

\begin{frame}{strength of statements}
\begin{itemize}
\item When a statement $P$ implies a statement $Q$ we can think of $P$ as being ``stronger.'' \pause
\item Compare ``being a 4th power'' to ``being a square.'' \pause
\item More formally, suppose $P(x) = $ ``$\exists k \in Integers, \; x=k^4$,'' \pause \newline
and $Q(x) = $ ``$\exists k \in \Integers, \; x=k^2$,'' \pause 
\item $\forall x \in \Integers, \; P(x) \implies Q(x)$ \pause
\item If you're a 4th power you're automatically a square -- being a fourth power is a bigger deal!\pause
\item So, if $A = \{0, 1, 16, 81, 256 \ldots \}$ is the set of fourth powers \pause \newline
and $B = \{0, 1, 4, 9, 16, \ldots \}$ is the set of squares. \pause
\item Being in $A$ automatically gets you into $B$. \pause
\item Think about exclusive clubs. \pause The smaller set is actually the ``stronger'' thing. \pause
\item Groucho Marx
\end{itemize}
\end{frame}

\section{computing}

\begin{frame}{CoCalc}
\begin{itemize}
\item Sage sets have member functions for checking the subset/superset relation
\item There's an infix operator ``in'' that does the job of $\in$.
\end{itemize}
\end{frame}

\section{typesetting}

\begin{frame}{Overleaf}
\begin{itemize}
\item Curly braces appear as part of the language constructs, so how do we get actual french braces to appear. 
\item Getting braces to match the size of the stuff inside.
\item The names of commands for $\subseteq$ and friends. (amssymb package)
\end{itemize}
\end{frame}

\end{document}
