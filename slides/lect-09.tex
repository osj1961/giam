%\documentclass[handout,landscape]{beamer}
\documentclass[landscape]{beamer}
%\hypersetup{pdfpagemode=FullScreen}
\mode<handout>
{
  \usepackage{pgf}
  \usepackage{pgfpages}

\pgfpagesdeclarelayout{6 on 1 boxed}
{
  \edef\pgfpageoptionheight{\the\paperheight} 
  \edef\pgfpageoptionwidth{\the\paperwidth}
  \edef\pgfpageoptionborder{0pt}
}
{
  \pgfpagesphysicalpageoptions
  {%
    logical pages=6,%
    physical height=\pgfpageoptionheight,%
    physical width=\pgfpageoptionwidth%
  }
  \pgfpageslogicalpageoptions{1}
  {%
    border code=\pgfsetlinewidth{2pt}\pgfstroke,%
    border shrink=\pgfpageoptionborder,%
    resized width=.5\pgfphysicalwidth,%
    resized height=.5\pgfphysicalheight,%
    center=\pgfpoint{.25\pgfphysicalwidth}{.833\pgfphysicalheight}%
  }%
  \pgfpageslogicalpageoptions{2}
  {%
    border code=\pgfsetlinewidth{2pt}\pgfstroke,%
    border shrink=\pgfpageoptionborder,%
    resized width=.5\pgfphysicalwidth,%
    resized height=.5\pgfphysicalheight,%
    center=\pgfpoint{.75\pgfphysicalwidth}{.833\pgfphysicalheight}%
  }%
  \pgfpageslogicalpageoptions{3}
  {%
    border code=\pgfsetlinewidth{2pt}\pgfstroke,%
    border shrink=\pgfpageoptionborder,%
    resized width=.5\pgfphysicalwidth,%
    resized height=.5\pgfphysicalheight,%
    center=\pgfpoint{.25\pgfphysicalwidth}{.5\pgfphysicalheight}%
  }%
  \pgfpageslogicalpageoptions{4}
  {%
    border code=\pgfsetlinewidth{2pt}\pgfstroke,%
    border shrink=\pgfpageoptionborder,%
    resized width=.5\pgfphysicalwidth,%
    resized height=.5\pgfphysicalheight,%
    center=\pgfpoint{.75\pgfphysicalwidth}{.5\pgfphysicalheight}%
  }%
  \pgfpageslogicalpageoptions{5}
  {%
    border code=\pgfsetlinewidth{2pt}\pgfstroke,%
    border shrink=\pgfpageoptionborder,%
    resized width=.5\pgfphysicalwidth,%
    resized height=.5\pgfphysicalheight,%
    center=\pgfpoint{.25\pgfphysicalwidth}{.167\pgfphysicalheight}%
  }%
  \pgfpageslogicalpageoptions{6}
  {%
    border code=\pgfsetlinewidth{2pt}\pgfstroke,%
    border shrink=\pgfpageoptionborder,%
    resized width=.5\pgfphysicalwidth,%
    resized height=.5\pgfphysicalheight,%
    center=\pgfpoint{.75\pgfphysicalwidth}{.167\pgfphysicalheight}%
  }%
}


  \pgfpagesuselayout{6 on 1 boxed}[letterpaper, border shrink=5mm]
  \nofiles
}

\usepackage{listings}
%\lstset{language=TeX}
\usepackage{multimedia}
\usepackage[normalem]{ulem}
\usepackage{amssymb}

%\usecolortheme[named=Purple]{structure} 
%\usetheme{Copenhagen}
\usetheme{Warsaw} 
\usecolortheme{seahorse}
\useoutertheme{infolines} 
%\usetheme[height=7mm]{Rochester} 
%\setbeamertemplate{items}[ball] 
\setbeamertemplate{blocks}[rounded][shadow=true] 
%\setbeamertemplate{navigation symbols}{} 
\author{Joe Fields}
\title{Introduction to Proof} 
%\subtitle{}
\date{Lecture 9 (GIAM \S 2.2)}
\institute[SCSU]{ {\tt fieldsj1@southernct.edu} }


\newlength{\cwidth}
\newcommand{\cents}{\settowidth{\cwidth}{c}%
\divide\cwidth by2
\advance\cwidth by-.1pt
c\kern-\cwidth
\vrule width .1pt depth.2ex height1.2ex
\kern\cwidth}

\newcommand{\sageprompt}{ {\tt sage$>$} }
\newcommand{\tab}{\rule{20pt}{0pt}}
\newcommand{\blnk}{\rule{1.5pt}{0pt}\rule{.4pt}{1.2pt}\rule{9pt}{.4pt}\rule{.4pt}{1.2pt}\rule{1.5pt}{0pt}}
\newcommand{\suchthat}{\; \rule[-3pt]{.25pt}{13pt} \;}
\newcommand{\divides}{\!\mid\!}
\newcommand{\tdiv}{\; \mbox{div} \;}
\newcommand{\restrict}[2]{#1 \,\rule[-4pt]{.125pt}{14pt}_{\,#2}}
\newcommand{\lcm}[2]{\mbox{lcm} (#1, #2)}
\renewcommand{\gcd}[2]{\mbox{gcd} (#1, #2)}
\newcommand{\Naturals}{{\mathbb N}}
\newcommand{\Integers}{{\mathbb Z}}
\newcommand{\Znoneg}{{\mathbb Z}^{\mbox{\tiny noneg}}}
\newcommand{\Enoneg}{{\mathbb E}^{\mbox{\tiny noneg}}}
\newcommand{\Qnoneg}{{\mathbb Q}^{\mbox{\tiny noneg}}}
\newcommand{\Rnoneg}{{\mathbb R}^{\mbox{\tiny noneg}}}
\newcommand{\Rationals}{{\mathbb Q}}
\newcommand{\Reals}{{\mathbb R}}
\newcommand{\Complexes}{{\mathbb C}}
%\newcommand{\F2}{{\mathbb F}_{2}}
\newcommand{\relQ}{\mbox{\textsf Q}}
\newcommand{\relR}{\mbox{\textsf R}}
\newcommand{\nrelR}{\mbox{\raisebox{1pt}{$\not$}\rule{1pt}{0pt}{\textsf R}}}
\newcommand{\relS}{\mbox{\textsf S}}
\newcommand{\relA}{\mbox{\textsf A}}
\newcommand{\Dom}[1]{\mbox{Dom}(#1)}
\newcommand{\Cod}[1]{\mbox{Cod}(#1)}
\newcommand{\Rng}[1]{\mbox{Rng}(#1)}

\DeclareMathOperator\caret{\raisebox{1ex}{$\scriptstyle\wedge$}}

\newtheorem*{defi}{Definition}
\newtheorem*{exer}{Exercise}
\newtheorem{thm}{Theorem}[section]
\newtheorem*{thm*}{Theorem}
\newtheorem{lem}[thm]{Lemma}
\newtheorem{cor}{Corollary}
\newtheorem{conj}{Conjecture}

\renewenvironment{proof}%
{\begin{quote} \emph{Proof:} }%
{\rule{0pt}{0pt} \newline \rule{0pt}{15pt} \hfill Q.E.D. \end{quote}}


\newcommand{\vs}{\rule{0pt}{12pt}}

\AtBeginSection[]
{
 \begin{frame}{Table of Contents} 
  \tableofcontents[currentsection]
 \end{frame}
}

%%%% SAVE %%%%
%{ %magic to get a full screen image...
%\setbeamertemplate{navigation symbols}{}  % hide navigation buttons 
%\setbeamertemplate{background canvas}{\centerline{\includegraphics 
%	[height=\paperheight]{Cantor_4.jpeg}}}
%\begin{frame}[plain]
%\rule{0pt}{0pt}
%\end{frame} 
%} %end of magic


\begin{document}

\begin{frame}[plain]
  \titlepage
\end{frame}


\section{conditionals}

\begin{frame}{A implies B}
\begin{itemize}
\item ``If they have eggs, get a dozen'' \pause
\item The ``If A then B'' construct we saw in the section on algorithms was for {\em conditional execution}. \pause
\item For instance, ``If the second input is $0$ print an error message and exit.'' \pause
\item Here, in the section on Logic we want to think of it instead as a way to hook statements together to get new statements. \pause
\item We use the symbol $\implies$ as infix notation for this ``hookup.'' \pause
\item The result ($A \implies B$) must be a Boolean. \pause Either true or false. 
\end{itemize}
\end{frame}

\begin{frame}{terminology}
\begin{itemize}
\item The statments on either side of the $\implies$ arrow have names. \pause
\item $A$ is the {\em antecedant}. \pause \newline
Many people (me included) call it the ``if part.'' \pause
\item $B$ is known as the {\em consequent}. \pause \newline 
a.k.a.\ the ``then part.'' \pause
\item It also sounds more grown up to call the whole statement a {\em conditional}. \pause \newline 
As opposed to an ``if-then sentence.'' \pause
\item You can use the word ``implies'' as a direct substitute for the $\implies$ arrow.
\end{itemize}
\end{frame}

\begin{frame}{truth table}
\begin{itemize}
\item When a conditional is true, truth of the antecedant will force truth of the consequent.\pause
\item I often mentally convert $A \implies B$ into ``When $A$ is true, it forces $B$ to be true.\pause
\begin{center}
\begin{tabular}{c|c||c}
$A$ & $B$ & $A \implies B$ \\ \hline
T & T & \uncover<4->{T} \\
T & $\phi$ & \uncover<5->{$\phi$} \\
 $\phi$ & T & \uncover<6->{T} \\
 $\phi$ & $\phi$ & \uncover<6->{T} \\
\end{tabular}
\end{center}
\end{itemize}


\end{frame}

\begin{frame}{the equivalent disjunction}
\begin{itemize}
\item What other statement would have the same truth values as $A \implies B$? \pause
\item Since there are 3 T's and only 1 $\phi$, we need a disjunction. \pause (the OR connector). \pause
\item It's clearly not just $A \lor B$, that would have its one $\phi$ in row 4 not row 2. \pause
\item We did this in the activity for lecture 8. \pause 
\item It's $\lnot A \lor B$. 
\end{itemize}
\end{frame}

\begin{frame}{or else}
\begin{itemize}
\item Have you ever heard someone use the word ``or'' when issuing a threat? \pause
\item Usually, threats are if-then sorts of things -- if you do or don't do something then some terrible consequence will ensue. \pause
\item ``If you stay out past 11, you'll be grounded.'' \pause
\item ``Get home by 11, or you're grounded.'' \pause
\item Notice that if the first is $A \implies B$, the other is $\lnot A \lor B$.
\end{itemize}
\end{frame}

\begin{frame}{vacuous truth}
\begin{itemize}
\item A conditional sentence is {\em vacuously true} when the antecedant is false. \pause
\item In that scenario, it doesn't matter what the truth values of the consequent are. \pause
\item We're talking about the bottom two rows of the conditional's truth table. \pause
\item You can prove that an if-then statement is true by just showing that the if part {\em never happens}!
\end{itemize}
\end{frame}

\section{biconditionals}

\begin{frame}{exactly when}
\begin{itemize}
\item When $ (A \implies B) \land (B \implies A) $ \pause
\item we write $A \iff B$. \pause
\item This is known as the {\em biconditional}.\pause
\item Of course, we've seen this symbol already as shorthand for ``exactly when'' in definitions.
\end{itemize}
\end{frame}

\begin{frame}{truth table}
\begin{center}
\begin{tabular}{c|c||c}
$A$ & $B$ & $A \iff B$ \\ \hline
T & T & T \\
T & $\phi$ & $\phi$ \\
 $\phi$ & T & $\phi$ \\
 $\phi$ & $\phi$ & T \\
\end{tabular}
\end{center}
\pause
\begin{itemize}
\item Perhaps we should note that this is what your Mom meant when she said \newline
``You can have dessert if you finish your vegetables'' when you were little.
\end{itemize}

\end{frame}


\section{other related statements}

\begin{frame}{converse}
\begin{itemize}
\item The second conditional we see in the definition of $\iff$ \pause
\item When $A \implies B$ is given, it's {\em converse} is $B \implies A$. \pause
\item This is a two-way street so it might be better to say ``the two conditionals are converses.'' \pause
\item It's a common error to mistake a conditional for its converse.
\end{itemize}
\end{frame}

\begin{frame}{lawnmower man}
\begin{itemize}
\item Consider the predicates $A =$ ``There's gas in the tank'' and $B =$ ``the motor runs.'' \pause
\item $A \implies B$ would mean ``If there's gas in the tank, the motor will run.''\pause
\item $B \implies A$ would mean ``If the motor runs, there is gas in the tank.''\pause
\item Are these the same? \pause
\item $A \implies B$ is an expression of confidence.  Maybe I know my lawnmower is well-maintained, so I'm sure that -- provided it's got fuel -- it will start.\pause
\item $B \implies A$ is more about objective reality, it's saying that if the motor is already running, we know there must at least be {\em some} gas in the tank! \pause (Unless we have one of those scary demonically possessed lawnmowers\textellipsis) 
\end{itemize}
\end{frame}

\begin{frame}{truth tables}
\begin{center}
\begin{tabular}{ccc}
\begin{tabular}{c|c||c}
$A$ & $B$ & $A \implies B$ \\ \hline
T & T & T \\
T & $\phi$ & $\phi$ \\
 $\phi$ & T & T\\
 $\phi$ & $\phi$ & T \\
\end{tabular}
& \rule{36pt}{0pt} &
\begin{tabular}{c|c||c}
$A$ & $B$ & $B \implies A$ \\ \hline
T & T & T \\
T & $\phi$ & T \\
 $\phi$ & T & $\phi$ \\
 $\phi$ & $\phi$ & T \\
\end{tabular}
\end{tabular}
\end{center}
\pause

\begin{itemize}
\item Locate the rows where each is vacuously true. \pause
\item Notice that if we ``and'' the two of these we get ``exactly when''
\end{itemize}
\end{frame}

\begin{frame}{inverse}
\begin{itemize}
\item Another common mistake is to think that we might get the negation of a conditional like so: \pause

\[ \lnot ( A \implies B ) \quad \mbox{is} \quad \lnot A \implies \lnot B \]

\pause

\item but that is just wrong! \pause
\item Why? \pause
\item We call $\lnot A \implies \lnot B$ the {\em inverse} of $A \implies B$. \pause \newline
(Or say that ``$A \implies B$ and $\lnot A \implies \lnot B$ are inverses.'')

\end{itemize}
\end{frame}

\begin{frame}{btw}
\begin{itemize}
\item So if the inverse {\em isn't} the right way to negate a conditional, what is? \pause
\item Think about the truth table. \pause

\begin{center}
\begin{tabular}{c|c||c|c}
$A$ & $B$ & $A \implies B$ & $\lnot (A \implies B)$ \\ \hline
T & T & T & \uncover<4->{$\phi$} \\
T & $\phi$ & $\phi$ & \uncover<5->{T}  \\
 $\phi$ & T & T & \uncover<6->{$\phi$}\\
 $\phi$ & $\phi$ & T & \uncover<7->{$\phi$}\\
\end{tabular}
\end{center}
\pause \pause \pause \pause \pause
\item One T in row 2. \pause
\item Which ``and'' sentence is that? \pause
\item Calling \xout{BS} out an error.
\end{itemize}
\end{frame}

\begin{frame}{two wrongs}
\begin{itemize}

\item If you both converse-ify and inverse-ify a conditional you get something with exactly the same meaning. \pause
\item In this instance two wrongs {\em do} make a right! \pause
\item The {\em contrapositive} of the conditional $A \implies B$ is $\lnot B \implies \lnot A$. 
\end{itemize}
\end{frame}

\begin{frame}{contrapositives}
\begin{itemize}
\item An example: ``If it just rained, the ground will be wet'' \newline
and ``If the ground isn't wet, it didn't just rain.'' \pause
\item Two conditionals that are contrapositives have identical patterns of $T$'s and $\phi$'s in their truth tables.  We call such things {\em equivalent}. \pause
\item Often one or the other is more usable in a proof. \pause
\item ``If you're stuck, try taking the contrapositive.''
\end{itemize}
\end{frame}

\begin{frame}{truth table}

\begin{center}
\begin{tabular}{c|c||c|c}
$A$ & $B$ & $A \implies B$ & $\lnot B \implies \lnot A$ \\ \hline
T & T & T & \uncover<3->{T} \\
T & $\phi$ & $\phi$ & \uncover<7->{$\phi$}  \\
 $\phi$ & T & T & \uncover<3->{T}\\
 $\phi$ & $\phi$ & T & \uncover<5->{T}\\
\end{tabular}
\end{center}

\pause
\begin{itemize}
\item Which positions are vacuously true? \pause \pause
\item When is it true because both sides are true? \pause \pause
\item The remaining position must be false. \pause \pause \newline
(It's a conditional after all -- they always have 3 T's and 1 $\phi$.)
\end{itemize}
\end{frame}

\section{relationships}

\begin{frame}{converse, inverse and contrapositive}
\begin{itemize}
\item Look at Table 2.1 in GIAM. \pause
\item Advice: watch out for thinking about if-then's in terms of causality.
\end{itemize}
\end{frame}

\end{document}
