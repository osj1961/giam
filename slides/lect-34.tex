\ifdefined\ishandout
  \documentclass[handout,landscape]{beamer} 
\else
  \documentclass[landscape]{beamer}
\fi

%\hypersetup{pdfpagemode=FullScreen} %Enabling this option will cause the slides to go full-screen on opening

\mode<handout>
{
  \usepackage{pgf}
  \usepackage{pgfpages}

\pgfpagesdeclarelayout{6 on 1 boxed}
{
  \edef\pgfpageoptionheight{\the\paperheight} 
  \edef\pgfpageoptionwidth{\the\paperwidth}
  \edef\pgfpageoptionborder{0pt}
}
{
  \pgfpagesphysicalpageoptions
  {%
    logical pages=6,%
    physical height=\pgfpageoptionheight,%
    physical width=\pgfpageoptionwidth%
  }
  \pgfpageslogicalpageoptions{1}
  {%
    border code=\pgfsetlinewidth{1pt}\pgfstroke,%
    border shrink=\pgfpageoptionborder,%
    resized width=.5\pgfphysicalwidth,%
    resized height=.5\pgfphysicalheight,%
    center=\pgfpoint{.25\pgfphysicalwidth}{.833\pgfphysicalheight}%
  }%
  \pgfpageslogicalpageoptions{2}
  {%
    border code=\pgfsetlinewidth{1pt}\pgfstroke,%
    border shrink=\pgfpageoptionborder,%
    resized width=.5\pgfphysicalwidth,%
    resized height=.5\pgfphysicalheight,%
    center=\pgfpoint{.75\pgfphysicalwidth}{.833\pgfphysicalheight}%
  }%
  \pgfpageslogicalpageoptions{3}
  {%
    border code=\pgfsetlinewidth{1pt}\pgfstroke,%
    border shrink=\pgfpageoptionborder,%
    resized width=.5\pgfphysicalwidth,%
    resized height=.5\pgfphysicalheight,%
    center=\pgfpoint{.25\pgfphysicalwidth}{.5\pgfphysicalheight}%
  }%
  \pgfpageslogicalpageoptions{4}
  {%
    border code=\pgfsetlinewidth{1pt}\pgfstroke,%
    border shrink=\pgfpageoptionborder,%
    resized width=.5\pgfphysicalwidth,%
    resized height=.5\pgfphysicalheight,%
    center=\pgfpoint{.75\pgfphysicalwidth}{.5\pgfphysicalheight}%
  }%
  \pgfpageslogicalpageoptions{5}
  {%
    border code=\pgfsetlinewidth{1pt}\pgfstroke,%
    border shrink=\pgfpageoptionborder,%
    resized width=.5\pgfphysicalwidth,%
    resized height=.5\pgfphysicalheight,%
    center=\pgfpoint{.25\pgfphysicalwidth}{.167\pgfphysicalheight}%
  }%
  \pgfpageslogicalpageoptions{6}
  {%
    border code=\pgfsetlinewidth{1pt}\pgfstroke,%
    border shrink=\pgfpageoptionborder,%
    resized width=.5\pgfphysicalwidth,%
    resized height=.5\pgfphysicalheight,%
    center=\pgfpoint{.75\pgfphysicalwidth}{.167\pgfphysicalheight}%
  }%
}


  \pgfpagesuselayout{6 on 1 boxed}[letterpaper, border shrink=5mm]
  \nofiles
}

\usepackage{listings}
\usepackage{multimedia}
\usepackage[normalem]{ulem}
\usepackage{ifthen}
\usepackage{textcomp}

\usetheme{Warsaw} 
\usecolortheme{seahorse}
\useoutertheme{infolines} 

\setbeamertemplate{blocks}[rounded][shadow=true] 

\author{Joe Fields}
\title{Introduction to Proof} 

\date{Lecture 34 (GIAM \S 6.6) \newline special functions}
\institute[SCSU]{ {\tt fieldsj1@southernct.edu} }


\newlength{\cwidth}
\newcommand{\cents}{\settowidth{\cwidth}{c}%
\divide\cwidth by2
\advance\cwidth by-.1pt
c\kern-\cwidth
\vrule width .1pt depth.2ex height1.2ex
\kern\cwidth}

\newcommand{\sageprompt}{ {\tt sage$>$} }
\newcommand{\tab}{\rule{20pt}{0pt}}
\newcommand{\blnk}{\rule{1.5pt}{0pt}\rule{.4pt}{1.2pt}\rule{9pt}{.4pt}\rule{.4pt}{1.2pt}\rule{1.5pt}{0pt}}
\newcommand{\suchthat}{\; \rule[-3pt]{.25pt}{13pt} \;}
\newcommand{\divides}{\!\mid\!}
\newcommand{\tdiv}{\; \mbox{div} \;}
\newcommand{\restrict}[2]{#1 \,\rule[-4pt]{.125pt}{14pt}_{\,#2}}
\newcommand{\lcm}[2]{\mbox{lcm} (#1, #2)}
\renewcommand{\gcd}[2]{\mbox{gcd} (#1, #2)}
\newcommand{\Naturals}{{\mathbb N}}
\newcommand{\Integers}{{\mathbb Z}}
\newcommand{\Znoneg}{{\mathbb Z}^{\mbox{\tiny noneg}}}
\newcommand{\Enoneg}{{\mathbb E}^{\mbox{\tiny noneg}}}
\newcommand{\Qnoneg}{{\mathbb Q}^{\mbox{\tiny noneg}}}
\newcommand{\Rnoneg}{{\mathbb R}^{\mbox{\tiny noneg}}}
\newcommand{\Rationals}{{\mathbb Q}}
\newcommand{\Reals}{{\mathbb R}}
\newcommand{\Complexes}{{\mathbb C}}
%\newcommand{\F2}{{\mathbb F}_{2}}
\newcommand{\relQ}{\mbox{\textsf Q}}
\newcommand{\relR}{\mbox{\textsf R}}
\newcommand{\nrelR}{\mbox{\raisebox{1pt}{$\not$}\rule{1pt}{0pt}{\textsf R}}}
\newcommand{\relS}{\mbox{\textsf S}}
\newcommand{\relA}{\mbox{\textsf A}}
\newcommand{\Dom}[1]{\mbox{Dom}(#1)}
\newcommand{\Cod}[1]{\mbox{Cod}(#1)}
\newcommand{\Rng}[1]{\mbox{Rng}(#1)}

\DeclareMathOperator\caret{\raisebox{1ex}{$\scriptstyle\wedge$}}

\newtheorem*{defi}{Definition}
\newtheorem*{exer}{Exercise}
\newtheorem{thm}{Theorem}[section]
\newtheorem*{thm*}{Theorem}
\newtheorem{lem}[thm]{Lemma}
\newtheorem{cor}{Corollary}
\newtheorem{conj}{Conjecture}

\renewenvironment{proof}%
{\begin{quote} \emph{Proof:} }%
{\rule{0pt}{0pt} \newline \rule{0pt}{15pt} \hfill Q.E.D. \end{quote}}


\newcommand{\vs}{\rule{0pt}{11pt}}
\newcommand{\notimplies}{\;\not\!\!\!\implies}
\newcommand{\dx}{\,\mbox{d}x}

\AtBeginSection[]
{
 \begin{frame}{Table of Contents} 
  \tableofcontents[currentsection]
 \end{frame}
}

%%%% SAVE %%%%
%{ %magic to get a full screen image...
%\setbeamertemplate{navigation symbols}{}  % hide navigation buttons 
%\setbeamertemplate{background canvas}{\centerline{\includegraphics 
%	[height=\paperheight]{Cantor_4.jpeg}}}
%\begin{frame}[plain]
%\rule{0pt}{0pt}
%\end{frame} 
%} %end of magic


\begin{document}

\begin{frame}[plain]
  \titlepage
\end{frame}

\section{more about restrictions}

\begin{frame}{how to notate a restriction}
\begin{itemize}
  \item Use a vertical line after the function w/ the new domain as a subscript. \pause
  \item Generically, that would be
  \[ \restrict{f}{D} \] \pause
  \item In a particular instance, if we want to restrict the sine function so that it becomes 1-1: \pause
  \[ \restrict{\sin}{[-\pi/2, \pi/2]} \] \pause
  \item You can also include formulae and inequalities in this sort of notation: \pause
  \[ \restrict{x^2}{x \geq 0} \] \pause
\end{itemize}
\end{frame}


\begin{frame}{domains??}
\begin{itemize}
  \item What is the domain of a composition? \pause
  \item Recalling that $f \circ g$ means ``do g first then apply f to that.'' \pause
  \item If an input isn't in the domain of $g$ we can't even get started. \pause
  \item So $\Dom{f \circ g} \; \subseteq \; \Dom{g}$ \pause
  \item Why wasn't that an equals sign?  ( $=$ vs $\subseteq$ )
\end{itemize}
\end{frame}



\section{right and left inverses}

\begin{frame}{what is an inverse?}
	\begin{itemize}
		\item What is the inverse of a given function $f$? \pause
		\item One answer: It's a function $g$ that undoes whatever $f$ does. \pause
		\item $x^2$ and $\sqrt{x}$ are great prototypes to consider. \pause
		\item What exactly is meant by these two compositions?
		\[ \sqrt{x^2} \quad \mbox{and} \quad \sqrt{x}^2 \] \pause
		\item Which one of those is ``$x$''  \pause
		\item Each order of composition has its issues.  One has a reduced domain, and one has a formula which is not just $x$. \pause
		\item Given a function $f$ if $f\circ g$ has formula $x$ (restricted to some domain) then $g$ is a left inverse for $f$.  If $g \circ f (x) = x$ on whatever domain it's defined on, then $g$ is a right inverse of $f$
	\end{itemize}
\end{frame}

\section{inverse trig functions}

\begin{frame}{SOHCATOA}
	\begin{itemize}
		\item For each of the six trig functions let's do the following: \pause
		\begin{enumerate}
			\item Figure out a domain so the restricted function is $1-1$. \pause
			\item Plot the function and its inverse. \pause
			\item Decide if that was a left or right inverse. 
		\end{enumerate}
	\end{itemize}
\end{frame}

\begin{frame}{hyperbolic}
	\begin{itemize}
		\item Lets try the same thing for the hyperbolic trig functions.
	\end{itemize}
\end{frame}

\section{projections}

\begin{frame}{squooshes}
	\begin{itemize}
		\item Think about the process of determining the range of a function from looking at its graph. \pause
		\item You imagine squooshing it onto the y-axis and see what you get. \pause
		\item This approach can be formalized using the projection functions. \pause
		
		\[  \pi_1 (x,y) = x \quad \mbox{and} \quad \pi_2(x,y) = y. \] \pause
		
		\item Both of these are functions from $\Reals^2$ to $\Reals$
	\end{itemize}
\end{frame}

\begin{frame}{unsquooshes}
	\begin{itemize}
		\item How can you undo a projection?  \pause
		\item It's impossible! \pause
		\item Okay, then is there something that projections undo? \pause
		
		\[  d(x) \; = \; (x,x)  \] \pause
		
		\item Which order of composition (of $d$ and, say, $\pi_1$) gives an identity function and what is it's domain?
	\end{itemize}
\end{frame}

\begin{frame}{the mysterious ``third'' dimension}
	\begin{itemize}
		\item Let's look at OpenSCAD
	\end{itemize}
\end{frame}

\section{set theoretic functions}

\begin{frame}{set stuff}
	\begin{itemize}
		\item GIAM page 293.
	\end{itemize}
\end{frame}

\end{document}
