%\documentclass[handout,landscape]{beamer}
\documentclass[landscape]{beamer}
\hypersetup{pdfpagemode=FullScreen}
\mode<handout>
{
  \usepackage{pgf}
  \usepackage{pgfpages}

\pgfpagesdeclarelayout{6 on 1 boxed}
{
  \edef\pgfpageoptionheight{\the\paperheight} 
  \edef\pgfpageoptionwidth{\the\paperwidth}
  \edef\pgfpageoptionborder{0pt}
}
{
  \pgfpagesphysicalpageoptions
  {%
    logical pages=6,%
    physical height=\pgfpageoptionheight,%
    physical width=\pgfpageoptionwidth%
  }
  \pgfpageslogicalpageoptions{1}
  {%
    border code=\pgfsetlinewidth{2pt}\pgfstroke,%
    border shrink=\pgfpageoptionborder,%
    resized width=.5\pgfphysicalwidth,%
    resized height=.5\pgfphysicalheight,%
    center=\pgfpoint{.25\pgfphysicalwidth}{.833\pgfphysicalheight}%
  }%
  \pgfpageslogicalpageoptions{2}
  {%
    border code=\pgfsetlinewidth{2pt}\pgfstroke,%
    border shrink=\pgfpageoptionborder,%
    resized width=.5\pgfphysicalwidth,%
    resized height=.5\pgfphysicalheight,%
    center=\pgfpoint{.75\pgfphysicalwidth}{.833\pgfphysicalheight}%
  }%
  \pgfpageslogicalpageoptions{3}
  {%
    border code=\pgfsetlinewidth{2pt}\pgfstroke,%
    border shrink=\pgfpageoptionborder,%
    resized width=.5\pgfphysicalwidth,%
    resized height=.5\pgfphysicalheight,%
    center=\pgfpoint{.25\pgfphysicalwidth}{.5\pgfphysicalheight}%
  }%
  \pgfpageslogicalpageoptions{4}
  {%
    border code=\pgfsetlinewidth{2pt}\pgfstroke,%
    border shrink=\pgfpageoptionborder,%
    resized width=.5\pgfphysicalwidth,%
    resized height=.5\pgfphysicalheight,%
    center=\pgfpoint{.75\pgfphysicalwidth}{.5\pgfphysicalheight}%
  }%
  \pgfpageslogicalpageoptions{5}
  {%
    border code=\pgfsetlinewidth{2pt}\pgfstroke,%
    border shrink=\pgfpageoptionborder,%
    resized width=.5\pgfphysicalwidth,%
    resized height=.5\pgfphysicalheight,%
    center=\pgfpoint{.25\pgfphysicalwidth}{.167\pgfphysicalheight}%
  }%
  \pgfpageslogicalpageoptions{6}
  {%
    border code=\pgfsetlinewidth{2pt}\pgfstroke,%
    border shrink=\pgfpageoptionborder,%
    resized width=.5\pgfphysicalwidth,%
    resized height=.5\pgfphysicalheight,%
    center=\pgfpoint{.75\pgfphysicalwidth}{.167\pgfphysicalheight}%
  }%
}


  \pgfpagesuselayout{6 on 1 boxed}[letterpaper, border shrink=5mm]
  \nofiles
}

\usepackage{listings}
%\lstset{language=TeX}
\usepackage{multimedia}
\usepackage[normalem]{ulem}
\usepackage{amssymb}
\usepackage{ifthen}

%\usecolortheme[named=Purple]{structure} 
%\usetheme{Copenhagen}
\usetheme{Warsaw} 
\usecolortheme{seahorse}
\useoutertheme{infolines} 
%\usetheme[height=7mm]{Rochester} 
%\setbeamertemplate{items}[ball] 
\setbeamertemplate{blocks}[rounded][shadow=true] 
%\setbeamertemplate{navigation symbols}{} 
\author{Joe Fields}
\title{Introduction to Proof} 
%\subtitle{}
\date{Lecture 1}
\institute[SCSU]{ {\tt fieldsj1@southernct.edu} }

\newcommand{\versionNum}{$3.2$\ }

\newboolean{InTextBook}
\setboolean{InTextBook}{false}
\newboolean{InWorkBook}
\setboolean{InWorkBook}{false}
\newboolean{InHints}
\setboolean{InHints}{false}

%When this boolean is true (beginning in Section 5.1) we will use the convention
% that $0 \in \Naturals$.  If it is false we will continue to count $1$ as the smallest
%natural number (thus making Giuseppe Peano spin in his grave...)
 
\newboolean{ZeroInNaturals}

%This boolean is used to distinguish the version where we use $\sim$ rather than $\lnot$

\newboolean{LNotIsSim}

%The values of the last two booleans are set in ``switches.tex''

%\input{switches}

\let\savedlnot\lnot
\ifthenelse{\boolean{LNotIsSim}}{\renewcommand{\lnot}{\sim} }{}

%This command puts different amounts of space depending on whether we are
% in the text, the workbook or the hints & solutions manual. 
\newcommand{\twsvspace}[3]{%
 \ifthenelse{\boolean{InTextBook} }{\vspace{#1}}{%
  \ifthenelse{\boolean{InWorkBook} }{\vspace{#2}}{%
   \ifthenelse{\boolean{InHints} }{\vspace{#3}}{} %
   }%
  }%
 }


\newcommand{\wbvfill}{\ifthenelse{\boolean{InWorkBook}}{\vfill}{}}
\newcommand{\wbitemsep}{\ifthenelse{\boolean{InWorkBook} }{\rule[-24pt]{0pt}{60pt}}{}}
\newcommand{\textbookpagebreak}{\ifthenelse{\boolean{InTextBook}}{\newpage}{}}
\newcommand{\workbookpagebreak}{\ifthenelse{\boolean{InWorkBook}}{\newpage}{}}
\newcommand{\hintspagebreak}{\ifthenelse{\boolean{InHints}}{\newpage}{}}

\newcommand{\hint}[1]{\ifthenelse{\boolean{InHints}}{ {\par \hspace{12pt} \color[rgb]{0,0,1} #1 } }{}}
\newcommand{\inlinehint}[1]{\ifthenelse{\boolean{InHints}}{ { \color[rgb]{0,0,1} #1 } }{}}

%\newlength{\cwidth}
%\newcommand{\cents}{\settowidth{\cwidth}{c}%
%\divide\cwidth by2
%\advance\cwidth by-.1pt
%c\kern-\cwidth
%\vrule width .1pt depth.2ex height1.2ex
%\kern 3\cwidth}
\newcommand{\cents}{\textcent\kern 5pt}

\newcommand{\sageprompt}{ {\tt sage$>$} }
\newcommand{\tab}{\rule{20pt}{0pt}}
\newcommand{\blnk}{\rule{1.5pt}{0pt}\rule{.4pt}{1.2pt}\rule{9pt}{.4pt}\rule{.4pt}{1.2pt}\rule{1.5pt}{0pt}}
\newcommand{\suchthat}{\; \rule[-3pt]{.5pt}{13pt} \;}
\newcommand{\divides}{\!\mid\!}
\newcommand{\tdiv}{\; \mbox{div} \;}
\newcommand{\restrict}[2]{#1 \,\rule[-4pt]{.25pt}{14pt}_{\,#2}}
\newcommand{\lcm}[2]{\mbox{lcm} (#1, #2)}
\renewcommand{\gcd}[2]{\mbox{gcd} (#1, #2)}
\newcommand{\Naturals}{{\mathbb N}}
\newcommand{\Integers}{{\mathbb Z}}
\newcommand{\Znoneg}{{\mathbb Z}^{\mbox{\tiny noneg}}}
\ifthenelse{\boolean{ZeroInNaturals}}{%
  \newcommand{\Zplus}{{\mathbb Z}^+} }{%
  \newcommand{\Zplus}{{\mathbb N}} }
\newcommand{\Enoneg}{{\mathbb E}^{\mbox{\tiny noneg}}}
\newcommand{\Qnoneg}{{\mathbb Q}^{\mbox{\tiny noneg}}}
\newcommand{\Rnoneg}{{\mathbb R}^{\mbox{\tiny noneg}}}
\newcommand{\Rationals}{{\mathbb Q}}
\newcommand{\Reals}{{\mathbb R}}
\newcommand{\Complexes}{{\mathbb C}}
%\newcommand{\F2}{{\mathbb F}_{2}}
\newcommand{\relQ}{\mbox{\textsf Q}}
\newcommand{\relR}{\mbox{\textsf R}}
\newcommand{\nrelR}{\mbox{\raisebox{1pt}{$\not$}\rule{1pt}{0pt}{\textsf R}}}
\newcommand{\relS}{\mbox{\textsf S}}
\newcommand{\relA}{\mbox{\textsf A}}
\newcommand{\Dom}[1]{\mbox{Dom}(#1)}
\newcommand{\Cod}[1]{\mbox{Cod}(#1)}
\newcommand{\Rng}[1]{\mbox{Rng}(#1)}

\DeclareMathOperator\caret{\raisebox{1ex}{$\scriptstyle\wedge$}}

\newtheorem*{defi}{Definition}
\newtheorem*{exer}{Exercise}
\newtheorem{thm}{Theorem}[section]
\newtheorem*{thm*}{Theorem}
\newtheorem{lem}[thm]{Lemma}
\newtheorem*{lem*}{Lemma}
\newtheorem{cor}{Corollary}
\newtheorem{conj}{Conjecture}

\renewenvironment{proof}%
{\begin{quote} \emph{Proof:} }%
{\rule{0pt}{0pt} \newline \rule{0pt}{15pt} \hfill Q.E.D. \end{quote}}


\newcommand{\vs}{\rule{0pt}{12pt}}

\AtBeginSection[]
{
 \begin{frame}{Table of Contents} 
  \tableofcontents[currentsection]
 \end{frame}
}

%%%% SAVE %%%%
%{ %magic to get a full screen image...
%\setbeamertemplate{navigation symbols}{}  % hide navigation buttons 
%\setbeamertemplate{background canvas}{\centerline{\includegraphics 
%	[height=\paperheight]{Cantor_4.jpeg}}}
%\begin{frame}[plain]
%\rule{0pt}{0pt}
%\end{frame} 
%} %end of magic


\begin{document}

\begin{frame}[plain]
  \titlepage
\end{frame}

\section{intro}

\begin{frame}{purpose/philosophy}
\begin{itemize}
\item The change from calculation to proof\pause
\item A transition \textellipsis
\end{itemize}
\end{frame}

\begin{frame}{the flipped classroom}
\begin{itemize}
\item Listening to lectures (a passive activity) happens at home. \pause
\item ``Homework'' (an active activity where you might need a hint or two from your prof.) happens during class time. \pause 
\item Typically you'll also have actual homework (done at home) too. \pause
\end{itemize}
\end{frame}

\begin{frame}{open educational resources}
\begin{itemize}
\item A descendent of the open source software movement.\pause
\item Free as in beer / free as in speech \pause
\item Major advantage: all students have access to the text on day 1. \pause
\item Another advantage: Professors can grab the source and modify the book to suit their tastes.\pause
\item You can contribute! 
\end{itemize}
\end{frame}

\begin{frame}{layout of the book}
\begin{itemize}
\item Recursive structure - topics are revisited with increasing levels of rigor. \pause
\item Chapter 1 contains most of the key ideas for the course (but not much detail). \pause
\item Objects of study (Logic, Sets, Relations \& Functions, Cardinality) are interspersed with chapters about proof techniques. 
\end{itemize}
\end{frame}

\section{resources} 

\begin{frame}{the book}
\begin{itemize}
\item Pdf available at GitHub for free. \href{http://osj1961.github.io/giam/}{http://osj1961.github.io/giam/} \pause
\item Source code is also available at Github.  
\item Listed at the American Institute of Mathematics - Open Textbook Inititative. \href{https://aimath.org/textbooks/approved-textbooks/}{https://aimath.org/textbooks/approved-textbooks/} \pause
\item Also available on Amazon.com for \$16 if you're wedded to dead-tree versions. 
\end{itemize}
\end{frame}

\begin{frame}{computational aids}
\begin{itemize}
\item A calculator \pause
\item sage - a computer algebra system \pause
\item desmos - for graphing \pause
\item geogebra - for geometric constructions 
\end{itemize}
\end{frame}

\section{section 1.1}

\begin{frame}{basic sets}
\begin{itemize}
\item The naturals $\Naturals$. \pause 
      (We include $0$ as a natural number.) \pause
\item The integers $\Integers$. \pause
\item The rational numbers $\Rationals$. \pause
\item The real numbers $\Reals$. \pause
\item The complex numbers $\Complexes$. 
\end{itemize}
\end{frame}


\begin{frame}{roster form vs.\ set-builder notation}
\begin{itemize}
\item Roster form is simply a listing of the elements in a set. \pause
\item For example, $\{ \heartsuit, \spadesuit, \diamondsuit, \clubsuit \}$ is the set of so-called ``suits'' in a standard deck of playing cards. \pause
\item Note the use of the curly braces -- sets are always denoted with curly braces.\pause
\item In set-builder notation we don't list everything, instead we name some property that elements of the set must satisfy. \pause
\item Example: $\{ x \suchthat 1 \leq x \leq 4 \}$.\pause
\item Problem: What kind of number is $x$ ? \pause
\item Solution: Always specify a {\em universe of discourse}. \pause
\item $\{ x \in \Reals \suchthat 1 \leq x \leq 4 \}$ and $\{ x \in \Integers \suchthat 1 \leq x \leq 4 \}$ are very different sets!
\end{itemize}
\end{frame}

\begin{frame}{reading set-builder notation}

\vspace{.1in}

\begin{tabular}{c|c|c}
\rule[-10pt]{0pt}{22pt} $\Rationals$ & $=$ & $\{$  \\ \hline
\rule[-6pt]{0pt}{22pt} The rational numbers & are defined to be & the set of all\\
\end{tabular}

\vspace{.1in}

\begin{tabular}{c|c}
\rule[-10pt]{0pt}{22pt} $\displaystyle \frac{a}{b}$ & $\suchthat$ \\ \hline
\rule[-6pt]{0pt}{22pt} fractions of the form $a$ over $b$ & such that \\
\end{tabular}

\vspace{.1in}

\begin{tabular}{c|c|c}
\rule[-10pt]{0pt}{22pt} $a \in \Integers$ & and & $b \in \Integers$ \\ \hline
\rule[-6pt]{0pt}{22pt} $a$ is an element of the integers & and & $b$ is an
element of the integers \\
\end{tabular}

\vspace{.1in}

\begin{tabular}{c|c|c}
\rule[-10pt]{0pt}{22pt} and & $b \neq 0$ & $\}$ \\ \hline
\rule[-6pt]{0pt}{22pt} and & $b$ is nonzero. & (the final curly brace
is silent) \\
\end{tabular}

\end{frame}


\begin{frame}{operations in Q}
\begin{itemize}
\item First: NOT EVERY REAL NUMBER CAN BE EXPRESSED AS A FRACTION!!! \pause
\item The Ancient Greeks were initially confused about this but eventually realized that $\Rationals \neq \Reals$.  (They didn't have a clue about $\Complexes$.)
\item Products, Sums and Mediants in $\Rationals$.
\end{itemize}
\end{frame}

\begin{frame}{operations in C}
\begin{itemize}
\item Products and Sums in $\Complexes$. \pause
\item Algebraically -- componentwise addition and the FLOI rule. \pause
\item Geometrically -- addition works like vectors - multiplication is \textellipsis \pause
 interesting.

\end{itemize}
\end{frame}

\begin{frame}{trichotomy}
\begin{itemize}
\item Contrast with {\em dichotomy}. \pause
\item To cut into three parts. \pause
\item The trichotomy property of the reals is:  Every real number is either positive, negative or zero. \pause
\item The subsets of $\Reals$ inherit this property. \pause (Although it's kinda weak for $\Naturals$.) \pause
\item There are fairly standard ways to denote the sets restricted to some of these parts. \pause

\[  \Reals^+ \quad \Reals^- \quad \Rnoneg \quad \Reals^\ast \]

\end{itemize}
\end{frame}


\end{document}
