%\documentclass[handout,landscape]{beamer}
\documentclass[landscape]{beamer}
%\hypersetup{pdfpagemode=FullScreen}
\mode<handout>
{
  \usepackage{pgf}
  \usepackage{pgfpages}

\pgfpagesdeclarelayout{6 on 1 boxed}
{
  \edef\pgfpageoptionheight{\the\paperheight} 
  \edef\pgfpageoptionwidth{\the\paperwidth}
  \edef\pgfpageoptionborder{0pt}
}
{
  \pgfpagesphysicalpageoptions
  {%
    logical pages=6,%
    physical height=\pgfpageoptionheight,%
    physical width=\pgfpageoptionwidth%
  }
  \pgfpageslogicalpageoptions{1}
  {%
    border code=\pgfsetlinewidth{2pt}\pgfstroke,%
    border shrink=\pgfpageoptionborder,%
    resized width=.5\pgfphysicalwidth,%
    resized height=.5\pgfphysicalheight,%
    center=\pgfpoint{.25\pgfphysicalwidth}{.833\pgfphysicalheight}%
  }%
  \pgfpageslogicalpageoptions{2}
  {%
    border code=\pgfsetlinewidth{2pt}\pgfstroke,%
    border shrink=\pgfpageoptionborder,%
    resized width=.5\pgfphysicalwidth,%
    resized height=.5\pgfphysicalheight,%
    center=\pgfpoint{.75\pgfphysicalwidth}{.833\pgfphysicalheight}%
  }%
  \pgfpageslogicalpageoptions{3}
  {%
    border code=\pgfsetlinewidth{2pt}\pgfstroke,%
    border shrink=\pgfpageoptionborder,%
    resized width=.5\pgfphysicalwidth,%
    resized height=.5\pgfphysicalheight,%
    center=\pgfpoint{.25\pgfphysicalwidth}{.5\pgfphysicalheight}%
  }%
  \pgfpageslogicalpageoptions{4}
  {%
    border code=\pgfsetlinewidth{2pt}\pgfstroke,%
    border shrink=\pgfpageoptionborder,%
    resized width=.5\pgfphysicalwidth,%
    resized height=.5\pgfphysicalheight,%
    center=\pgfpoint{.75\pgfphysicalwidth}{.5\pgfphysicalheight}%
  }%
  \pgfpageslogicalpageoptions{5}
  {%
    border code=\pgfsetlinewidth{2pt}\pgfstroke,%
    border shrink=\pgfpageoptionborder,%
    resized width=.5\pgfphysicalwidth,%
    resized height=.5\pgfphysicalheight,%
    center=\pgfpoint{.25\pgfphysicalwidth}{.167\pgfphysicalheight}%
  }%
  \pgfpageslogicalpageoptions{6}
  {%
    border code=\pgfsetlinewidth{2pt}\pgfstroke,%
    border shrink=\pgfpageoptionborder,%
    resized width=.5\pgfphysicalwidth,%
    resized height=.5\pgfphysicalheight,%
    center=\pgfpoint{.75\pgfphysicalwidth}{.167\pgfphysicalheight}%
  }%
}


  \pgfpagesuselayout{6 on 1 boxed}[letterpaper, border shrink=5mm]
  \nofiles
}

\usepackage{listings}
%\lstset{language=TeX}
\usepackage{multimedia}
\usepackage[normalem]{ulem}
\usepackage{amssymb}

%\usecolortheme[named=Purple]{structure} 
%\usetheme{Copenhagen}
\usetheme{Warsaw} 
\usecolortheme{seahorse}
\useoutertheme{infolines} 
%\usetheme[height=7mm]{Rochester} 
%\setbeamertemplate{items}[ball] 
\setbeamertemplate{blocks}[rounded][shadow=true] 
%\setbeamertemplate{navigation symbols}{} 
\author{Joe Fields}
\title{Introduction to Proof} 
%\subtitle{}
\date{Lecture 15 (GIAM \S 3.2)}
\institute[SCSU]{ {\tt fieldsj1@southernct.edu} }

\newcommand{\versionNum}{$3.2$\ }

\newboolean{InTextBook}
\setboolean{InTextBook}{false}
\newboolean{InWorkBook}
\setboolean{InWorkBook}{false}
\newboolean{InHints}
\setboolean{InHints}{false}

%When this boolean is true (beginning in Section 5.1) we will use the convention
% that $0 \in \Naturals$.  If it is false we will continue to count $1$ as the smallest
%natural number (thus making Giuseppe Peano spin in his grave...)
 
\newboolean{ZeroInNaturals}

%This boolean is used to distinguish the version where we use $\sim$ rather than $\lnot$

\newboolean{LNotIsSim}

%The values of the last two booleans are set in ``switches.tex''

%\input{switches}

\let\savedlnot\lnot
\ifthenelse{\boolean{LNotIsSim}}{\renewcommand{\lnot}{\sim} }{}

%This command puts different amounts of space depending on whether we are
% in the text, the workbook or the hints & solutions manual. 
\newcommand{\twsvspace}[3]{%
 \ifthenelse{\boolean{InTextBook} }{\vspace{#1}}{%
  \ifthenelse{\boolean{InWorkBook} }{\vspace{#2}}{%
   \ifthenelse{\boolean{InHints} }{\vspace{#3}}{} %
   }%
  }%
 }


\newcommand{\wbvfill}{\ifthenelse{\boolean{InWorkBook}}{\vfill}{}}
\newcommand{\wbitemsep}{\ifthenelse{\boolean{InWorkBook} }{\rule[-24pt]{0pt}{60pt}}{}}
\newcommand{\textbookpagebreak}{\ifthenelse{\boolean{InTextBook}}{\newpage}{}}
\newcommand{\workbookpagebreak}{\ifthenelse{\boolean{InWorkBook}}{\newpage}{}}
\newcommand{\hintspagebreak}{\ifthenelse{\boolean{InHints}}{\newpage}{}}

\newcommand{\hint}[1]{\ifthenelse{\boolean{InHints}}{ {\par \hspace{12pt} \color[rgb]{0,0,1} #1 } }{}}
\newcommand{\inlinehint}[1]{\ifthenelse{\boolean{InHints}}{ { \color[rgb]{0,0,1} #1 } }{}}

%\newlength{\cwidth}
%\newcommand{\cents}{\settowidth{\cwidth}{c}%
%\divide\cwidth by2
%\advance\cwidth by-.1pt
%c\kern-\cwidth
%\vrule width .1pt depth.2ex height1.2ex
%\kern 3\cwidth}
\newcommand{\cents}{\textcent\kern 5pt}

\newcommand{\sageprompt}{ {\tt sage$>$} }
\newcommand{\tab}{\rule{20pt}{0pt}}
\newcommand{\blnk}{\rule{1.5pt}{0pt}\rule{.4pt}{1.2pt}\rule{9pt}{.4pt}\rule{.4pt}{1.2pt}\rule{1.5pt}{0pt}}
\newcommand{\suchthat}{\; \rule[-3pt]{.5pt}{13pt} \;}
\newcommand{\divides}{\!\mid\!}
\newcommand{\tdiv}{\; \mbox{div} \;}
\newcommand{\restrict}[2]{#1 \,\rule[-4pt]{.25pt}{14pt}_{\,#2}}
\newcommand{\lcm}[2]{\mbox{lcm} (#1, #2)}
\renewcommand{\gcd}[2]{\mbox{gcd} (#1, #2)}
\newcommand{\Naturals}{{\mathbb N}}
\newcommand{\Integers}{{\mathbb Z}}
\newcommand{\Znoneg}{{\mathbb Z}^{\mbox{\tiny noneg}}}
\ifthenelse{\boolean{ZeroInNaturals}}{%
  \newcommand{\Zplus}{{\mathbb Z}^+} }{%
  \newcommand{\Zplus}{{\mathbb N}} }
\newcommand{\Enoneg}{{\mathbb E}^{\mbox{\tiny noneg}}}
\newcommand{\Qnoneg}{{\mathbb Q}^{\mbox{\tiny noneg}}}
\newcommand{\Rnoneg}{{\mathbb R}^{\mbox{\tiny noneg}}}
\newcommand{\Rationals}{{\mathbb Q}}
\newcommand{\Reals}{{\mathbb R}}
\newcommand{\Complexes}{{\mathbb C}}
%\newcommand{\F2}{{\mathbb F}_{2}}
\newcommand{\relQ}{\mbox{\textsf Q}}
\newcommand{\relR}{\mbox{\textsf R}}
\newcommand{\nrelR}{\mbox{\raisebox{1pt}{$\not$}\rule{1pt}{0pt}{\textsf R}}}
\newcommand{\relS}{\mbox{\textsf S}}
\newcommand{\relA}{\mbox{\textsf A}}
\newcommand{\Dom}[1]{\mbox{Dom}(#1)}
\newcommand{\Cod}[1]{\mbox{Cod}(#1)}
\newcommand{\Rng}[1]{\mbox{Rng}(#1)}

\DeclareMathOperator\caret{\raisebox{1ex}{$\scriptstyle\wedge$}}

\newtheorem*{defi}{Definition}
\newtheorem*{exer}{Exercise}
\newtheorem{thm}{Theorem}[section]
\newtheorem*{thm*}{Theorem}
\newtheorem{lem}[thm]{Lemma}
\newtheorem*{lem*}{Lemma}
\newtheorem{cor}{Corollary}
\newtheorem{conj}{Conjecture}

\renewenvironment{proof}%
{\begin{quote} \emph{Proof:} }%
{\rule{0pt}{0pt} \newline \rule{0pt}{15pt} \hfill Q.E.D. \end{quote}}


\newcommand{\vs}{\rule{0pt}{12pt}}

\def\mycommand{\setlength{\abovedisplayskip}{-12pt}%
\setlength{\belowdisplayskip}{-12pt}%
\setlength{\abovedisplayshortskip}{0pt}%
\setlength{\belowdisplayshortskip}{0pt}}

\let\oldselectfont\selectfont
\def\selectfont{\oldselectfont\mycommand}

\mycommand

\AtBeginSection[]
{
 \begin{frame}{Table of Contents} 
  \tableofcontents[currentsection]
 \end{frame}
}

%%%% SAVE %%%%
%{ %magic to get a full screen image...
%\setbeamertemplate{navigation symbols}{}  % hide navigation buttons 
%\setbeamertemplate{background canvas}{\centerline{\includegraphics 
%	[height=\paperheight]{Cantor_4.jpeg}}}
%\begin{frame}[plain]
%\rule{0pt}{0pt}
%\end{frame} 
%} %end of magic


\begin{document}

\begin{frame}[plain]
  \titlepage
\end{frame}

\section{The forward-backward method}

\begin{frame}{try backing in}
\begin{itemize}
\item ``If you get stuck, try working backwards'' \pause
\item Ideally, a proof looks like this: \pause
\begin{enumerate}
\item[0)] Hypothesis \pause (although there may be several of these) \pause
\item[1)] Deduction 1 \pause
\item[2)] Deduction 2 \pause 
\item[3)] Deduction 3 \pause \newline
\rule{24pt}{0pt} $\displaystyle \vdots$  \pause
\item[n)] Conclusion \pause \rule{12pt}{0pt} $=$ Deduction $n$ \pause
\end{enumerate}
\item Each deduction is implied ($\implies$) by the statements prior to it.
\end{itemize}
\end{frame}

\begin{frame}{stuck?}
\begin{itemize}
\item If you're stuck at Deduction $k$, and don't see any way forward towards the conclusion\textellipsis \pause
\item Maybe you can figure out a Deduction $n-1$ \pause \newline
\rule{.2in}{0in} a statement that implies the conclusion. \pause
\item At least you'll have closed up the gap! \pause
\item Recognize that the deduction are often connected via $\iff$ \pause \newline
\rule{.2in}{0in} (which is great if you're working backwards!) \pause
\item If not, try to intentionally make converse errors ;-) 
\end{itemize}
\end{frame}

\begin{frame}{back and forth}
\begin{itemize}
\item Sometimes you can reason from the conclusion all the way back to the hypotheses. \pause
\item If not, try working a little from the beginning, then a little from the end, and so on\textellipsis \pause
\item Hopefully you'll be able to meet somewhere in the middle. 
\end{itemize}
\end{frame}

\begin{frame}{the write up}
\begin{itemize}
\item Note that everything we've discussed so far is about discovering the proof.\pause
\item Think of it as ``scratch work''! \pause
\item Every proof must, in the end, proceed from hypotheses to conclusion. \pause
\item IN THE FORWARD ORDER! \pause
\item Make sure your write-up goes in the forward direction.
\end{itemize}
\end{frame}

\section{understanding AM and GM}

\begin{frame}{a classic example}
\begin{itemize}
\item ``The arithmetic mean, geometric mean inequality.''\pause
\item You are probably quite familiar with the ``arithmetic mean,'' \pause \newline
that's just a fancy way to say ``average.'' \pause
\item If the two numbers are $x$ and $y$, the arithmetic mean is $(x+y)/2$. \pause
\item If you think of moving each operation up one in the difficulty heirarchy, \pause
\item adding becomes multiplying \pause \newline 
and dividing by 2 (or multiplying by $1/2$) becomes raising to the $1/2$ power. \pause
\item So we get $\sqrt{xy}$. \pause The Greeks called this the {\em geometric mean}.
\end{itemize}
\end{frame}

\begin{frame}{why geometric?}
\begin{itemize}
\item Desmos animation \pause
\item If you have an $x$ by $y$ rectangle, then it has the same area as an $s$ by $s$ square, provided that $s=\sqrt{xy}$. \pause
\item Also, notice that $s$ will lie in between $x$ and $y$, so it does something rather like the ordinary mean.\pause
\item There are actually {\em many} ways of finding a number that lies in between two others! \pause
\item BTW, notice that the geometric mean would be  nonsensical if one of $x$ or $y$ were negative. 
\end{itemize}
\end{frame}

\begin{frame}{filling in holes}
\begin{itemize}
\item You may have heard the terms arithmetic and geometric applied to sequences. \pause
\item In an arithmetic sequence there is a common difference between succesive terms. \pause \newline
(A difference table for such a sequence will have just two rows.) \pause
\item A geometric sequence has a common ratio between terms. \pause \newline
(P.S.\ This gives us another way to analyze a sequence where each row contains the ratio of the terms in the row above.) \pause
\item An arithmetic sequence has the form $a+bx$. \pause A geometric sequence has the form $a\cdot b^x$. \pause
\item These usages are related to the ones for means.
\end{itemize}
\end{frame}

\begin{frame}{how?}
\begin{itemize}


\item If you have a missing entry in an arithmetic sequence you can fill it in using the arithmetic mean. \pause

\item Example:

\[ 7 \qquad 10 \qquad 13 \qquad \underline{\hspace{.3in}} \qquad 19 \qquad 22 \qquad 25 \]
\pause

\item The blank should be filled-in with $(13+19)/2 \; = \; 16$. \pause

\item If you have a missing entry in an geometric sequence you can fill it in using the geometric mean. \pause

\item Example:

\[ 3 \qquad 6 \qquad 12 \qquad \underline{\hspace{.3in}} \qquad 48 \qquad 96 \qquad 192 \]

\pause

\item The blank should be filled-in with $\sqrt{12 \cdot 48} \; = \; \sqrt{576} \; = \; 24$. 

\end{itemize}
\end{frame}

\section{the AM-GM inequality}

\begin{frame}{first, what is it?}
\begin{itemize}
\item The AM-GM inequality states that the arithmetic mean is usually bigger than the geometric mean.\pause
\item Notice that both means give the same result when $x=y$. \pause
\item So

\[ \frac{a+b}{2} \; \, \geq \, \; \sqrt{ab} \]

\pause 

\item Because of the weirdness that the geometric mean encounters with negative numbers under the radical, \pause \newline
(and because the result is simply not true when both numbers are negative!) \pause \newline
We need to restrict the universe of discourse. \pause

\item Here's the formal statement: \pause

\[ \forall\; x, y \, \in \Reals^+, \; \frac{x+y}{2} \; \, \geq \, \; \sqrt{xy} \]

\end{itemize}
\end{frame}

\begin{frame}{prove it!}
\begin{itemize}
\item Well, here's the problem, the only hypotheses we can use are that $x$ and $y$ are positive.\pause
\item A common thing to do when there are two variables is to say that we may as well assume that one of them is the smaller one. \pause
\item WLOG \pause \hspace{.2in} (without loss of generality) \pause
\item ``There is no loss in generality in assuming that $x \geq y$.'' \pause
\item Other than that, how can we possibly move forward? \pause \newline
\centerline{\tiny Try working backwards from the conclusion.}
\end{itemize}
\end{frame}

\begin{frame}{okay, here goes}

\begin{tabular}{ccrcl}
\uncover<1->{ \rule[-6pt]{0pt}{24pt} &            & $\frac{x+y}{2}$ & $\, \geq \,$ & $\sqrt{xy}$   \\ }
\uncover<2->{ \rule[-6pt]{0pt}{24pt} & $\iff \; $ & $x+y$           & $\, \geq \,$ & $2\sqrt{xy}$  \\ }
\uncover<3->{ \rule[-6pt]{0pt}{24pt} & $\iff \; $ & $(x+y)^2$       & $\, \geq \,$ & $4xy$   \\  }
\uncover<4->{ \rule[-6pt]{0pt}{24pt} & $\iff \; $ & $x^2+2xy+y^2$   & $\, \geq \,$ & $4xy$  \\ }
\uncover<5->{ \rule[-6pt]{0pt}{24pt} & $\iff \; $ & $x^2-2xy+y^2$   & $\, \geq \,$ & $0$ \\ }
\end{tabular}

\end{frame}

\begin{frame}{finishing up}
\begin{itemize}
\item In GIAM, it says ``whoa, we're done!'' when it gets to $a^2-2ab+b^2 \, \geq \, 0$ \pause
\item Here's one more step: \pause

\[ \iff \; (x-y)^2 \, \geq \, 0 \] 

\pause

\item In our proof we'll need to introduce the tautology \pause \newline
that $(x-y)^2 \geq 0$.  
\item This is a statement that is true for all real numbers $x$ and $y$. \pause
\item The proof of this is something we'll highlight in the section on ``Proof by cases,'' \pause \newline 
it is an easy consequence of the trichotomy property of $\Reals$. \pause
\item But, for now, don't we all believe that the smallest a square can be (in the real numbers) is $0$?
\end{itemize}
\end{frame}

\begin{frame}{really finishing up}


{\em Proof:} Suppose that $x$ and $y$ are particular, but arbitrarily chosen real numbers.  Without loss of general
$x \geq y$, so $x-y \geq 0$.  Note that the square of $x-y$ is the square of a real number so it satisfies $(x-y)^2 \geq 0$.

Thus 
\begin{align*}
 &         & (x-y)^2 \, \geq \, 0 \\
 & \iff \; & x^2-2xy+y^2 \, \geq \, 0 \\
 & \iff \; & x^2+2xy+y^2 \, \geq \, 4xy \\
 & \iff \; & (x+y)^2 \, \geq \, 4xy \\
 & \iff \; & x+y \, \, \geq \, \, 2\sqrt{xy} \\
 & \iff \; & \frac{x+y}{2} \, \geq \, \sqrt{xy}, 
\end{align*}

\hspace{\fill} Q.E.D.

\end{frame}

\begin{frame}{should we double-check?}
\begin{itemize}
\item Absolutely! We should always double check! \pause
\item Since our deductions were made in the wrong direction, \pause \newline
there's the potential that what we thought was a $\iff$ is really a $\implies$ and its converse is false. \pause
\item To me the suspicious step is when we took the square root on both sides of an inequality. \pause
\item Is it actually true that (for positive numbers $A$ and $B$) $A \geq B \; \implies \; \sqrt{A} \geq \sqrt{B}$? 
\end{itemize}
\end{frame}

\begin{frame}{yep, we're good}
\begin{itemize}
\item So, yes, it's true that $A \geq B \; \implies \; \sqrt{A} \geq \sqrt{B}$. \pause
\item Let's look at the graph on Desmos.
\end{itemize}
\end{frame}
 
\begin{frame}{can we finally sort out how inequalities work?}
\begin{itemize}
\item If you apply a strictly increasing function to both sides of an inequality, \pause \newline
the sense of the inequality remains the same. \pause
\item If you apply a strictly decreasing function to both sides of an inequality, \pause \newline
the sense of the inequality is reversed. \pause \newline
(If the function is not monotonic, you just can't use it!)\pause 
\item This is why ``you have to switch the inequality if you multiply by a negative.'' \pause
\item Also, for instance, squaring both sides of an inequality is generally invalid. \pause \newline
(Although if you're working in $\Reals^+$ it's okay.) 
\end{itemize}
\end{frame}

\section{more examples}

\begin{frame}{party like it's 1999}
\begin{itemize}
\item Which fraction is bigger? \pause

\[ \frac{1999}{2000} \quad \mbox{vs.} \quad \frac{2000}{2001} \] \pause

\item Let me point out that you could generalize this.  We're looking at the $x=2000$ case of 

\[ \frac{x-1}{x} \quad \mbox{vs.} \quad \frac{x}{x+1}. \] \pause

\item I don't really know what the conclusion will be yet. \pause (Does ``vs.'' get replaced by $<$ or $>$ ?) \pause
\item It's okay, we can still ``work backwards from the conclusion'' and figure it out later. \pause
\item But, since we replaced $2000$ with $x$ we've introduced a potential problem. \pause What if $x$ was $0$ or $-1$? \pause
\item Let's add the hypothesis that $x>0$.

\end{itemize}
\end{frame}

\begin{frame}{hmmm\textellipsis}
\begin{itemize}
\item Since $x>0$ it follows that $x+1>0$ \pause (this is known as weakening the inequality) \pause
\item Thus it's fine to ``cross-multiply.'' \pause \newline
Because that really just means multiplying both sides of the inequality by $x$ and by $x+1$. \pause \newline
And they're both positive. \pause
\item So we have 

\[ (x-1)\cdot(x+1) \quad \mbox{vs.} \quad x\cdot x, \] 

\pause

\noindent so, 

\[ x^2-1 \quad \mbox{vs.} \quad x^2.  \] 
\pause

\item Clearly the ``vs.'' should be $<$.
\end{itemize}
\end{frame}

\begin{frame}{the product rule}
\begin{itemize}
\item Exercise 8, in Section 3.2 asks us to use the forward-backward approach in proving the product rule for derivatives.\pause
\item This is the rule that says $(f(x)\cdot g(x))' \; = \; f'(x)\cdot g(x) \, + \, f(x)\cdot g'(x)$. \pause
\item Previously you may not have taken much notice of this, but there is a hypothesis! \pause
\item The product rule is only valid if $f(x)$ and $g(x)$ are differentiable at $x$. \pause 
\item Recall that ``differentiability implies continuity.'' \pause (we'll need this at one point)
\end{itemize}
\end{frame}

\begin{frame}{a scattered proof}
We're going low-tech for this.
\end{frame}


\end{document}
