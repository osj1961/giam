%\documentclass[handout,landscape]{beamer}
\documentclass[landscape]{beamer}
%\hypersetup{pdfpagemode=FullScreen}
\mode<handout>
{
  \usepackage{pgf}
  \usepackage{pgfpages}

\pgfpagesdeclarelayout{6 on 1 boxed}
{
  \edef\pgfpageoptionheight{\the\paperheight} 
  \edef\pgfpageoptionwidth{\the\paperwidth}
  \edef\pgfpageoptionborder{0pt}
}
{
  \pgfpagesphysicalpageoptions
  {%
    logical pages=6,%
    physical height=\pgfpageoptionheight,%
    physical width=\pgfpageoptionwidth%
  }
  \pgfpageslogicalpageoptions{1}
  {%
    border code=\pgfsetlinewidth{2pt}\pgfstroke,%
    border shrink=\pgfpageoptionborder,%
    resized width=.5\pgfphysicalwidth,%
    resized height=.5\pgfphysicalheight,%
    center=\pgfpoint{.25\pgfphysicalwidth}{.833\pgfphysicalheight}%
  }%
  \pgfpageslogicalpageoptions{2}
  {%
    border code=\pgfsetlinewidth{2pt}\pgfstroke,%
    border shrink=\pgfpageoptionborder,%
    resized width=.5\pgfphysicalwidth,%
    resized height=.5\pgfphysicalheight,%
    center=\pgfpoint{.75\pgfphysicalwidth}{.833\pgfphysicalheight}%
  }%
  \pgfpageslogicalpageoptions{3}
  {%
    border code=\pgfsetlinewidth{2pt}\pgfstroke,%
    border shrink=\pgfpageoptionborder,%
    resized width=.5\pgfphysicalwidth,%
    resized height=.5\pgfphysicalheight,%
    center=\pgfpoint{.25\pgfphysicalwidth}{.5\pgfphysicalheight}%
  }%
  \pgfpageslogicalpageoptions{4}
  {%
    border code=\pgfsetlinewidth{2pt}\pgfstroke,%
    border shrink=\pgfpageoptionborder,%
    resized width=.5\pgfphysicalwidth,%
    resized height=.5\pgfphysicalheight,%
    center=\pgfpoint{.75\pgfphysicalwidth}{.5\pgfphysicalheight}%
  }%
  \pgfpageslogicalpageoptions{5}
  {%
    border code=\pgfsetlinewidth{2pt}\pgfstroke,%
    border shrink=\pgfpageoptionborder,%
    resized width=.5\pgfphysicalwidth,%
    resized height=.5\pgfphysicalheight,%
    center=\pgfpoint{.25\pgfphysicalwidth}{.167\pgfphysicalheight}%
  }%
  \pgfpageslogicalpageoptions{6}
  {%
    border code=\pgfsetlinewidth{2pt}\pgfstroke,%
    border shrink=\pgfpageoptionborder,%
    resized width=.5\pgfphysicalwidth,%
    resized height=.5\pgfphysicalheight,%
    center=\pgfpoint{.75\pgfphysicalwidth}{.167\pgfphysicalheight}%
  }%
}


  \pgfpagesuselayout{6 on 1 boxed}[letterpaper, border shrink=5mm]
  \nofiles
}

\usepackage{listings}
%\lstset{language=TeX}
\usepackage{multimedia}
\usepackage[normalem]{ulem}
\usepackage{amssymb}

%\usecolortheme[named=Purple]{structure} 
%\usetheme{Copenhagen}
\usetheme{Warsaw} 
\usecolortheme{seahorse}
\useoutertheme{infolines} 
%\usetheme[height=7mm]{Rochester} 
%\setbeamertemplate{items}[ball] 
\setbeamertemplate{blocks}[rounded][shadow=true] 
%\setbeamertemplate{navigation symbols}{} 
\author{Joe Fields}
\title{Introduction to Proof} 
%\subtitle{}
\date{Lecture 7}
\institute[SCSU]{ {\tt fieldsj1@southernct.edu} }


\newlength{\cwidth}
\newcommand{\cents}{\settowidth{\cwidth}{c}%
\divide\cwidth by2
\advance\cwidth by-.1pt
c\kern-\cwidth
\vrule width .1pt depth.2ex height1.2ex
\kern\cwidth}

\newcommand{\sageprompt}{ {\tt sage$>$} }
\newcommand{\tab}{\rule{20pt}{0pt}}
\newcommand{\blnk}{\rule{1.5pt}{0pt}\rule{.4pt}{1.2pt}\rule{9pt}{.4pt}\rule{.4pt}{1.2pt}\rule{1.5pt}{0pt}}
\newcommand{\suchthat}{\; \rule[-3pt]{.25pt}{13pt} \;}
\newcommand{\divides}{\!\mid\!}
\newcommand{\tdiv}{\; \mbox{div} \;}
\newcommand{\restrict}[2]{#1 \,\rule[-4pt]{.125pt}{14pt}_{\,#2}}
\newcommand{\lcm}[2]{\mbox{lcm} (#1, #2)}
\renewcommand{\gcd}[2]{\mbox{gcd} (#1, #2)}
\newcommand{\Naturals}{{\mathbb N}}
\newcommand{\Integers}{{\mathbb Z}}
\newcommand{\Znoneg}{{\mathbb Z}^{\mbox{\tiny noneg}}}
\newcommand{\Enoneg}{{\mathbb E}^{\mbox{\tiny noneg}}}
\newcommand{\Qnoneg}{{\mathbb Q}^{\mbox{\tiny noneg}}}
\newcommand{\Rnoneg}{{\mathbb R}^{\mbox{\tiny noneg}}}
\newcommand{\Rationals}{{\mathbb Q}}
\newcommand{\Reals}{{\mathbb R}}
\newcommand{\Complexes}{{\mathbb C}}
%\newcommand{\F2}{{\mathbb F}_{2}}
\newcommand{\relQ}{\mbox{\textsf Q}}
\newcommand{\relR}{\mbox{\textsf R}}
\newcommand{\nrelR}{\mbox{\raisebox{1pt}{$\not$}\rule{1pt}{0pt}{\textsf R}}}
\newcommand{\relS}{\mbox{\textsf S}}
\newcommand{\relA}{\mbox{\textsf A}}
\newcommand{\Dom}[1]{\mbox{Dom}(#1)}
\newcommand{\Cod}[1]{\mbox{Cod}(#1)}
\newcommand{\Rng}[1]{\mbox{Rng}(#1)}

\DeclareMathOperator\caret{\raisebox{1ex}{$\scriptstyle\wedge$}}

\newtheorem*{defi}{Definition}
\newtheorem*{exer}{Exercise}
\newtheorem{thm}{Theorem}[section]
\newtheorem*{thm*}{Theorem}
\newtheorem{lem}[thm]{Lemma}
\newtheorem{cor}{Corollary}
\newtheorem{conj}{Conjecture}

\renewenvironment{proof}%
{\begin{quote} \emph{Proof:} }%
{\rule{0pt}{0pt} \newline \rule{0pt}{15pt} \hfill Q.E.D. \end{quote}}


\newcommand{\vs}{\rule{0pt}{12pt}}

\AtBeginSection[]
{
 \begin{frame}{Table of Contents} 
  \tableofcontents[currentsection]
 \end{frame}
}

%%%% SAVE %%%%
%{ %magic to get a full screen image...
%\setbeamertemplate{navigation symbols}{}  % hide navigation buttons 
%\setbeamertemplate{background canvas}{\centerline{\includegraphics 
%	[height=\paperheight]{Cantor_4.jpeg}}}
%\begin{frame}[plain]
%\rule{0pt}{0pt}
%\end{frame} 
%} %end of magic


\begin{document}

\begin{frame}[plain]
  \titlepage
\end{frame}


\section{intro}

\begin{frame}{functional relationships}
\begin{itemize}
\item A moving object's energy is given by $E=mv^2$.\pause
\item If the mass is fixed, we say there's a {\em functional relationship} between $v$ and $E$\pause
\item We use the term {\em relation} when there's a relationship but it's not necessarily functional. \pause (although it could be.) \pause
\item Relations are Boolean things -- if you provide them with inputs you'll either get True or False. \pause
\item At first we usually think of relations as being between {\em two} things, like $x<y$, but there are other options!
\end{itemize}
\end{frame}

\section{binary relations}

\begin{frame}{first example}
\begin{itemize}
\item $x < y$ \pause
\item What does the graph look like? \pause
\item equivalence to a set of ordered pairs. \pause
\item the graph {\em is} the relation! \pause 
\end{itemize}
\end{frame}

\begin{frame}{negations}
\begin{itemize}
\item If we have a given relation (for example $x < y$) there is another relation that gives exactly the opposite answer.  If $x$ isn't less than $y$ this other relation will be true, and if $x$ {\em is} less than $y$ the new thing will be false.  Do you know what it is? \pause
\item $x \geq y$ \pause
\item Let's compare their graphs.
\end{itemize}
\end{frame}

\begin{frame}{what relations do we know so far?}
\begin{itemize}
\item In the last slide we saw that relations naturally come in pairs. \pause
\item For example, $<$ and $\geq$. \pause
\item What other relations do we know and what are their corresponding negations? \pause
\item Just think about what symbols we know that produce a True/False thing when inserted between 2 numbers.\pause
\item $=, \; <, \; >, \; \leq, \; \geq, \; \divides $
\end{itemize}
\end{frame}

\section{some sets related to relations}
\begin{frame}{}
\begin{itemize}
\item Some relations ($\in$ is a prime example) have different sorts of things on the left and right. \pause
\item The set of things that appear on the left of a relation symbol is called the {\em domain} of the relation. \pause You've certainly encountered the word ``domain'' when talking about functions -- it's the same thing here. \pause
\item The set of things that may appear on the right of a relation symbol is called the {\em codomain} of the relation. \pause
\item If we denote the relation $\relR$, then we write $\Dom{\relR}$ and $\Cod{\relR}$ (respectively) for the domain and codomain. \pause
\item Bonus points: Did anyone notice the one-word difference between the descriptions of domain and codomain?
\end{itemize}
\end{frame}

\begin{frame}{range}
\begin{itemize}
\item There is an inherent assymetry in the way the mathematical world approaches the sets related to a relation. \pause
\item For the items on the right-hand side of the relation we have both the codomain (the things that {\em might} appear) and another set. \pause
\item The {\em range} of a relation $\relR$ is the set of things which actually do appear somewhere as a right-hand side. \pause
\item On the left-hand side we don't make a distinction between the things that might appear and the things that actually do appear. 
\end{itemize}
\end{frame}

\begin{frame}{arbitrary relations}
\begin{itemize}
\item Any arbitrary set of points in the plane may be regarded as a relation whose domain and range consist of real numbers. \pause
\item As a consequence every function is a relation. \pause
\item The inverses of functions (even when they fail VLT) are also relations. \pause
\item But we are free to define literally any subset of the plane and say it's contents constitute a relation.
\end{itemize}
\end{frame}

\begin{frame}{a non-graphical example}
\begin{itemize}
\item There are plenty of examples where graphing a relation is not the best choice. \pause
\item Consider the `$\divides$' relation on the set $\{2,  \ldots , 12\}. $\pause
\item The graph is just a smattering of points in the plane, it's just as informative to simply list them all.\pause
\item 
\begin{align*}
& \{ (2,2), (2,4), (2,6), (2,8), (2,10), (2,12), \\
& (3,3), (3,6), (3,9), (3,12), \\ 
& (4,4), (4, 8), (4,12) \\
& (5,5), (5,10), \\
& (6,6), (6,12), \\
& (7,7), (8,8), (9,9), (10,10), (11,11), (12,12) \} 
\end{align*}
\pause
\item Domain, range, codomain?  \pause What's the negation?
\end{itemize}
\end{frame}

\begin{frame}{other examples}
\begin{itemize}
\item Define a relation $\relR$ by

\[ x \relR y \quad \iff \quad x^2+y^2 \leq 1 \]

\pause

\item As a set, $\relR$ is the boundary, together with the interior, of the unit circle. \pause 

\item Often we create unusual relations by combining the more basic sorts, for instance

\[ x-1 \leq y < x+1. \]
\pause

\item Compound inequalities are either True or False too!
\end{itemize}
\end{frame}

\section{ternary relations}

\begin{frame}{}
\begin{itemize}
\item If there are three variables involved in a relation then the relation may be thought of as a subset of $3$-dimensional space. \pause
\item An example might be the relation where $x$, $y$ and $z$ appear in the natural order:

\[ x < y < z. \]
\pause
\item Such things can be very difficult to visualize. \pause
\item Fortunately, often there's no need!
\end{itemize}
\end{frame}

\begin{frame}{betweenness}
\begin{itemize}
\item The betweenness relation, applies to triples of points that lie on a line. \pause
\item We write $A \ast B \ast C$ if $B$ lies on the segment determined by $A$ and $C$. \pause
\item Notice that $A$ and $C$ behave symmetrically in the betweenness relation, so if 
$A \ast B \ast C$ is true, then so is $C \ast B \ast A$, and {\em vice versa}. \pause
\item If we have a geometrical diagram with a finite number of points, there are only so many possible triples of points that can satisfy ``betweenness.'' 
\end{itemize}
\end{frame}

\begin{frame}{a quaternary relation}
\begin{itemize}
\item The idea we ran into recently of when two fractions represent the same rational number leads to a quaternary relation on $\Naturals$. \pause
\item In other words a subset of $4$-dimensional space. (Good luck trying to visualize it!) \pause
\item Two fraction $a/b$ and $c/d$ are equal when $ad=bc$. (cross-multiplication) \pause
\item So the relation we're after is 
\[ \relR \quad = \quad \{ (x,y,z,w) \in \Reals^4 \; \suchthat \; x,y,z,w \in \Naturals, \; \mbox{and} \; xw=yz \}. \]
\end{itemize}
\end{frame}



\end{document}
