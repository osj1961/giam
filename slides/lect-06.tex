%\documentclass[handout,landscape]{beamer}
\documentclass[landscape]{beamer}
%\hypersetup{pdfpagemode=FullScreen}
\mode<handout>
{
  \usepackage{pgf}
  \usepackage{pgfpages}

\pgfpagesdeclarelayout{6 on 1 boxed}
{
  \edef\pgfpageoptionheight{\the\paperheight} 
  \edef\pgfpageoptionwidth{\the\paperwidth}
  \edef\pgfpageoptionborder{0pt}
}
{
  \pgfpagesphysicalpageoptions
  {%
    logical pages=6,%
    physical height=\pgfpageoptionheight,%
    physical width=\pgfpageoptionwidth%
  }
  \pgfpageslogicalpageoptions{1}
  {%
    border code=\pgfsetlinewidth{2pt}\pgfstroke,%
    border shrink=\pgfpageoptionborder,%
    resized width=.5\pgfphysicalwidth,%
    resized height=.5\pgfphysicalheight,%
    center=\pgfpoint{.25\pgfphysicalwidth}{.833\pgfphysicalheight}%
  }%
  \pgfpageslogicalpageoptions{2}
  {%
    border code=\pgfsetlinewidth{2pt}\pgfstroke,%
    border shrink=\pgfpageoptionborder,%
    resized width=.5\pgfphysicalwidth,%
    resized height=.5\pgfphysicalheight,%
    center=\pgfpoint{.75\pgfphysicalwidth}{.833\pgfphysicalheight}%
  }%
  \pgfpageslogicalpageoptions{3}
  {%
    border code=\pgfsetlinewidth{2pt}\pgfstroke,%
    border shrink=\pgfpageoptionborder,%
    resized width=.5\pgfphysicalwidth,%
    resized height=.5\pgfphysicalheight,%
    center=\pgfpoint{.25\pgfphysicalwidth}{.5\pgfphysicalheight}%
  }%
  \pgfpageslogicalpageoptions{4}
  {%
    border code=\pgfsetlinewidth{2pt}\pgfstroke,%
    border shrink=\pgfpageoptionborder,%
    resized width=.5\pgfphysicalwidth,%
    resized height=.5\pgfphysicalheight,%
    center=\pgfpoint{.75\pgfphysicalwidth}{.5\pgfphysicalheight}%
  }%
  \pgfpageslogicalpageoptions{5}
  {%
    border code=\pgfsetlinewidth{2pt}\pgfstroke,%
    border shrink=\pgfpageoptionborder,%
    resized width=.5\pgfphysicalwidth,%
    resized height=.5\pgfphysicalheight,%
    center=\pgfpoint{.25\pgfphysicalwidth}{.167\pgfphysicalheight}%
  }%
  \pgfpageslogicalpageoptions{6}
  {%
    border code=\pgfsetlinewidth{2pt}\pgfstroke,%
    border shrink=\pgfpageoptionborder,%
    resized width=.5\pgfphysicalwidth,%
    resized height=.5\pgfphysicalheight,%
    center=\pgfpoint{.75\pgfphysicalwidth}{.167\pgfphysicalheight}%
  }%
}


  \pgfpagesuselayout{6 on 1 boxed}[letterpaper, border shrink=5mm]
  \nofiles
}

\usepackage{listings}
%\lstset{language=TeX}
\usepackage{multimedia}
\usepackage[normalem]{ulem}
\usepackage{amssymb}

%\usecolortheme[named=Purple]{structure} 
%\usetheme{Copenhagen}
\usetheme{Warsaw} 
\usecolortheme{seahorse}
\useoutertheme{infolines} 
%\usetheme[height=7mm]{Rochester} 
%\setbeamertemplate{items}[ball] 
\setbeamertemplate{blocks}[rounded][shadow=true] 
%\setbeamertemplate{navigation symbols}{} 
\author{Joe Fields}
\title{Introduction to Proof} 
%\subtitle{}
\date{Lecture 6}
\institute[SCSU]{ {\tt fieldsj1@southernct.edu} }

\newcommand{\versionNum}{$3.2$\ }

\newboolean{InTextBook}
\setboolean{InTextBook}{false}
\newboolean{InWorkBook}
\setboolean{InWorkBook}{false}
\newboolean{InHints}
\setboolean{InHints}{false}

%When this boolean is true (beginning in Section 5.1) we will use the convention
% that $0 \in \Naturals$.  If it is false we will continue to count $1$ as the smallest
%natural number (thus making Giuseppe Peano spin in his grave...)
 
\newboolean{ZeroInNaturals}

%This boolean is used to distinguish the version where we use $\sim$ rather than $\lnot$

\newboolean{LNotIsSim}

%The values of the last two booleans are set in ``switches.tex''

%\input{switches}

\let\savedlnot\lnot
\ifthenelse{\boolean{LNotIsSim}}{\renewcommand{\lnot}{\sim} }{}

%This command puts different amounts of space depending on whether we are
% in the text, the workbook or the hints & solutions manual. 
\newcommand{\twsvspace}[3]{%
 \ifthenelse{\boolean{InTextBook} }{\vspace{#1}}{%
  \ifthenelse{\boolean{InWorkBook} }{\vspace{#2}}{%
   \ifthenelse{\boolean{InHints} }{\vspace{#3}}{} %
   }%
  }%
 }


\newcommand{\wbvfill}{\ifthenelse{\boolean{InWorkBook}}{\vfill}{}}
\newcommand{\wbitemsep}{\ifthenelse{\boolean{InWorkBook} }{\rule[-24pt]{0pt}{60pt}}{}}
\newcommand{\textbookpagebreak}{\ifthenelse{\boolean{InTextBook}}{\newpage}{}}
\newcommand{\workbookpagebreak}{\ifthenelse{\boolean{InWorkBook}}{\newpage}{}}
\newcommand{\hintspagebreak}{\ifthenelse{\boolean{InHints}}{\newpage}{}}

\newcommand{\hint}[1]{\ifthenelse{\boolean{InHints}}{ {\par \hspace{12pt} \color[rgb]{0,0,1} #1 } }{}}
\newcommand{\inlinehint}[1]{\ifthenelse{\boolean{InHints}}{ { \color[rgb]{0,0,1} #1 } }{}}

%\newlength{\cwidth}
%\newcommand{\cents}{\settowidth{\cwidth}{c}%
%\divide\cwidth by2
%\advance\cwidth by-.1pt
%c\kern-\cwidth
%\vrule width .1pt depth.2ex height1.2ex
%\kern 3\cwidth}
\newcommand{\cents}{\textcent\kern 5pt}

\newcommand{\sageprompt}{ {\tt sage$>$} }
\newcommand{\tab}{\rule{20pt}{0pt}}
\newcommand{\blnk}{\rule{1.5pt}{0pt}\rule{.4pt}{1.2pt}\rule{9pt}{.4pt}\rule{.4pt}{1.2pt}\rule{1.5pt}{0pt}}
\newcommand{\suchthat}{\; \rule[-3pt]{.5pt}{13pt} \;}
\newcommand{\divides}{\!\mid\!}
\newcommand{\tdiv}{\; \mbox{div} \;}
\newcommand{\restrict}[2]{#1 \,\rule[-4pt]{.25pt}{14pt}_{\,#2}}
\newcommand{\lcm}[2]{\mbox{lcm} (#1, #2)}
\renewcommand{\gcd}[2]{\mbox{gcd} (#1, #2)}
\newcommand{\Naturals}{{\mathbb N}}
\newcommand{\Integers}{{\mathbb Z}}
\newcommand{\Znoneg}{{\mathbb Z}^{\mbox{\tiny noneg}}}
\ifthenelse{\boolean{ZeroInNaturals}}{%
  \newcommand{\Zplus}{{\mathbb Z}^+} }{%
  \newcommand{\Zplus}{{\mathbb N}} }
\newcommand{\Enoneg}{{\mathbb E}^{\mbox{\tiny noneg}}}
\newcommand{\Qnoneg}{{\mathbb Q}^{\mbox{\tiny noneg}}}
\newcommand{\Rnoneg}{{\mathbb R}^{\mbox{\tiny noneg}}}
\newcommand{\Rationals}{{\mathbb Q}}
\newcommand{\Reals}{{\mathbb R}}
\newcommand{\Complexes}{{\mathbb C}}
%\newcommand{\F2}{{\mathbb F}_{2}}
\newcommand{\relQ}{\mbox{\textsf Q}}
\newcommand{\relR}{\mbox{\textsf R}}
\newcommand{\nrelR}{\mbox{\raisebox{1pt}{$\not$}\rule{1pt}{0pt}{\textsf R}}}
\newcommand{\relS}{\mbox{\textsf S}}
\newcommand{\relA}{\mbox{\textsf A}}
\newcommand{\Dom}[1]{\mbox{Dom}(#1)}
\newcommand{\Cod}[1]{\mbox{Cod}(#1)}
\newcommand{\Rng}[1]{\mbox{Rng}(#1)}

\DeclareMathOperator\caret{\raisebox{1ex}{$\scriptstyle\wedge$}}

\newtheorem*{defi}{Definition}
\newtheorem*{exer}{Exercise}
\newtheorem{thm}{Theorem}[section]
\newtheorem*{thm*}{Theorem}
\newtheorem{lem}[thm]{Lemma}
\newtheorem*{lem*}{Lemma}
\newtheorem{cor}{Corollary}
\newtheorem{conj}{Conjecture}

\renewenvironment{proof}%
{\begin{quote} \emph{Proof:} }%
{\rule{0pt}{0pt} \newline \rule{0pt}{15pt} \hfill Q.E.D. \end{quote}}


\newcommand{\vs}{\rule{0pt}{12pt}}

\AtBeginSection[]
{
 \begin{frame}{Table of Contents} 
  \tableofcontents[currentsection]
 \end{frame}
}

%%%% SAVE %%%%
%{ %magic to get a full screen image...
%\setbeamertemplate{navigation symbols}{}  % hide navigation buttons 
%\setbeamertemplate{background canvas}{\centerline{\includegraphics 
%	[height=\paperheight]{Cantor_4.jpeg}}}
%\begin{frame}[plain]
%\rule{0pt}{0pt}
%\end{frame} 
%} %end of magic


\begin{document}

\begin{frame}[plain]
  \titlepage
\end{frame}


\section{{\em no} divisors in common?}

\begin{frame}{gcd equals one}
\begin{itemize}
\item Notice that since $1$ is a divisor of any integer, there's always at least $1$ as a common entry in two number's lists of divisors. \pause
\item Does it actually happen that for some values of $a$ and $b$, $\gcd{a}{b} = 1$? \pause
\item Sure!  For example if $a$ and $b$ are both prime.\pause
\item This also happens when the numbers are not prime but their factorizations don't have any primes in common. \pause
\item Example: $33$ and $35$. \pause
\item Or $41,140$ and $46,683$???
\end{itemize}
\end{frame}

\begin{frame}{relative primality}
\begin{itemize}
\item When $\gcd{a}{b} = 1$, we say that $a$ and $b$ are {\em relatively prime}. \pause
\item Of course two distinct prime numbers will be relatively prime, but there are other possibilities\textellipsis \pause
\item Notice that when a fraction is in lowest terms (aka simplest form), the numerator and denomenator will be relatively prime.\pause
\item Example: $\displaystyle \frac{186}{210} $ \pause \hspace{.1in} $\displaystyle = \; \frac{31}{35}.$ 
\end{itemize}
\end{frame}

\section{be rational}

\begin{frame}{Righting a wrong}
\begin{itemize}
\item The definition we gave for the rational numbers:

\[  \Rationals = \{ \frac{a}{b} \suchthat a \in \Integers \; \mbox{and} \;
b \in \Integers \; \mbox{and} \; b \neq 0 \} \]

\noindent has an issue. \pause

\item We're failing to take account of the fact that the same rational number can be written in multiple ways. \pause

\item In a sense this is okay since when duplicates occur in a set we automatically drop them. \pause

\item But in another sense it's problematic because how are we supposed to know that $2/3$ and $4/6$ are actually the same?  (I know you know they're the same, but shouldn't the definition of $\Rationals$ be the place where issues like that get sorted?) \pause

\item The final officially approved version of $\Rationals$:

\[  \Rationals = \{ \frac{a}{b} \suchthat a \in \Integers \; \mbox{and} \;
b \in \Integers \; \mbox{and} \; b \neq 0 \; \mbox{and} \; \gcd{a}{b} = 1 \} \]

\end{itemize}
\end{frame}

\begin{frame}{Another translation job}
\begin{itemize}
\item Let's practice parsing math into english: \pause

\vspace{.2in}

\begin{tabular}{c|c|c}
\rule[-10pt]{0pt}{22pt} $\Rationals$ & $=$ & $\{$  \\ \hline
\rule[-6pt]{0pt}{22pt} The rational numbers & are defined to be & the set of all\\
\end{tabular}

\vspace{.2in}

\begin{tabular}{c|c|c}
\rule[-10pt]{0pt}{22pt} $\displaystyle \frac{a}{b}$ & $\suchthat$ & $a,b \in \Integers$ \\ \hline
\rule[-6pt]{0pt}{22pt} fractions of the form $a$ over $b$ & such that
& $a$ and $b$ are integers \\
\end{tabular}

\vspace{.2in}

\begin{tabular}{c|c|c|c|c}
\rule[-10pt]{0pt}{22pt} and & $b \neq 0$ & and & $\gcd{a}{b}=1$ & $\}$
\\ \hline
\rule[-6pt]{0pt}{22pt}  and & $b$ is non-zero & and & $a$ and $b$ are relatively prime. &  \\
\end{tabular}

\end{itemize}
\end{frame}

\section{Hippasus of Metapontum}


\begin{frame}{are there irrational numbers?}
\begin{itemize}
\item Pythagoras -- ``all is number'' \pause
\item For quite a while people believed that the rational numbers were the biggest set of numbers we needed -- in other words they thought all physical quantities (lengths, weights, areas, etc) were rational.\pause
\item You probably {\em believe} in irrational numbers (real numbers that aren't rational) because you know that the decimal expansion of things like $\pi$ and $\sqrt{3}/2$ don't have a repeating pattern. \pause
\item But if you're being honest with yourself, you probably just took somebody's word for that.\pause
\item Hippasus {\em proved it!} (at least for $\sqrt{2}$) \pause
\item and then what happened?
\end{itemize}
\end{frame}

\begin{frame}{an elegant argument}
\begin{itemize}
\item G.H.\ Hardy -- {\em A Mathematician's Apology}. \pause
\item Available in many formats at the Internet Archive.\newline
\href{https://archive.org/details/AMathematiciansApology/}{https://archive.org/details/AMathematiciansApology/} \pause
\item The proof that we're about to see is one of two that Hardy chose to illustrate the concept of {\em elegance}. \pause
\item The basic form of the argument is {\em reductio ad absurdam}. \pause
\item `Proof by contradiction' in modern parlance.
\end{itemize}
\end{frame}

\begin{frame}{lemmas}
\begin{itemize}
\item What's yellow and equivalent to the axiom of choice? \pause
\item Zorn's lemma. \pause
\item A {\em lemma} is a result that is often proved only for the purpose of proving something else. \pause
\item It doesn't necessarily mean that it is of lower status than a {\em theorem}\textellipsis
\pause
\item For Hippasus' proof we need a lemma that connects the evenness of a square to the evenness of the corresponding (unsquared) number. \pause
\item $\displaystyle \forall x \in \Naturals, \; \mbox{if} \; x^2 \; \mbox{is even, then} \; x \mbox{is even}.$ \pause
\item sage 
\end{itemize}
\end{frame}

\begin{frame}{the proof}

\begin{theorem}
$\sqrt{2} \notin \Rationals$
\end{theorem}

\emph{Proof:}
Suppose to the contrary that $\sqrt{2}$ {\em is} a rational number.
Then by the definition of the set of rational numbers, we know that
there are integers 
$a$ and $b$ having the following properties: 
$\displaystyle \sqrt{2} = \frac{a}{b}$ and $\gcd{a}{b} = 1$.  

Consider the expression $\displaystyle \sqrt{2} = \frac{a}{b}$.   
By squaring both sides of this we obtain

\[ 2 = \frac{a^2}{b^2}. \]

This last expression can be rearranged to give

\begin{equation*}
a^2 = 2 b^2
\end{equation*}
\end{frame}

\begin{frame}{cont}

An immediate consequence of this last equation is that $a^2$ is an
even number.  Using the lemma, we deduce that $a$ is an even
integer and hence that there is an integer $m$ such that $a=2m$.
Substituting this last expression into the previous equation gives

\begin{equation*}
(2m)^2 = 2 b^2,
\end{equation*}

thus,

\begin{equation*}
4m^2 = 2 b^2,
\end{equation*}

\end{frame}

\begin{frame}{cont}

so

\begin{equation*}
2m^2 = b^2.
\end{equation*}

This tells us that $b^2$ is even, and hence (by the lemma), $b$ is even.

Finally, we have arrived at the desired absurdity because if $a$ and
$b$ are both even then $\gcd{a}{b} \geq 2$, but, on the other hand,
one of our initial assumptions is that $\gcd{a}{b} = 1$. 

\rule{0pt}{0pt} \newline \rule{0pt}{15pt} \hfill Q.E.D.

\end{frame}


\end{document}
