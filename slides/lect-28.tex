\ifdefined\ishandout
  \documentclass[handout,landscape]{beamer} 
\else
  \documentclass[landscape]{beamer}
\fi

%\hypersetup{pdfpagemode=FullScreen} %Enabling this option will cause the slides to go full-screen on opening

\mode<handout>
{
  \usepackage{pgf}
  \usepackage{pgfpages}

\pgfpagesdeclarelayout{6 on 1 boxed}
{
  \edef\pgfpageoptionheight{\the\paperheight} 
  \edef\pgfpageoptionwidth{\the\paperwidth}
  \edef\pgfpageoptionborder{0pt}
}
{
  \pgfpagesphysicalpageoptions
  {%
    logical pages=6,%
    physical height=\pgfpageoptionheight,%
    physical width=\pgfpageoptionwidth%
  }
  \pgfpageslogicalpageoptions{1}
  {%
    border code=\pgfsetlinewidth{1pt}\pgfstroke,%
    border shrink=\pgfpageoptionborder,%
    resized width=.5\pgfphysicalwidth,%
    resized height=.5\pgfphysicalheight,%
    center=\pgfpoint{.25\pgfphysicalwidth}{.833\pgfphysicalheight}%
  }%
  \pgfpageslogicalpageoptions{2}
  {%
    border code=\pgfsetlinewidth{1pt}\pgfstroke,%
    border shrink=\pgfpageoptionborder,%
    resized width=.5\pgfphysicalwidth,%
    resized height=.5\pgfphysicalheight,%
    center=\pgfpoint{.75\pgfphysicalwidth}{.833\pgfphysicalheight}%
  }%
  \pgfpageslogicalpageoptions{3}
  {%
    border code=\pgfsetlinewidth{1pt}\pgfstroke,%
    border shrink=\pgfpageoptionborder,%
    resized width=.5\pgfphysicalwidth,%
    resized height=.5\pgfphysicalheight,%
    center=\pgfpoint{.25\pgfphysicalwidth}{.5\pgfphysicalheight}%
  }%
  \pgfpageslogicalpageoptions{4}
  {%
    border code=\pgfsetlinewidth{1pt}\pgfstroke,%
    border shrink=\pgfpageoptionborder,%
    resized width=.5\pgfphysicalwidth,%
    resized height=.5\pgfphysicalheight,%
    center=\pgfpoint{.75\pgfphysicalwidth}{.5\pgfphysicalheight}%
  }%
  \pgfpageslogicalpageoptions{5}
  {%
    border code=\pgfsetlinewidth{1pt}\pgfstroke,%
    border shrink=\pgfpageoptionborder,%
    resized width=.5\pgfphysicalwidth,%
    resized height=.5\pgfphysicalheight,%
    center=\pgfpoint{.25\pgfphysicalwidth}{.167\pgfphysicalheight}%
  }%
  \pgfpageslogicalpageoptions{6}
  {%
    border code=\pgfsetlinewidth{1pt}\pgfstroke,%
    border shrink=\pgfpageoptionborder,%
    resized width=.5\pgfphysicalwidth,%
    resized height=.5\pgfphysicalheight,%
    center=\pgfpoint{.75\pgfphysicalwidth}{.167\pgfphysicalheight}%
  }%
}


  \pgfpagesuselayout{6 on 1 boxed}[letterpaper, border shrink=5mm]
  \nofiles
}

\usepackage{listings}
\usepackage{multimedia}
\usepackage[normalem]{ulem}
\usepackage{ifthen}
\usepackage{textcomp}

\usetheme{Warsaw} 
\usecolortheme{seahorse}
\useoutertheme{infolines} 

\setbeamertemplate{blocks}[rounded][shadow=true] 

\author{Joe Fields}
\title{Introduction to Proof} 

\date{Lecture 28 (GIAM \S 5.4) \newline the strong form of mathematical induction}
\institute[SCSU]{ {\tt fieldsj1@southernct.edu} }

\newcommand{\versionNum}{$3.2$\ }

\newboolean{InTextBook}
\setboolean{InTextBook}{false}
\newboolean{InWorkBook}
\setboolean{InWorkBook}{false}
\newboolean{InHints}
\setboolean{InHints}{false}

%When this boolean is true (beginning in Section 5.1) we will use the convention
% that $0 \in \Naturals$.  If it is false we will continue to count $1$ as the smallest
%natural number (thus making Giuseppe Peano spin in his grave...)
 
\newboolean{ZeroInNaturals}

%This boolean is used to distinguish the version where we use $\sim$ rather than $\lnot$

\newboolean{LNotIsSim}

%The values of the last two booleans are set in ``switches.tex''

%\input{switches}

\let\savedlnot\lnot
\ifthenelse{\boolean{LNotIsSim}}{\renewcommand{\lnot}{\sim} }{}

%This command puts different amounts of space depending on whether we are
% in the text, the workbook or the hints & solutions manual. 
\newcommand{\twsvspace}[3]{%
 \ifthenelse{\boolean{InTextBook} }{\vspace{#1}}{%
  \ifthenelse{\boolean{InWorkBook} }{\vspace{#2}}{%
   \ifthenelse{\boolean{InHints} }{\vspace{#3}}{} %
   }%
  }%
 }


\newcommand{\wbvfill}{\ifthenelse{\boolean{InWorkBook}}{\vfill}{}}
\newcommand{\wbitemsep}{\ifthenelse{\boolean{InWorkBook} }{\rule[-24pt]{0pt}{60pt}}{}}
\newcommand{\textbookpagebreak}{\ifthenelse{\boolean{InTextBook}}{\newpage}{}}
\newcommand{\workbookpagebreak}{\ifthenelse{\boolean{InWorkBook}}{\newpage}{}}
\newcommand{\hintspagebreak}{\ifthenelse{\boolean{InHints}}{\newpage}{}}

\newcommand{\hint}[1]{\ifthenelse{\boolean{InHints}}{ {\par \hspace{12pt} \color[rgb]{0,0,1} #1 } }{}}
\newcommand{\inlinehint}[1]{\ifthenelse{\boolean{InHints}}{ { \color[rgb]{0,0,1} #1 } }{}}

%\newlength{\cwidth}
%\newcommand{\cents}{\settowidth{\cwidth}{c}%
%\divide\cwidth by2
%\advance\cwidth by-.1pt
%c\kern-\cwidth
%\vrule width .1pt depth.2ex height1.2ex
%\kern 3\cwidth}
\newcommand{\cents}{\textcent\kern 5pt}

\newcommand{\sageprompt}{ {\tt sage$>$} }
\newcommand{\tab}{\rule{20pt}{0pt}}
\newcommand{\blnk}{\rule{1.5pt}{0pt}\rule{.4pt}{1.2pt}\rule{9pt}{.4pt}\rule{.4pt}{1.2pt}\rule{1.5pt}{0pt}}
\newcommand{\suchthat}{\; \rule[-3pt]{.5pt}{13pt} \;}
\newcommand{\divides}{\!\mid\!}
\newcommand{\tdiv}{\; \mbox{div} \;}
\newcommand{\restrict}[2]{#1 \,\rule[-4pt]{.25pt}{14pt}_{\,#2}}
\newcommand{\lcm}[2]{\mbox{lcm} (#1, #2)}
\renewcommand{\gcd}[2]{\mbox{gcd} (#1, #2)}
\newcommand{\Naturals}{{\mathbb N}}
\newcommand{\Integers}{{\mathbb Z}}
\newcommand{\Znoneg}{{\mathbb Z}^{\mbox{\tiny noneg}}}
\ifthenelse{\boolean{ZeroInNaturals}}{%
  \newcommand{\Zplus}{{\mathbb Z}^+} }{%
  \newcommand{\Zplus}{{\mathbb N}} }
\newcommand{\Enoneg}{{\mathbb E}^{\mbox{\tiny noneg}}}
\newcommand{\Qnoneg}{{\mathbb Q}^{\mbox{\tiny noneg}}}
\newcommand{\Rnoneg}{{\mathbb R}^{\mbox{\tiny noneg}}}
\newcommand{\Rationals}{{\mathbb Q}}
\newcommand{\Reals}{{\mathbb R}}
\newcommand{\Complexes}{{\mathbb C}}
%\newcommand{\F2}{{\mathbb F}_{2}}
\newcommand{\relQ}{\mbox{\textsf Q}}
\newcommand{\relR}{\mbox{\textsf R}}
\newcommand{\nrelR}{\mbox{\raisebox{1pt}{$\not$}\rule{1pt}{0pt}{\textsf R}}}
\newcommand{\relS}{\mbox{\textsf S}}
\newcommand{\relA}{\mbox{\textsf A}}
\newcommand{\Dom}[1]{\mbox{Dom}(#1)}
\newcommand{\Cod}[1]{\mbox{Cod}(#1)}
\newcommand{\Rng}[1]{\mbox{Rng}(#1)}

\DeclareMathOperator\caret{\raisebox{1ex}{$\scriptstyle\wedge$}}

\newtheorem*{defi}{Definition}
\newtheorem*{exer}{Exercise}
\newtheorem{thm}{Theorem}[section]
\newtheorem*{thm*}{Theorem}
\newtheorem{lem}[thm]{Lemma}
\newtheorem*{lem*}{Lemma}
\newtheorem{cor}{Corollary}
\newtheorem{conj}{Conjecture}

\renewenvironment{proof}%
{\begin{quote} \emph{Proof:} }%
{\rule{0pt}{0pt} \newline \rule{0pt}{15pt} \hfill Q.E.D. \end{quote}}


\newcommand{\vs}{\rule{0pt}{11pt}}
\newcommand{\notimplies}{\;\not\!\!\!\implies}
\newcommand{\dx}{\,\mbox{d}x}

\AtBeginSection[]
{
 \begin{frame}{Table of Contents} 
  \tableofcontents[currentsection]
 \end{frame}
}

%%%% SAVE %%%%
%{ %magic to get a full screen image...
%\setbeamertemplate{navigation symbols}{}  % hide navigation buttons 
%\setbeamertemplate{background canvas}{\centerline{\includegraphics 
%	[height=\paperheight]{Cantor_4.jpeg}}}
%\begin{frame}[plain]
%\rule{0pt}{0pt}
%\end{frame} 
%} %end of magic


\begin{document}

\begin{frame}[plain]
  \titlepage
\end{frame}

\section{intro}

\begin{frame}{strong induction}
\begin{itemize}
\item Strong induction goes by many names: \pause
\begin{itemize}
\item Strong induction. \pause
\item Complete induction. \pause
\item Course of values induction. \pause
\end{itemize}
\end{itemize}
\end{frame}


\begin{frame}{the principle of complete induction}
\begin{itemize}
\item $ \displaystyle \forall k (P_0 \land P_1 \land \ldots \land  P_{k-1}) \implies P_k.$
\item PMI and PCI are logically equivalent. \pause
\item Nevertheless, there are times when PCI is much more convenient. \pause
\item You get to use the truth of {\em all} the statements prior to $P_n$ to prove it. \pause
\item That's what makes it ``strong'' -- the hypotheses are much stronger!
\end{itemize}
\end{frame}

\begin{frame}{outline}
\begin{center}
\begin{tabular}{|c|} \hline
\rule{16pt}{0pt}\begin{minipage}{.75\textwidth}

\rule{0pt}{16pt}{\bf \large Theorem} $ \displaystyle \forall n \in \Naturals, \; P_n $
\medskip

\rule{0pt}{20pt} {\em Proof:} (By complete induction)

\noindent {\bf Basis:}

\begin{center}
$\vdots$ \rule{36pt}{0pt} \begin{minipage}[c]{2.3 in} (Technically, a PCI %
proof doesn't require a basis.   We recommend that you show that $P_0$ %
is true anyway.) \end{minipage}
\end{center}

\noindent {\bf Inductive step:}

\begin{center}
$\vdots$ \rule{36pt}{0pt} \begin{minipage}[c]{2.3 in} (Here we must show that $\forall k,  \left( \bigwedge_{i=0}^{k-1} P_i \right) \implies P_{k}$ is true.) \end{minipage}
\end{center}

\rule{0pt}{0pt} \hspace{\fill} Q.E.D. \rule[-10pt]{0pt}{16pt}
\end{minipage} \rule{16pt}{0pt} \\ \hline
\end{tabular}
\end{center}
\end{frame}

\begin{frame}{an issue}
\begin{itemize}
\item When $n=0$ in a PCI proof the inductive step looks like this: \pause
\item (The conjunction of an empty set of statements) \rule{1pt}{0pt} $\implies \; P_0$. \pause
\item The conjunction of {\em no} statements should be equivalent to the identity for conjunction. \pause
\item So if you do the inductive step in a strong induction argument correctly it should include the case where $P_0$ is implied by a tautology.\pause
\item This is really the base case! \pause
\item It's so easy to overlook this that I recommend doing a seperate base case, even though technically it isn't necessary.
\end{itemize}
\end{frame}

\begin{frame}{equivalent?}
\begin{itemize}
\item If a statement can be proved by ordinary induction, then it can certainly be proved by strong induction. \pause
\item In the strong induction version you know that all the statements from $P_0$ to $P_{n-1}$ are true,\pause \newline
 so in particular you know that $P_{n-1}$ is true \pause \newline
so you can just use the ordinary induction argument from there. \pause
\item To show the other implication we need to create a new statement family that's related to the $P$'s: \pause
\[ Q_k \; = \; \forall i \in \{0, \ldots k\}, \; P_i \; \mbox{is true.} \] \pause
\item Proving $Q_{k-1} \implies Q_k$ is really the same thing as the strong induction argument, but the $Q$ statement family can be proved using ordinary induction!
\end{itemize}
\end{frame}

\section{examples}

\begin{frame}{The fundamental theorem of arithmetic}
\begin{itemize}
\item Many areas of mathematics have something called the ``fundamental theorem.'' \pause
\item The fundamental theorem of calculus is
\[ \int_a^b f(x) \dx \; = \; F(b) - F(a), \; \mbox{where} \; F(x) \; \mbox{is an antiderivative of} \; f(x).\] \pause
\item The fundamental theorem of algebra is that every polynomial with real coeffiecients can be factored into a product of linear and irreducible quadratic terms. \pause
\item The fundamental theorem of arithmetic is that every positive integer has a unique representation as a product of prime powers. 

\end{itemize}
\end{frame}

\begin{frame}{comments}
\begin{itemize}
\item Strong induction is necessary in proving the fundamental theorem of arithmetic because the prime factorization of $k$ and $k-1$ have nothing to do with one another! \pause
\item To get started on proving the fundamental theorem of arithmetic we first prove a lemma: \pause
\end{itemize}

\begin{lem*}
For every natural number $n$, if $n>1$, then $n$ has a prime factor.
\end{lem*}
\end{frame}

\begin{frame}{proof}
\begin{proof} (By strong induction)
Consider an arbitrary natural number $n>1$.  If $n$ is prime then $n$ clearly
has a prime factor (itself), so suppose that $n$ is not prime.  By 
definition, a composite
natural number can be factored, so $n=a \cdot b$ for some pair of natural
numbers $a$ and $b$ which are both greater than 1.  Since $a$ and $b$ are  
factors of $n$ both greater than 1, it follows that $a < n$ (it is also 
true that $b < n$ but we don't need that \ldots).  The inductive hypothesis
can now be applied to deduce that $a$ has a prime factor $p$.  Since
$p \divides a$ and $a \divides n$, by transitivity $p \divides n$.  Thus
$n$ has a prime factor.
\end{proof}
\end{frame}

\begin{frame}{recurrences}
\begin{itemize}
\item Most people have heard of the Fibonacci numbers, which are defined recursively by 
\[ F_{n} \; = \; F_{n-1} \, + \, F_{n-2}. \] \pause
\item There are {\em many} other sequences of numbers that are also defined recursively. \pause
\item When proving something about such a sequence it will be handy to know that that something is true for all the earlier values used in the recurrence! \pause 
\end{itemize}
\end{frame}

\begin{frame}{a formula}
\begin{itemize}
\item This example has been carefully cooked up so that the numbers work out nicely.\pause
\item A sequence is defined recursively by the formula
\[ X_n \; = \; X_{n-1} \, + \, 2X_{n-2}. \] \pause
\item ``The next number is the sum of the previous and twice the one before that.''
\item The initial values $X_0 = 2$ and $X_1 = 1$ are also given to get things started. \pause
\item From there we can use the recursion to find $X_2 = 5$, $X_3 = 7$, $X_4 = 17$, {\em et cetera}. \pause
\item It turns out that there is a relatively easy formula for this sequence: \pause
\[ X_n \; = \; 2^n \, + \, (-1)^n. \]

\end{itemize}
\end{frame}

\begin{frame}{the result}
\begin{thm*}
\[ \forall n \in \Naturals, \; n \geq 2 \; \implies \; X_n  \; = \; 2^n \, + \, (-1)^n. \]
\end{thm*}
\pause
\begin{itemize}
\item Let's prove that using PCI. \pause
\item Notice that issue about base cases in a PCI proof is present.
\end{itemize}
\end{frame}

\begin{frame}{proof}
\begin{itemize}
\item Notice that when $n=2$ the recursion for $X_n$ refers to values that are not covered by the theorem. \pause
\item We note that $X_0 = 2$ and that is also equal to $2^0+(-1)^0$.  Additionally, note that $X_1 = 1$ and that is also equal to $2^1 + (-1)^1$. \pause
\item Suppose $k \geq 3$ is an integer such that for all $j<k$ we have $X_j = 2^j+(-1)^j$. \pause
\item In particular we need that $X_{k-1} = 2^{k-1}+(-1)^{k-1}$ and $X_{k-2} = 2^{k-2}+(-1)^{k-2}$. \pause
\item Then, 
\end{itemize}
\end{frame}

\begin{frame}{proof continued}
\begin{align*}
X_k \; &= \; X_{k-1} \, + \, 2 X_{k-2} & & \mbox{the given recurrence} \\
       &= \; (2^{k-1}\,+\,(-1)^{k-1}) \; + \; 2(2^{k-2}\,+\,(-1)^{k-2}) & & \mbox{by the inductive hypothesis}\\
       &= \; (2^{k-1}\,+\,2(2^{k-2})) \; + \; ((-1)^{k-1}\,+\,2(-1)^{k-2}) & & \mbox{algebra}\\
       &= \; (2^{k-1}\,+\,2^{k-1}) \; + \; (-1)^{k-2}((-1)^1\,+\,2) & & \mbox{more algebra}\\
       &= \; (2^{k}) \; + \; (-1)^{k-2}(1) & & \mbox{more more algebra}\\
       &= \; (2^{k}) \; + \; (-1)^{k-2}(-1)^2 & & \mbox{more more more algebra}\\
       &= \; (2^{k}) \; + \; (-1)^{k} & & \mbox{as desired}\\
\end{align*}

\hfill {\em Q.E.D.}

\end{frame}

\begin{frame}{usps}
\begin{itemize}
\item A postage stamp problem involves showing that given two types of stamps we can use them to achieve any amount of postage. \pause
\item There will be ``holes'' early on -- postages that can't be achieved exactly. \pause
\item But after some a finite number of initial misses, every postage can be achieved. \pause
\item For instance consider the postages that can be created using $3$\textcent\rule{1pt}{0pt} and $5$\textcent\rule{1pt}{0pt} stamps: \pause
\[ \{ 0, 3, 5, 6, 8, 9, 10, 11, \ldots \} \] \pause
\item As soon as we have three consecutive achievable postages we get everything. \pause \newline
(we can add some number $3$\textcent\rule{1pt}{0pt} stamps to one of those to get any larger value.)
\end{itemize}
\end{frame}

\begin{frame}{proof}
\begin{thm*}
Any postage greater than $7$\textcent\rule{1pt}{0pt} can be achieved exactly using only $3$\textcent\rule{1pt}{0pt} and $5$\textcent\rule{1pt}{0pt} stamps.
\end{thm*} \pause
\begin{itemize}
\item The proof can be completed using PCI. \pause
\item A simpler approach (IMHO) is to combine ordinary induction with the three exhaustive cases, $x=3k+8$, $x=3k+9$ and $x=3k+10$.
\end{itemize}

\end{frame}

\end{document}
