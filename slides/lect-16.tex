%\documentclass[handout,landscape]{beamer}
\documentclass[landscape]{beamer}
%\hypersetup{pdfpagemode=FullScreen}
\mode<handout>
{
  \usepackage{pgf}
  \usepackage{pgfpages}

\pgfpagesdeclarelayout{6 on 1 boxed}
{
  \edef\pgfpageoptionheight{\the\paperheight} 
  \edef\pgfpageoptionwidth{\the\paperwidth}
  \edef\pgfpageoptionborder{0pt}
}
{
  \pgfpagesphysicalpageoptions
  {%
    logical pages=6,%
    physical height=\pgfpageoptionheight,%
    physical width=\pgfpageoptionwidth%
  }
  \pgfpageslogicalpageoptions{1}
  {%
    border code=\pgfsetlinewidth{2pt}\pgfstroke,%
    border shrink=\pgfpageoptionborder,%
    resized width=.5\pgfphysicalwidth,%
    resized height=.5\pgfphysicalheight,%
    center=\pgfpoint{.25\pgfphysicalwidth}{.833\pgfphysicalheight}%
  }%
  \pgfpageslogicalpageoptions{2}
  {%
    border code=\pgfsetlinewidth{2pt}\pgfstroke,%
    border shrink=\pgfpageoptionborder,%
    resized width=.5\pgfphysicalwidth,%
    resized height=.5\pgfphysicalheight,%
    center=\pgfpoint{.75\pgfphysicalwidth}{.833\pgfphysicalheight}%
  }%
  \pgfpageslogicalpageoptions{3}
  {%
    border code=\pgfsetlinewidth{2pt}\pgfstroke,%
    border shrink=\pgfpageoptionborder,%
    resized width=.5\pgfphysicalwidth,%
    resized height=.5\pgfphysicalheight,%
    center=\pgfpoint{.25\pgfphysicalwidth}{.5\pgfphysicalheight}%
  }%
  \pgfpageslogicalpageoptions{4}
  {%
    border code=\pgfsetlinewidth{2pt}\pgfstroke,%
    border shrink=\pgfpageoptionborder,%
    resized width=.5\pgfphysicalwidth,%
    resized height=.5\pgfphysicalheight,%
    center=\pgfpoint{.75\pgfphysicalwidth}{.5\pgfphysicalheight}%
  }%
  \pgfpageslogicalpageoptions{5}
  {%
    border code=\pgfsetlinewidth{2pt}\pgfstroke,%
    border shrink=\pgfpageoptionborder,%
    resized width=.5\pgfphysicalwidth,%
    resized height=.5\pgfphysicalheight,%
    center=\pgfpoint{.25\pgfphysicalwidth}{.167\pgfphysicalheight}%
  }%
  \pgfpageslogicalpageoptions{6}
  {%
    border code=\pgfsetlinewidth{2pt}\pgfstroke,%
    border shrink=\pgfpageoptionborder,%
    resized width=.5\pgfphysicalwidth,%
    resized height=.5\pgfphysicalheight,%
    center=\pgfpoint{.75\pgfphysicalwidth}{.167\pgfphysicalheight}%
  }%
}


  \pgfpagesuselayout{6 on 1 boxed}[letterpaper, border shrink=5mm]
  \nofiles
}

\usepackage{listings}
%\lstset{language=TeX}
\usepackage{multimedia}
\usepackage[normalem]{ulem}
\usepackage{amssymb}

\usepackage{ifthen}

%\usecolortheme[named=Purple]{structure} 
%\usetheme{Copenhagen}
\usetheme{Warsaw} 
\usecolortheme{seahorse}
\useoutertheme{infolines} 
%\usetheme[height=7mm]{Rochester} 
%\setbeamertemplate{items}[ball] 
\setbeamertemplate{blocks}[rounded][shadow=true] 
%\setbeamertemplate{navigation symbols}{} 
\author{Joe Fields}
\title{Introduction to Proof} 
%\subtitle{}
\date{Lecture 16 (GIAM \S 3.3)}
\institute[SCSU]{ {\tt fieldsj1@southernct.edu} }

\newcommand{\versionNum}{$3.2$\ }

\newboolean{InTextBook}
\setboolean{InTextBook}{false}
\newboolean{InWorkBook}
\setboolean{InWorkBook}{false}
\newboolean{InHints}
\setboolean{InHints}{false}

%When this boolean is true (beginning in Section 5.1) we will use the convention
% that $0 \in \Naturals$.  If it is false we will continue to count $1$ as the smallest
%natural number (thus making Giuseppe Peano spin in his grave...)
 
\newboolean{ZeroInNaturals}

%This boolean is used to distinguish the version where we use $\sim$ rather than $\lnot$

\newboolean{LNotIsSim}

%The values of the last two booleans are set in ``switches.tex''

%\input{switches}

\let\savedlnot\lnot
\ifthenelse{\boolean{LNotIsSim}}{\renewcommand{\lnot}{\sim} }{}

%This command puts different amounts of space depending on whether we are
% in the text, the workbook or the hints & solutions manual. 
\newcommand{\twsvspace}[3]{%
 \ifthenelse{\boolean{InTextBook} }{\vspace{#1}}{%
  \ifthenelse{\boolean{InWorkBook} }{\vspace{#2}}{%
   \ifthenelse{\boolean{InHints} }{\vspace{#3}}{} %
   }%
  }%
 }


\newcommand{\wbvfill}{\ifthenelse{\boolean{InWorkBook}}{\vfill}{}}
\newcommand{\wbitemsep}{\ifthenelse{\boolean{InWorkBook} }{\rule[-24pt]{0pt}{60pt}}{}}
\newcommand{\textbookpagebreak}{\ifthenelse{\boolean{InTextBook}}{\newpage}{}}
\newcommand{\workbookpagebreak}{\ifthenelse{\boolean{InWorkBook}}{\newpage}{}}
\newcommand{\hintspagebreak}{\ifthenelse{\boolean{InHints}}{\newpage}{}}

\newcommand{\hint}[1]{\ifthenelse{\boolean{InHints}}{ {\par \hspace{12pt} \color[rgb]{0,0,1} #1 } }{}}
\newcommand{\inlinehint}[1]{\ifthenelse{\boolean{InHints}}{ { \color[rgb]{0,0,1} #1 } }{}}

%\newlength{\cwidth}
%\newcommand{\cents}{\settowidth{\cwidth}{c}%
%\divide\cwidth by2
%\advance\cwidth by-.1pt
%c\kern-\cwidth
%\vrule width .1pt depth.2ex height1.2ex
%\kern 3\cwidth}
\newcommand{\cents}{\textcent\kern 5pt}

\newcommand{\sageprompt}{ {\tt sage$>$} }
\newcommand{\tab}{\rule{20pt}{0pt}}
\newcommand{\blnk}{\rule{1.5pt}{0pt}\rule{.4pt}{1.2pt}\rule{9pt}{.4pt}\rule{.4pt}{1.2pt}\rule{1.5pt}{0pt}}
\newcommand{\suchthat}{\; \rule[-3pt]{.5pt}{13pt} \;}
\newcommand{\divides}{\!\mid\!}
\newcommand{\tdiv}{\; \mbox{div} \;}
\newcommand{\restrict}[2]{#1 \,\rule[-4pt]{.25pt}{14pt}_{\,#2}}
\newcommand{\lcm}[2]{\mbox{lcm} (#1, #2)}
\renewcommand{\gcd}[2]{\mbox{gcd} (#1, #2)}
\newcommand{\Naturals}{{\mathbb N}}
\newcommand{\Integers}{{\mathbb Z}}
\newcommand{\Znoneg}{{\mathbb Z}^{\mbox{\tiny noneg}}}
\ifthenelse{\boolean{ZeroInNaturals}}{%
  \newcommand{\Zplus}{{\mathbb Z}^+} }{%
  \newcommand{\Zplus}{{\mathbb N}} }
\newcommand{\Enoneg}{{\mathbb E}^{\mbox{\tiny noneg}}}
\newcommand{\Qnoneg}{{\mathbb Q}^{\mbox{\tiny noneg}}}
\newcommand{\Rnoneg}{{\mathbb R}^{\mbox{\tiny noneg}}}
\newcommand{\Rationals}{{\mathbb Q}}
\newcommand{\Reals}{{\mathbb R}}
\newcommand{\Complexes}{{\mathbb C}}
%\newcommand{\F2}{{\mathbb F}_{2}}
\newcommand{\relQ}{\mbox{\textsf Q}}
\newcommand{\relR}{\mbox{\textsf R}}
\newcommand{\nrelR}{\mbox{\raisebox{1pt}{$\not$}\rule{1pt}{0pt}{\textsf R}}}
\newcommand{\relS}{\mbox{\textsf S}}
\newcommand{\relA}{\mbox{\textsf A}}
\newcommand{\Dom}[1]{\mbox{Dom}(#1)}
\newcommand{\Cod}[1]{\mbox{Cod}(#1)}
\newcommand{\Rng}[1]{\mbox{Rng}(#1)}

\DeclareMathOperator\caret{\raisebox{1ex}{$\scriptstyle\wedge$}}

\newtheorem*{defi}{Definition}
\newtheorem*{exer}{Exercise}
\newtheorem{thm}{Theorem}[section]
\newtheorem*{thm*}{Theorem}
\newtheorem{lem}[thm]{Lemma}
\newtheorem*{lem*}{Lemma}
\newtheorem{cor}{Corollary}
\newtheorem{conj}{Conjecture}

\renewenvironment{proof}%
{\begin{quote} \emph{Proof:} }%
{\rule{0pt}{0pt} \newline \rule{0pt}{15pt} \hfill Q.E.D. \end{quote}}


\newcommand{\vs}{\rule{0pt}{12pt}}

%\def\mycommand{\setlength{\abovedisplayskip}{-12pt}%
%\setlength{\belowdisplayskip}{-12pt}%
%\setlength{\abovedisplayshortskip}{0pt}%
%\setlength{\belowdisplayshortskip}{0pt}}

%\let\oldselectfont\selectfont
%\def\selectfont{\oldselectfont\mycommand}

%\mycommand

\AtBeginSection[]
{
 \begin{frame}{Table of Contents} 
  \tableofcontents[currentsection]
 \end{frame}
}

%%%% SAVE %%%%
%{ %magic to get a full screen image...
%\setbeamertemplate{navigation symbols}{}  % hide navigation buttons 
%\setbeamertemplate{background canvas}{\centerline{\includegraphics 
%	[height=\paperheight]{Cantor_4.jpeg}}}
%\begin{frame}[plain]
%\rule{0pt}{0pt}
%\end{frame} 
%} %end of magic


\begin{document}

\begin{frame}[plain]
  \titlepage
\end{frame}

\section{contradiction}

\begin{frame}{layout}
\begin{itemize}
\item  Assume the conclusion you wish to prove is actually false.\pause
\item From that premise deduce a complete absurdity (i.e.\ a contradiction) \pause
\item Since $\lnot P$ is equivalent to $c$ we deduce that $P \cong t$. \pause
\item Back in chapter 1 we saw a proof of this type -- the proof that $\sqrt{2}$ is not rational. \pause
\item Next we'll look at this theorem: \pause
\begin{thm} (Euclid) The set of all prime numbers is infinite.
\end{thm}
\end{itemize}
\end{frame}

\begin{frame}{infinitely many primes}

\noindent {\em Proof:}

\noindent Suppose on the contrary that there are only a finite number
of primes.  This finite set of prime numbers could, in principle, be listed
in ascending order.

\[  \{ p_1, p_2, p_3, \ldots , p_n \} \]

Consider the number $N$ formed by adding 1 to the product of all of these 
primes.

\[ N = 1 + \prod_{k=1}^n p_k \]

\end{frame}

\begin{frame}{infinitely many primes (cont.)}

Clearly, $N$ is much larger than the largest prime $p_n$, so $N$ cannot
be a prime number itself.  Thus $N$ must be a product of some of the 
primes in the list.  Suppose that $p_j$ is one of the primes that 
divides $N$.  Now notice that, by construction, $N$ would leave remainder
$1$ upon division by $p_j$.  This is a contradiction since we cannot have
both $p_j \divides N$ and $p_j \nmid N$. 

Since the supposition that there are only finitely many primes leads to
a contradiction, there must indeed be an infinite number of primes.

\hspace{\fill} Q.E.D.

\end{frame}

\begin{frame}{a short digression}
\begin{itemize}
\item The notation $p\#$ is used for the product of all the primes up to and including $p$. \pause
\item This is pronounced ``$p$ prime-orial.'' \pause
\item The number $N$ in our proof could have been expressed as $p_n\# + 1$. \pause
\item An incorrect (but somehow attractive) way to end the proof might be ``$N$ must be a prime since it isn't divisible by any of the known primes, so we've discovered a prime larger than the largest prime.  Which is a contradiction.'' \pause
\item This isn't quite right because while $N$ implies the existence of primes bigger than $p_n$, it's not necessary for $N$ itself to be prime. \pause
\item But is it? \pause Let's use CoCalc.
\end{itemize}
\end{frame}

\begin{frame}{another example}
\begin{itemize}
\item  There is no largest integer. \pause

\noindent {\em Proof:}

Suppose (by way of contradiction) that there is a largest integer $L$.   
Then $L \in \Integers$ and $\forall z \in \Integers, L \geq z$.
Consider the quantity $L+1$.  Clearly $L+1$ is an integer (because it is the sum of two integers) and also
$L+1 > L$.   This is a contradiction so the original supposition is false.   Hence there is no largest integer.

\hspace{\fill} Q.E.D.

\end{itemize}
\end{frame}

\begin{frame}{yet another example}
\begin{itemize}
\item  Prove that the sum of a rational and an irrational 
number is irrational.\pause

\noindent {\em Proof:}

\noindent Suppose (by way of contradiction) that $x$ is a rational number, that $y$ is an irrational number, and that $x+y$ is rational. 
Note that $y = (x+y) - x$ is the difference of two rational numbers.  By a previous result, the difference of rational numbers is rational.  This is a contradiction since it is impossible for $y$ to be both rational and irrational, so the original supposition is false.   Hence the sum of a rational and an irrational 
number must be irrational.

\hspace{\fill} Q.E.D.


\end{itemize}
\end{frame}

\section{contrapositives}

\begin{frame}{layout}
\begin{itemize}
\item  You'll start with an opening sentence that says something like:\pause \newline

``We will instead consider the equivalent statement (contrapositive of whatever we really want to prove).'' \pause
\item Then we just do a regular direct proof of that contrapositive statement.
\end{itemize}
\end{frame}

\begin{frame}{a long anticipated lemma proof}
\begin{itemize}
\item  Recall the lemma we wanted for the proof that $\sqrt{2} \notin \Rationals$ ? \pause
\item $ \displaystyle \forall x \in \Integers, \; x^2 \; \mbox{is even} \; \implies \; x \;  \mbox{is even.} $ \pause \newline
\noindent {\em Proof:} \newline
\noindent We will instead consider the equivalent statement,
\[ \forall x \in \Integers, \; x \; \mbox{is odd} \; \implies \; x^2 \;  \mbox{is odd.}  \]
\noindent (This is the contrapositive of the desired result.) \newline
\end{itemize}
\end{frame}

\begin{frame}{lemma (cont.)}

Suppose that $x$ is a particular, but arbitrary odd number.  By the definition of `odd' we know that there is an integer $k$ such that $x = 2k+1$.  Note that $x^2 = (2k+1)^2 = 4k^2+4k+1$, by substitution and simple algebra.  Finally, we see that $x^2 = 2(2k^2+2k) + 1$ satisfies the definition of `odd' because $2k^2 + 2k$ is the sum of products of integers, which is an integer.

\hspace{\fill} Q.E.D.
\end{frame}

\section{some LaTeX hints}

\begin{frame}{environments}
\begin{itemize}
\item Environments are used to typeset things of the same type in a reusable way. \pause
\item Let's open up Overleaf and check out the theorem and proof environments.
\end{itemize}
\end{frame}



\end{document}
