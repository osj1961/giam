\begin{enumerate}

\item Design a digital logic circuit (using and, or \& not gates) that 
implements an exclusive or.

\wbvfill

\hint{First, it's essential to know what is meant by the term "exclusive or". This is the interpretation that many people give to the word "or" -- where "X or Y" means either X is true or Y is true, but that it isn't the case that both X and Y are true. This (wrong) understanding of what "or" means is common because it is often the case that X and Y represent complimentary possibilities: old or new, cold or hot, right or wrong... The truth table for exclusive or (often written xor, pronounced "ex-or", symbolically it is usually $\oplus$) is

\begin{tabular}{|c|c|c|} \hline
\rule[-8pt]{0pt}{30pt}$X$ & $Y$ & $X \,\oplus\, Y$ \\ \hline
\rule[-8pt]{0pt}{30pt}$T$ & $T$ & $\phi$ \\ \hline
\rule[-8pt]{0pt}{30pt}$T$ & $\phi$ & $T$ \\ \hline
\rule[-8pt]{0pt}{30pt}$\phi$ & $T$ & $T$ \\ \hline
\rule[-8pt]{0pt}{30pt}$\phi$ & $\phi$ & $\phi$  \\ \hline
\end{tabular}

\noindent So it's true when one, or the other, but not both of its inputs are true.  The upshot of the last sentence is that we can write $X \oplus Y \; \equiv \; (X \lor Y) \land {\lnot}(X \land Y)$.

The above reformulation should help\ldots 

\vfill

}

\workbookpagebreak

\item Consider the sentence 
``This is a sentence which does not refer to itself.''
which was given in the beginning of this chapter as an example.
Is this sentence a statement?  If so, what is its truth value?

\hint{The only question in your mind, when deciding whether a sentence is a statement, should be "Does this thing have a definite truth value?"
Well?

Isn't it just plainly false?}

%\vspace{.5in}
\vfill

\item Consider the sentence ``This sentence is false.''  Is this 
sentence a statement?

\hint{Try to justify why this sentence can't be either true or false.}

\hintspagebreak

%\vspace{.5in}
\vfill

\workbookpagebreak

\item Complete truth tables for each of the sentences 
$(A \land B) \lor C$ and
$A \land (B \lor C)$.  Does it seem that these sentences have
the same logical content?

\hint{

\vfill

A tiny hint here: since the sentences involve 3 variables you'll need truth tables with 8 rows. Here's a template.

\vfill

\begin{tabular}{|c|c|c|c|c|} \hline
\rule[-8pt]{0pt}{30pt}$A$ & $B$ & $C$ & $(A \land B) \lor C$ & $A \land (B \lor C)$ \\ \hline
\rule[-8pt]{0pt}{30pt}$T$ & $T$ & $T$ & \rule{100pt}{0pt} & \rule{100pt}{0pt} \\ \hline
\rule[-8pt]{0pt}{30pt}$T$ & $T$ & $\phi$  & & \\ \hline
\rule[-8pt]{0pt}{30pt}$T$ & $\phi$  & $T$ & & \\ \hline
\rule[-8pt]{0pt}{30pt}$T$ & $\phi$  & $\phi$  & & \\  \hline
\rule[-8pt]{0pt}{30pt}$\phi$  & $T$ & $T$ & & \\ \hline
\rule[-8pt]{0pt}{30pt}$\phi$  & $T$ & $\phi$  & & \\ \hline
\rule[-8pt]{0pt}{30pt}$\phi$  & $\phi$  & $T$ & & \\ \hline
\rule[-8pt]{0pt}{30pt}$\phi$  & $\phi$  & $\phi$  & & \\  \hline
\end{tabular}
}
\vfill

\hintspagebreak
\workbookpagebreak

\item \label{ex:nand_nor} There are two other logical connectives that are
used somewhat less commonly than $\lor$ and $\land$.
These are the \index{Scheffer stroke} Scheffer stroke and the 
\index{Peirce arrow}Peirce arrow
-- written $\vert$ and $\downarrow$, respectively ---  they are 
also known as \index{NAND} NAND and \index{NOR} NOR.

\noindent The truth tables for these connectives are:
\medskip

\begin{tabular}{c|c|c}
$A$ & $B$ & $A \,\vert\, B$ \\ \hline
$T$ & $T$ & $\phi$ \\
$T$ & $\phi$ & $T$ \\
$\phi$ & $T$ & $T$ \\
$\phi$ & $\phi$ & $T$ 
\end{tabular}
\hspace{.25 in} and \hspace{.25 in}
\begin{tabular}{c|c|c}
$A$ & $B$ & $A \downarrow B$ \\ \hline
$T$ & $T$ & $\phi$ \\
$T$ & $\phi$ & $\phi$ \\
$\phi$ & $T$ & $\phi$ \\
$\phi$ & $\phi$ & $T$ 
\end{tabular}
\medskip

Find an expression for $(A\, \land {\lnot}B) \lor C$
using only these new connectives (as well as negation and the
variable symbols themselves).


\hint{Sorry, I know this is probably the hardest problem in the chapter, but I'm (mostly) not going to help...
Just one hint to help you get started: NAND and NOR are the negations of AND and OR (respectively) so, for example, $(X \land Y) \; \equiv \; {\lnot}(A \,\vert\, B)$.}

\textbookpagebreak
\workbookpagebreak


\item \label{IKK} The famous logician \index{Smullyan, Raymond} Raymond Smullyan devised 
a family of logical puzzles around a fictitious place he called 
\index{Knights and Knaves} ``the Island of Knights and Knaves.''  The inhabitants of the island are either knaves, who always make false statements, or knights, who always make truthful statements.  

In the most famous knight/knave puzzle, you are in a room which has only two exits.  One leads to certain death and the other to freedom.  There are two 
individuals in the room, and you know that one of them is a knight and the other is a knave, but you don't know which.   Your challenge is to determine the door which leads to freedom by asking a single question.

\hint{Ask one of them what the other one would say to do.}

\end{enumerate}
