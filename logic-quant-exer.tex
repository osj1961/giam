\begin{enumerate}
\item There is a common variant of the existential quantifier,
$\exists !$, if you write $\exists ! \, x, \, P(x)$ you are asserting 
that there is a \index{unique existence}\emph{unique} element 
in the universe that makes $P(x)$ true.
Determine how to negate the sentence $\exists ! \, x, \, P(x)$.

\item The order in which quantifiers appear is important.  Let $L(x,y)$
be the open sentence ``$x$ is in love with $y$.''  Discuss the meanings of the
following quantified statements and find their negations.

\begin{enumerate}
\item $\forall x \, \exists y \; L(x,y)$.
\item $\exists x \, \forall y \; L(x, y)$.
\item $\forall x \, \forall y \; L(x, y)$.
\item $\exists x \, \exists y \; L(x, y)$.
\end{enumerate}

\item Determine a useful denial of: 

$\displaystyle \forall \epsilon>0 \, \exists 
\delta>0 \, \forall x \, (|x-c| < \delta) \implies (|f(x)-l| < \epsilon) $.

The denial above gives a criterion for saying $\lim_{x\rightarrow c}f(x) \neq l.$

\item A \index{Sophie Germain prime} \emph{Sophie Germain prime} is a prime number $p$
such that the corresponding odd number $2p+1$ is also a prime.  For example 11 is a 
Sophie Germain prime since $23 = 2\cdot 11 + 1$ is also prime.  Almost all Sophie Germain
primes are congruent to $5 \pmod{6}$, nevertheless, there are exceptions -- so the
statement ``There are Sophie Germain primes that are not 5 mod 6.'' is true.  Verify this.

\item  Alvin, Betty, and Charlie enter a cafeteria which offers three different
entrees, turkey sandwich, veggie burger, and pizza; four different
beverages, soda, water, coffee, and milk; and two types of desserts,
pie and pudding. Alvin takes a turkey sandwich, a soda, and a pie.
Betty takes a veggie burger, a soda, and a pie. Charlie takes a pizza
and a soda. Based on this information, determine whether the following
statements are true or false.

\begin{enumerate}
\item \label{negated}$\forall$ people $p$, $\exists$ dessert $d$ such that $ p$
took $d$.
\item \label{compare}$\exists$ person $p$ such that $\forall$ desserts
$d$, $p$ did not take $d$.
\item $\forall$ entrees $e$, $\exists$ person $p$ such that $ p$ took
$e$. 
\item \label{entree}$\exists$ entree $e$ such that  $\forall$ people
$p,\ p$ took $e$.
\item $\forall$ people $p$, $p$ took a dessert $\iff p$ did not take
a pizza.
\item Change one word of statement \ref{entree} so that it becomes true.
\item Write down the negation of \ref{negated} and compare it to statement
\ref{compare}. Hopefully you will see that they are the same! Does
this make you want to modify one or both of your answers to \ref{negated}
and \ref{compare}?
\end{enumerate}

\end{enumerate}
