\begin{enumerate}
\item The \index{lexicographic order}\emph{lexicographic order}, 
$<_{\mbox{lex}}$, is a relation on the
set of all words, where $x <_{\mbox{lex}} y$ means that $x$ would come before
$y$ in the dictionary.  Consider just the three letter words like ``iff'',
``fig'', ``the'', et cetera.  Come up with a usable definition for
$x_1x_2x_3  <_{\mbox{lex}} y_1y_2y_3$.

\wbvfill

\workbookpagebreak

\item What is the graph of ``$=$'' in $\Reals \times \Reals$?

\wbvfill

%\workbookpagebreak

\item The \index{inverse relation} \emph{inverse} of a relation $\relR$
is denoted $\relR^{-1}$.  It contains exactly the same ordered pairs
as $\relR$ but with the order switched.  (So technically, they aren't
\emph{exactly} the same ordered pairs \ldots)

\[ \relR^{-1} = \{ (b,a) \suchthat (a,b) \in \relR \} \]

\noindent Define a relation $\relS$ on $\Reals \times \Reals$ by
$\relS = \{ (x,y) \suchthat y = \sin x \}$.  What is $\relS^{-1}$?
Draw a single graph containing $\relS$ and $\relS^{-1}$.

\wbvfill

\rule{0pt}{0pt}

\wbvfill

\workbookpagebreak


\item The ``socks and shoes'' rule is a very silly little mnemonic
for remembering how to invert a composition.  If we think of undoing
the process of putting on our socks and shoes (that's socks first, then
shoes) we have to first remove our shoes, \emph{then} take off our socks.

The socks and shoes rule is valid for relations as well.

Prove that $(\relS \circ \relR)^{-1} = \relR^{-1} \circ \relS^{-1}$.

\wbvfill

\workbookpagebreak

\end{enumerate} 

%% Emacs customization
%% 
%% Local Variables: ***
%% TeX-master: "GIAM-hw.tex" ***
%% comment-column:0 ***
%% comment-start: "%% "  ***
%% comment-end:"***" ***
%% End: ***
