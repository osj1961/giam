\begin{enumerate}

\item The $n$-th triangular number, denoted $T(n)$, is given by the formula
$T(n) = (n^2 + n)/2$.  If we regard this formula as a function from $\Reals$ to
$\Reals$, it fails the horizontal line test and so it is not invertible.  Find a
suitable restriction so that T is invertible.

\wbvfill

\item The usual algebraic procedure for inverting $T(x) = (x^2+x)/2$ fails.  Use
your knowledge of the geometry of functions and their inverses to find
a formula for the inverse. (Hint: it may be instructive to first invert
the simpler formula $S(x) = x^2/2$ --- this will get you the right vertical
scaling factor.)

\wbvfill

\item What is $\pi_2(W(t))$?  

\wbvfill

\item Find a right inverse for $f(x) = |x|$.

\wbvfill

\workbookpagebreak

\item In three-dimensional space we have projection functions that go onto
the three coordinate axes ($\pi_1$, $\pi_2$ and $\pi_3$) and we also have
projections onto coordinate planes.  For example,
$\pi_{12}: \Reals \times \Reals \times \Reals \longrightarrow \Reals \times \Reals$, defined by

\[ \pi_{12}((x,y,z)) = (x,y) \]

\noindent is the projection onto the $x$--$y$ coordinate plane.

The triple of functions  $(\cos{t}, \sin{t}, t)$ is a parametric
expression for a helix.  Let 
$H = \{ (\cos{t}, \sin{t}, t) \suchthat t \in \Reals \}$ be the set of all
points on the helix.  What is the set $\pi_{12}(H)$ ?  What are the
sets $\pi_{13}(H)$ and $\pi_{23}(H)$?

\wbvfill

\workbookpagebreak

\item Consider the set $\{1, 2, 3, \ldots, 10\}$.  Express the characteristic
function of the subset $S = \{1, 2, 3 \}$ as a set of ordered pairs.

\wbvfill

%\workbookpagebreak

\item If $S$ and $T$ are subsets of a set $D$, what is the product of
their characteristic functions $1_S \cdot 1_T$ ?

\wbvfill

%\workbookpagebreak

\item Evaluate the sum

\[ \sum_{i=1}^{10} \frac{1}{i} \cdot [ i \; \mbox{is prime} ]. \]

\wbvfill

\workbookpagebreak
\end{enumerate}

%% Emacs customization
%% 
%% Local Variables: ***
%% TeX-master: "GIAM-hw.tex" ***
%% comment-column:0 ***
%% comment-start: "%% "  ***
%% comment-end:"***" ***
%% End: ***
