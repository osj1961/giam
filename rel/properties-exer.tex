\begin{enumerate}
\item Consider the relation $\relS$ defined by 
\[ \relS = \{ (x,y) \suchthat \; x \; \mbox{is smarter than} \, y \}. \]
\noindent Is $\relS$ symmetric or anti-symmetric?  Explain.

\wbvfill

\item Consider the relation $\relA$ defined by 
\[ \relA = \{ (x,y) \suchthat \; x \; \mbox{has the same astrological sign as} \, y \}. \]
\noindent Is $\relA$ symmetric or anti-symmetric?  Explain.

\wbvfill

\item Explain why both of the relations just described (in problems 1 and 2)
have the transitive property.

\wbvfill

\item For each of the five properties, name a relation that has it
and a relation that doesn't.

\wbvfill

\rule{0pt}{0pt}

\wbvfill

\workbookpagebreak

\item Show by counterexample that ``$\divides$'' (divisibility) is not symmetric as a relation on $\Integers$.

 \wbvfill
 
 \item Prove that ``$\divides$'' is an ordering relation (you must verify that it is reflexive, anti-symmetric and transitive).

 \wbvfill

\rule{0pt}{0pt}

\end{enumerate} 

%% Emacs customization
%% 
%% Local Variables: ***
%% TeX-master: "GIAM-hw.tex" ***
%% comment-column:0 ***
%% comment-start: "%% "  ***
%% comment-end:"***" ***
%% End: ***
