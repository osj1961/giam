\begin{enumerate}
\item Prove that if $n$ is an odd number then $n^4 \pmod{16} = 1$.

\item Prove that every prime number other than 2 and 3 has the form
$6q+1$ or $6q+5$ for some integer $q$.  (Hint: this problem involves
thinking about cases as well as contrapositives.)

\item Show that the sum of any three consecutive integers is divisible
by 3.

\item Find the pebbling number of a graph whose nodes are the corners and 
whose edges are the, uhmm, edges of a cube.

\item A \index{vampire number}\emph{vampire number} is a $2n$ digit number $v$ that factors as $v=xy$
where $x$ and $y$ are $n$ digit numbers and the digits of $v$ are the 
union of the digits in $x$ and $y$ in some order.  The numbers $x$ and $y$
are known as the ``fangs'' of $v$.  To eliminate trivial
cases, pairs of trailing zeros are disallowed.  

Show that there are no 2-digit vampire numbers.

Show that there are seven 4-digit vampire numbers.

\item Lagrange's theorem on representation of integers as sums of squares
says that every positive integer can be expressed as the sum of at most 
4 squares.  For example, $79 = 7^2 + 5^2 + 2^2 + 1^2$.  Show (exhaustively) 
that 15 can not be represented using fewer than 4 squares.

\item Show that there are exactly 15 numbers $x$ in the range $1 \leq x \leq 100$ that can't be represented using fewer than 4 squares.

\item The \index{trichotomy property}\emph{trichotomy property} of the real 
numbers simply states that every real number is either positive or negative 
or zero.  Trichotomy can be used to prove many statements by looking at the
three cases that it guarantees.  Develop a proof (by cases) that the square of
any real number is non-negative.

\item Consider the game called ``binary determinant tic-tac-toe''\footnote{ %
This question was problem A4 in the 63rd annual %
\index{William Lowell Putnam Mathematics Competition} %
William Lowell Putnam Mathematics Competition (2002).  %
There are three collections of questions %
and answers  from previous Putnam exams available from the MAA % 
\cite{putnam1,putnam2,putnam3}% 
}
which is played by two players who alternately fill in the entries of a 
$3 \times 3$ array.  Player One goes first, placing 1's in the array and 
player Zero goes second, placing 0's.  Player One's goal is that the 
final array have determinant 1, and player Zero's goal is that the 
determinant be 0.  The determinant calculations are carried out mod 2.

Show that player Zero can always win a game of binary determinant tic-tac-toe
by the method of exhaustion.

\end{enumerate}
