\begin{enumerate}
\item Determine the logical form of the following arguments.  Use symbols
to express that form and determine whether the form is valid or invalid.
If the form is invalid, determine the type of error made.  Comment on the 
soundness of the argument as well, in particular, determine whether any of
the premises are questionable.
\begin{enumerate}
\item All who are guilty are in prison. \newline
  George is not in prison.  \newline
  Therefore, George is not guilty.
\item If one eats oranges one will have high levels of vitamin C. \newline
  You do have high levels of vitamin C. \newline
  Therefore, you must eat oranges.
\item All fish live in water. \newline
  The mackerel is a fish. \newline
  Therefore, the mackerel lives in water. 
\item If you're lazy, don't take math courses.\newline
  Everyone is lazy. \newline
  Therefore, no one should take math courses.
\item All fish live in water. \newline
  The octopus lives in water. \newline
  Therefore, the octopus is a fish.
\item If a person goes into politics, they are a scoundrel.\newline
  Harold has gone into politics. \newline
  Therefore, Harold is a scoundrel. 
\end{enumerate}

\item Below is a rule of inference that we call extended elimination.

\begin{tabular}{cl}
 & $(A \lor B) \lor C$ \\
 & $\lnot A$ \\
 & $\lnot B$ \\ \hline
$\therefore$ & $C$ \\
\end{tabular}

Use a truth table to verify that this rule is valid.

\item If we allow quantifiers and open sentences in an argument form we
get arguments that are termed ``universal'' and ``particular.''

For example  \begin{tabular}{cl}
 & $\forall x, A(x) \implies B(x)$ \\
 & $A(p)$ \\ \hline
$\therefore$ & $B(p)$ \\
\end{tabular}  is the particular form of modus ponens (here, $p$
is not a variable -- it stands for some particular element of the universe of
discourse)
and \begin{tabular}{cl}
 & $\forall x, A(x) \implies B(x)$ \\
 & $\forall x, \lnot B(x)$ \\ \hline
$\therefore$ & $\forall x, \lnot A(x)$ \\
\end{tabular} is the universal form of modus tollens.

Reexamine the arguments from problem (1), determine their forms
(including quantifiers) and whether they are universal or particular.

\item Identify the rule of inference being used.

\begin{enumerate}
\item The Buley Library is very tall.\\
Therefore, either the Buley Library is very tall or it has many
levels underground.

\item The grass is green.\\
The sky is blue.\\
Therefore, the grass is green and the sky is blue.

\item $g$ has order 3 or it has order 4.\\
If $g$ has order 3, then $g$ has an inverse.\\
If $g$ has order 4, then $g$ has an inverse.\\
Therefore, $g$ has an inverse.

\item $x$ is greater than 5 and $x$ is less than 53.\\
Therefore, $x$ is less than 53.

\item If $a|b$, then $a$ is a perfect square.\\
If $a|b$, then $b$ is a perfect square.\\
Therefore, if $a|b$, then $a$ is a perfect square and $b$ is
a perfect square.

\end{enumerate}

\item Read the following proof that the sum of two odd numbers is even.
Discuss the rules of inference used.\\
\begin{proof}
Let $x$ and $y$ be odd numbers. Then $x=2k+1$
and $y=2j+1$ for some integers $j$ and $k$. By algebra,
\[
x+y = 2k+1 + 2j+1 = 2(k+j+1).
\]

Note that $k+j+1$ is an integer because $k$ and $j$ are integers.
Hence $x+y$ is even. 
\end{proof}
\end{enumerate}
