\documentclass{amsart}
%\usepackage{}
\renewcommand{\baselinestretch}{1.5}


\newlength{\cwidth}
\newcommand{\cents}{\settowidth{\cwidth}{c}%
\divide\cwidth by2
\advance\cwidth by-.1pt
c\kern-\cwidth
\vrule width .1pt depth.2ex height1.2ex
\kern\cwidth}

\newcommand{\sageprompt}{ {\tt sage$>$} }
\newcommand{\tab}{\rule{20pt}{0pt}}
\newcommand{\blnk}{\rule{1.5pt}{0pt}\rule{.4pt}{1.2pt}\rule{9pt}{.4pt}\rule{.4pt}{1.2pt}\rule{1.5pt}{0pt}}
\newcommand{\suchthat}{\; \rule[-3pt]{.25pt}{13pt} \;}
\newcommand{\divides}{\!\mid\!}
\newcommand{\tdiv}{\; \mbox{div} \;}
\newcommand{\restrict}[2]{#1 \,\rule[-4pt]{.125pt}{14pt}_{\,#2}}
\newcommand{\lcm}[2]{\mbox{lcm} (#1, #2)}
\renewcommand{\gcd}[2]{\mbox{gcd} (#1, #2)}
\newcommand{\Naturals}{{\mathbb N}}
\newcommand{\Integers}{{\mathbb Z}}
\newcommand{\Znoneg}{{\mathbb Z}^{\mbox{\tiny noneg}}}
\newcommand{\Enoneg}{{\mathbb E}^{\mbox{\tiny noneg}}}
\newcommand{\Qnoneg}{{\mathbb Q}^{\mbox{\tiny noneg}}}
\newcommand{\Rnoneg}{{\mathbb R}^{\mbox{\tiny noneg}}}
\newcommand{\Rationals}{{\mathbb Q}}
\newcommand{\Reals}{{\mathbb R}}
\newcommand{\Complexes}{{\mathbb C}}
%\newcommand{\F2}{{\mathbb F}_{2}}
\newcommand{\relQ}{\mbox{\textsf Q}}
\newcommand{\relR}{\mbox{\textsf R}}
\newcommand{\nrelR}{\mbox{\raisebox{1pt}{$\not$}\rule{1pt}{0pt}{\textsf R}}}
\newcommand{\relS}{\mbox{\textsf S}}
\newcommand{\relA}{\mbox{\textsf A}}
\newcommand{\Dom}[1]{\mbox{Dom}(#1)}
\newcommand{\Cod}[1]{\mbox{Cod}(#1)}
\newcommand{\Rng}[1]{\mbox{Rng}(#1)}

\DeclareMathOperator\caret{\raisebox{1ex}{$\scriptstyle\wedge$}}

\newtheorem*{defi}{Definition}
\newtheorem*{exer}{Exercise}
\newtheorem{thm}{Theorem}[section]
\newtheorem*{thm*}{Theorem}
\newtheorem{lem}[thm]{Lemma}
\newtheorem{cor}{Corollary}
\newtheorem{conj}{Conjecture}

\renewenvironment{proof}%
{\begin{quote} \emph{Proof:} }%
{\rule{0pt}{0pt} \newline \rule{0pt}{15pt} \hfill Q.E.D. \end{quote}}


\begin{document}
\thispagestyle{empty}

\centerline{\Large Activity 3 -- Introduction to Proof}
\centerline{\large Universal and Existential quantifiers}

\bigskip
\Large


\begin{enumerate}
\item Our definitions of even and odd can be re-expressed using the existential quantifier.  (The central question of whether a number is even is the existance of an integer half as big.) Do so.

\vfill

\item People use the term {\em denial} to refer to a statement that shows another statement is false.  What is the denial of ``All animals living in the ocean are fish''  ?

\vfill

\item Regarding the `$x$ loves $y$' sentence, the ordering of variables and quantifiers that we didn't see in the lecture is:
 
\rule{0pt}{24pt} $\displaystyle \forall y \in P, \; \exists x \in P, \; x \, \mbox{loves} \, y$.

Interpret this statement.

\vfill

\newpage

\item Despite all the warnings about not switching the order of quantifiers, if two quantifiers are of the same type it's okay to interchange them.  Try replacing the quantifiers in the $x$ loves $y$ sentences with two universal quantifiers and examine the meaning (and whether it changes based on order).  Try the same thing with two existential quantifiers.

\vfill

\item Because of the feature we noticed in the last problem, it is common to use a single quantifier for groups of variables that have the same quantification.  Re-express the following using separate universal quantifiers (in both possible orders).

\[ \forall x, y \in \Rationals^\ast, \; \exists z \in \Rationals^\ast, z =x/y. \]

\vfill

\item Look up the definition of the limit of a function $f(x)$ as $x$ approaches $a$ (it's googleable) and write it down using quantifiers.

\vfill

\end{enumerate}

\end{document}
