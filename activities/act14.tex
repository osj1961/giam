\documentclass{amsart}
\usepackage{amssymb}
\renewcommand{\baselinestretch}{1.5}
\addtolength{\textwidth}{.2in}
\addtolength{\topmargin}{-.5in}
\addtolength{\textheight}{1in}
\newcommand{\versionNum}{$3.2$\ }

\newboolean{InTextBook}
\setboolean{InTextBook}{false}
\newboolean{InWorkBook}
\setboolean{InWorkBook}{false}
\newboolean{InHints}
\setboolean{InHints}{false}

%When this boolean is true (beginning in Section 5.1) we will use the convention
% that $0 \in \Naturals$.  If it is false we will continue to count $1$ as the smallest
%natural number (thus making Giuseppe Peano spin in his grave...)
 
\newboolean{ZeroInNaturals}

%This boolean is used to distinguish the version where we use $\sim$ rather than $\lnot$

\newboolean{LNotIsSim}

%The values of the last two booleans are set in ``switches.tex''

%\input{switches}

\let\savedlnot\lnot
\ifthenelse{\boolean{LNotIsSim}}{\renewcommand{\lnot}{\sim} }{}

%This command puts different amounts of space depending on whether we are
% in the text, the workbook or the hints & solutions manual. 
\newcommand{\twsvspace}[3]{%
 \ifthenelse{\boolean{InTextBook} }{\vspace{#1}}{%
  \ifthenelse{\boolean{InWorkBook} }{\vspace{#2}}{%
   \ifthenelse{\boolean{InHints} }{\vspace{#3}}{} %
   }%
  }%
 }


\newcommand{\wbvfill}{\ifthenelse{\boolean{InWorkBook}}{\vfill}{}}
\newcommand{\wbitemsep}{\ifthenelse{\boolean{InWorkBook} }{\rule[-24pt]{0pt}{60pt}}{}}
\newcommand{\textbookpagebreak}{\ifthenelse{\boolean{InTextBook}}{\newpage}{}}
\newcommand{\workbookpagebreak}{\ifthenelse{\boolean{InWorkBook}}{\newpage}{}}
\newcommand{\hintspagebreak}{\ifthenelse{\boolean{InHints}}{\newpage}{}}

\newcommand{\hint}[1]{\ifthenelse{\boolean{InHints}}{ {\par \hspace{12pt} \color[rgb]{0,0,1} #1 } }{}}
\newcommand{\inlinehint}[1]{\ifthenelse{\boolean{InHints}}{ { \color[rgb]{0,0,1} #1 } }{}}

%\newlength{\cwidth}
%\newcommand{\cents}{\settowidth{\cwidth}{c}%
%\divide\cwidth by2
%\advance\cwidth by-.1pt
%c\kern-\cwidth
%\vrule width .1pt depth.2ex height1.2ex
%\kern 3\cwidth}
\newcommand{\cents}{\textcent\kern 5pt}

\newcommand{\sageprompt}{ {\tt sage$>$} }
\newcommand{\tab}{\rule{20pt}{0pt}}
\newcommand{\blnk}{\rule{1.5pt}{0pt}\rule{.4pt}{1.2pt}\rule{9pt}{.4pt}\rule{.4pt}{1.2pt}\rule{1.5pt}{0pt}}
\newcommand{\suchthat}{\; \rule[-3pt]{.5pt}{13pt} \;}
\newcommand{\divides}{\!\mid\!}
\newcommand{\tdiv}{\; \mbox{div} \;}
\newcommand{\restrict}[2]{#1 \,\rule[-4pt]{.25pt}{14pt}_{\,#2}}
\newcommand{\lcm}[2]{\mbox{lcm} (#1, #2)}
\renewcommand{\gcd}[2]{\mbox{gcd} (#1, #2)}
\newcommand{\Naturals}{{\mathbb N}}
\newcommand{\Integers}{{\mathbb Z}}
\newcommand{\Znoneg}{{\mathbb Z}^{\mbox{\tiny noneg}}}
\ifthenelse{\boolean{ZeroInNaturals}}{%
  \newcommand{\Zplus}{{\mathbb Z}^+} }{%
  \newcommand{\Zplus}{{\mathbb N}} }
\newcommand{\Enoneg}{{\mathbb E}^{\mbox{\tiny noneg}}}
\newcommand{\Qnoneg}{{\mathbb Q}^{\mbox{\tiny noneg}}}
\newcommand{\Rnoneg}{{\mathbb R}^{\mbox{\tiny noneg}}}
\newcommand{\Rationals}{{\mathbb Q}}
\newcommand{\Reals}{{\mathbb R}}
\newcommand{\Complexes}{{\mathbb C}}
%\newcommand{\F2}{{\mathbb F}_{2}}
\newcommand{\relQ}{\mbox{\textsf Q}}
\newcommand{\relR}{\mbox{\textsf R}}
\newcommand{\nrelR}{\mbox{\raisebox{1pt}{$\not$}\rule{1pt}{0pt}{\textsf R}}}
\newcommand{\relS}{\mbox{\textsf S}}
\newcommand{\relA}{\mbox{\textsf A}}
\newcommand{\Dom}[1]{\mbox{Dom}(#1)}
\newcommand{\Cod}[1]{\mbox{Cod}(#1)}
\newcommand{\Rng}[1]{\mbox{Rng}(#1)}

\DeclareMathOperator\caret{\raisebox{1ex}{$\scriptstyle\wedge$}}

\newtheorem*{defi}{Definition}
\newtheorem*{exer}{Exercise}
\newtheorem{thm}{Theorem}[section]
\newtheorem*{thm*}{Theorem}
\newtheorem{lem}[thm]{Lemma}
\newtheorem*{lem*}{Lemma}
\newtheorem{cor}{Corollary}
\newtheorem{conj}{Conjecture}

\renewenvironment{proof}%
{\begin{quote} \emph{Proof:} }%
{\rule{0pt}{0pt} \newline \rule{0pt}{15pt} \hfill Q.E.D. \end{quote}}


\begin{document}
\thispagestyle{empty}

\centerline{\Large Activity 14 -- Introduction to Proof}
\centerline{\large direct proofs}

\bigskip
\Large


\begin{enumerate}

\item Use the table method to multiply $x^3 + 3x^2 + 3x + 1$ and $x^2 + 2x + 1$

\vfill

\item Use a difference table to help you deduce a formula for the sequence

\[ 1 \qquad 3 \qquad 7 \qquad 13 \qquad 21 \qquad 31 \qquad \ldots \]


\vfill

\newpage

\item The square of an odd number is always odd.

Re-express this more formally (as a universal conditional sentence).

\vfill

\item From the previous problem we should know that the initial line of a proof of that statement will be

\begin{quote}
Suppose that $x$ is a particular, but arbitrarily chosen odd integer. 
\end{quote}

What is the second line?

\vfill

\item Write a complete proof of 

\[ \forall x \in \Integers, \; x \; \mbox{is odd} \; \implies x^2 \; \mbox{is odd.} \]


\vfill

\vfill

\newpage

\item The informal statement ``the sum of an odd and an even is odd'' can be expressed with mathematical formalism as

\[ \forall x,y \in \Integers, \; ( x \; \mbox{is odd} \; \land \; y \; \mbox{is even}) \quad \implies \quad x+y \; \mbox{is odd.} \]

What is the first line of a proof of this statement?

\vfill

\item Write a complete proof of the statement in the previous problem.

\vfill

\vfill

\newpage

\item There are quite a few ``small words'' and phrases that are used in connecting the statements in a formal proof.

Add as many words or phrases as you can to the following lists.

\begin{enumerate}
\item \rule{0pt}{30pt} When adding a new deduction:

Therefore \rule{36pt}{0pt} Then \rule{36pt}{0pt} It follows that 


\item \rule{0pt}{30pt} When introducing something new:

Consider \rule{36pt}{0pt} Observe that \rule{36pt}{0pt} 

\item \rule{0pt}{30pt} Introducing a ``we want to show'' sentence:

We want to show that \rule{36pt}{0pt} The desired conclusion is that 

\end{enumerate}

\vspace{.5in}

\item Did your proof that the sum of an odd and an even is odd use connecting language?  If not, write it a second time here and include some ``Thus''s and ``Therefore''s (and maybe a ``since'' or two).

\vfill

\newpage

\item Recall that complex numbers are expressions of the form $a + bi$, where $a$ and $b$ are real numbers and $i^2 = -1$.  A proof that the complex numbers are closed under multiplication would have a fairly weak hypothesis:  That two ``particular but arbitrary'' complex numbers are given.  In writing up such a proof you would need the closure axioms for $\Reals$ (in particular that sums, products and differences of real numbers are real numbers) and some basic rules of algebra (e.g.\ the FOIL rule).

I'll give you the first two statements and you finish.

\vspace{.5in}

{\em Proof:} Suppose that $x$ and $y$ are particular, but arbitrarily chosen complex numbers.  Since $x$ and $y$ are complex numbers it follows that there are real numbers $a$, $b$, $c$ and $d$ such that

\[ x \; = \; a+bi \quad \mbox{and} \quad y \; = \; c+di \qquad ( \mbox{where} \; i \; \mbox{satisfies} \; i^2 = -1). \]



\vfill

\end{enumerate}

\end{document}
