\documentclass{amsart}
%\usepackage{}
\renewcommand{\baselinestretch}{1.5}


\newlength{\cwidth}
\newcommand{\cents}{\settowidth{\cwidth}{c}%
\divide\cwidth by2
\advance\cwidth by-.1pt
c\kern-\cwidth
\vrule width .1pt depth.2ex height1.2ex
\kern\cwidth}

\newcommand{\sageprompt}{ {\tt sage$>$} }
\newcommand{\tab}{\rule{20pt}{0pt}}
\newcommand{\blnk}{\rule{1.5pt}{0pt}\rule{.4pt}{1.2pt}\rule{9pt}{.4pt}\rule{.4pt}{1.2pt}\rule{1.5pt}{0pt}}
\newcommand{\suchthat}{\; \rule[-3pt]{.25pt}{13pt} \;}
\newcommand{\divides}{\!\mid\!}
\newcommand{\tdiv}{\; \mbox{div} \;}
\newcommand{\restrict}[2]{#1 \,\rule[-4pt]{.125pt}{14pt}_{\,#2}}
\newcommand{\lcm}[2]{\mbox{lcm} (#1, #2)}
\renewcommand{\gcd}[2]{\mbox{gcd} (#1, #2)}
\newcommand{\Naturals}{{\mathbb N}}
\newcommand{\Integers}{{\mathbb Z}}
\newcommand{\Znoneg}{{\mathbb Z}^{\mbox{\tiny noneg}}}
\newcommand{\Enoneg}{{\mathbb E}^{\mbox{\tiny noneg}}}
\newcommand{\Qnoneg}{{\mathbb Q}^{\mbox{\tiny noneg}}}
\newcommand{\Rnoneg}{{\mathbb R}^{\mbox{\tiny noneg}}}
\newcommand{\Rationals}{{\mathbb Q}}
\newcommand{\Reals}{{\mathbb R}}
\newcommand{\Complexes}{{\mathbb C}}
%\newcommand{\F2}{{\mathbb F}_{2}}
\newcommand{\relQ}{\mbox{\textsf Q}}
\newcommand{\relR}{\mbox{\textsf R}}
\newcommand{\nrelR}{\mbox{\raisebox{1pt}{$\not$}\rule{1pt}{0pt}{\textsf R}}}
\newcommand{\relS}{\mbox{\textsf S}}
\newcommand{\relA}{\mbox{\textsf A}}
\newcommand{\Dom}[1]{\mbox{Dom}(#1)}
\newcommand{\Cod}[1]{\mbox{Cod}(#1)}
\newcommand{\Rng}[1]{\mbox{Rng}(#1)}

\DeclareMathOperator\caret{\raisebox{1ex}{$\scriptstyle\wedge$}}

\newtheorem*{defi}{Definition}
\newtheorem*{exer}{Exercise}
\newtheorem{thm}{Theorem}[section]
\newtheorem*{thm*}{Theorem}
\newtheorem{lem}[thm]{Lemma}
\newtheorem{cor}{Corollary}
\newtheorem{conj}{Conjecture}

\renewenvironment{proof}%
{\begin{quote} \emph{Proof:} }%
{\rule{0pt}{0pt} \newline \rule{0pt}{15pt} \hfill Q.E.D. \end{quote}}


\begin{document}
\thispagestyle{empty}

\centerline{\Large Activity 4 -- Introduction to Proof}
\centerline{\large Definitions of Elementary Number Theory}

\bigskip
\Large


\begin{enumerate}
\item What integer must exist in order to show that 42 is even?
\vfill

\item What integer must exist to show that 17 is odd?
\vfill

\item Try writing a formal argument that the sum of an even and an odd integer must be odd.
\vfill

\item If we want to write numbers in base-12, we'll need have single digits for all of the numbers from 0 to 11.  We can certainly reuse 0 through 9, let's use $\delta$ and $\epsilon$ for 10 and 11.  In the little 12 toes schoolhouse rock episode these are pronounced ``dec'' and ``ell'' respectively.
\begin{enumerate}
\item What base-12 number comes immediately after $\epsilon \epsilon \epsilon_{12}$ ?
\vfill

\item What base-12 number comes immediately before $\epsilon \epsilon \epsilon_{12}$ ?
\vfill

\item Suppose I tell you that the decimal number 99 translates to $83_{12}$ in base-12.  What is 100 in base-12?  In contrast, what decimal corresponds to $100_{12}$?
\vfill

\end{enumerate}

\newpage

\item In using the definition of place value to reinterpret a base-12 number in decimal, we have to use the base-10 equivalents for the digits.
\[ \delta_{12} \; = \; 10_{10} \quad \mbox{and} \quad \epsilon_{12} \; = \; 11_{10} \]
So what is the decimal value of $89\delta\epsilon_{12}$ ?

\vfill

\item Recall that in hexadecimal, the digits are in 
\[ \{0,1,2,3,4,5,6,7,8,9,A,B,C,D,E,F\} \]
\noindent which correspond to the decimal numbers 1 through 15.  In hexadecimal, $F$ ``acts like 9.''  Use this to say what the decimal value of $FFFF_{16}$ is.

\vfill 

\item Circle the divisibility statements that are true:

\centerline{
\begin{tabular}{cccc}
 & \rule{72pt}{0pt} & \rule{72pt}{0pt} & \rule{72pt}{0pt} \\
\rule[-15pt]{0pt}{44pt} & $5\divides 100$ &  $7 \divides 43 $ & $11 \divides 77$ \\
\rule[-15pt]{0pt}{44pt} & $1\divides 2$ & $21 \divides 0$ & $3 \divides 322$ \\
\rule[-15pt]{0pt}{44pt} & $8\divides 64$ & $22 \divides 11$ & $3 \divides 321$ \\
\end{tabular}
}

\vspace{.4in}

\newpage

\item Is it true that $0$ divides all integers?  Is $0\divides 0$ true?

\vfill

\item Suppose you are taking down Christmas decorations and you have 79 ornaments that need to be stored in boxes that hold 12 ornaments each.  Use either a floor or a ceiling computation to determine the number of boxes that will be needed.

\vfill 

\item What is the value of $\lfloor \lceil \pi \rceil \rfloor$ ?  Can you simplify  $\lfloor \lceil x \rceil \rfloor$ in general?

\vfill


\item Calculate the following:

\begin{enumerate}
\item \rule[-12pt]{0pt}{36pt} $17 \mod 5$ 
\item \rule[-12pt]{0pt}{36pt} $77 \mod 5$ 
\item \rule[-12pt]{0pt}{36pt} $25 \mod 12$ 
\item \rule[-12pt]{0pt}{36pt} $99 \mod 7$ 
\item \rule[-12pt]{0pt}{36pt} $10003 \mod 1000$ 
\end{enumerate}

\vspace{.2 in}

\newpage

\item Recall that the notation $n \mod d$ is used for the remainder obtained when we divide $n$ by $d$.  What is $(n \mod d) \mod d$ ?

\vfill

\item Complete the table:

\centerline{
\begin{tabular}{c|c}
\rule[-6pt]{0pt}{24pt}$n$ & \rule{12pt}{0pt} $n!$ \rule{12pt}{0pt} \\ \hline
\rule[-6pt]{0pt}{24pt} 0 & \\
\rule[-6pt]{0pt}{24pt} 1 & \\
\rule[-6pt]{0pt}{24pt} 2 & \\
\rule[-6pt]{0pt}{24pt} 3 & \\
\rule[-6pt]{0pt}{24pt} 4 & \\
\rule[-6pt]{0pt}{24pt} 5 & \\
\rule[-6pt]{0pt}{24pt} 6 & \\
\rule[-6pt]{0pt}{24pt} 7 & \\
\end{tabular}
}

\vspace{.2in}

\item Which binomial coefficient would be obtained by simplifying the following fraction?

\[ \frac{10!}{4! \cdot 6!} \]

Is there another correct answer?

\vfill


\end{enumerate}

\end{document}
