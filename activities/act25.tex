\documentclass{amsart}
\usepackage{amssymb}
\renewcommand{\baselinestretch}{1.5}
\addtolength{\textwidth}{.2in}
\addtolength{\topmargin}{-.5in}
\addtolength{\textheight}{1in}

\usepackage{ifthen}
\usepackage{graphicx}
\usepackage{xcolor}

\newcommand{\versionNum}{$3.2$\ }

\newboolean{InTextBook}
\setboolean{InTextBook}{false}
\newboolean{InWorkBook}
\setboolean{InWorkBook}{false}
\newboolean{InHints}
\setboolean{InHints}{false}

%When this boolean is true (beginning in Section 5.1) we will use the convention
% that $0 \in \Naturals$.  If it is false we will continue to count $1$ as the smallest
%natural number (thus making Giuseppe Peano spin in his grave...)
 
\newboolean{ZeroInNaturals}

%This boolean is used to distinguish the version where we use $\sim$ rather than $\lnot$

\newboolean{LNotIsSim}

%The values of the last two booleans are set in ``switches.tex''

%\input{switches}

\let\savedlnot\lnot
\ifthenelse{\boolean{LNotIsSim}}{\renewcommand{\lnot}{\sim} }{}

%This command puts different amounts of space depending on whether we are
% in the text, the workbook or the hints & solutions manual. 
\newcommand{\twsvspace}[3]{%
 \ifthenelse{\boolean{InTextBook} }{\vspace{#1}}{%
  \ifthenelse{\boolean{InWorkBook} }{\vspace{#2}}{%
   \ifthenelse{\boolean{InHints} }{\vspace{#3}}{} %
   }%
  }%
 }


\newcommand{\wbvfill}{\ifthenelse{\boolean{InWorkBook}}{\vfill}{}}
\newcommand{\wbitemsep}{\ifthenelse{\boolean{InWorkBook} }{\rule[-24pt]{0pt}{60pt}}{}}
\newcommand{\textbookpagebreak}{\ifthenelse{\boolean{InTextBook}}{\newpage}{}}
\newcommand{\workbookpagebreak}{\ifthenelse{\boolean{InWorkBook}}{\newpage}{}}
\newcommand{\hintspagebreak}{\ifthenelse{\boolean{InHints}}{\newpage}{}}

\newcommand{\hint}[1]{\ifthenelse{\boolean{InHints}}{ {\par \hspace{12pt} \color[rgb]{0,0,1} #1 } }{}}
\newcommand{\inlinehint}[1]{\ifthenelse{\boolean{InHints}}{ { \color[rgb]{0,0,1} #1 } }{}}

%\newlength{\cwidth}
%\newcommand{\cents}{\settowidth{\cwidth}{c}%
%\divide\cwidth by2
%\advance\cwidth by-.1pt
%c\kern-\cwidth
%\vrule width .1pt depth.2ex height1.2ex
%\kern 3\cwidth}
\newcommand{\cents}{\textcent\kern 5pt}

\newcommand{\sageprompt}{ {\tt sage$>$} }
\newcommand{\tab}{\rule{20pt}{0pt}}
\newcommand{\blnk}{\rule{1.5pt}{0pt}\rule{.4pt}{1.2pt}\rule{9pt}{.4pt}\rule{.4pt}{1.2pt}\rule{1.5pt}{0pt}}
\newcommand{\suchthat}{\; \rule[-3pt]{.5pt}{13pt} \;}
\newcommand{\divides}{\!\mid\!}
\newcommand{\tdiv}{\; \mbox{div} \;}
\newcommand{\restrict}[2]{#1 \,\rule[-4pt]{.25pt}{14pt}_{\,#2}}
\newcommand{\lcm}[2]{\mbox{lcm} (#1, #2)}
\renewcommand{\gcd}[2]{\mbox{gcd} (#1, #2)}
\newcommand{\Naturals}{{\mathbb N}}
\newcommand{\Integers}{{\mathbb Z}}
\newcommand{\Znoneg}{{\mathbb Z}^{\mbox{\tiny noneg}}}
\ifthenelse{\boolean{ZeroInNaturals}}{%
  \newcommand{\Zplus}{{\mathbb Z}^+} }{%
  \newcommand{\Zplus}{{\mathbb N}} }
\newcommand{\Enoneg}{{\mathbb E}^{\mbox{\tiny noneg}}}
\newcommand{\Qnoneg}{{\mathbb Q}^{\mbox{\tiny noneg}}}
\newcommand{\Rnoneg}{{\mathbb R}^{\mbox{\tiny noneg}}}
\newcommand{\Rationals}{{\mathbb Q}}
\newcommand{\Reals}{{\mathbb R}}
\newcommand{\Complexes}{{\mathbb C}}
%\newcommand{\F2}{{\mathbb F}_{2}}
\newcommand{\relQ}{\mbox{\textsf Q}}
\newcommand{\relR}{\mbox{\textsf R}}
\newcommand{\nrelR}{\mbox{\raisebox{1pt}{$\not$}\rule{1pt}{0pt}{\textsf R}}}
\newcommand{\relS}{\mbox{\textsf S}}
\newcommand{\relA}{\mbox{\textsf A}}
\newcommand{\Dom}[1]{\mbox{Dom}(#1)}
\newcommand{\Cod}[1]{\mbox{Cod}(#1)}
\newcommand{\Rng}[1]{\mbox{Rng}(#1)}

\DeclareMathOperator\caret{\raisebox{1ex}{$\scriptstyle\wedge$}}

\newtheorem*{defi}{Definition}
\newtheorem*{exer}{Exercise}
\newtheorem{thm}{Theorem}[section]
\newtheorem*{thm*}{Theorem}
\newtheorem{lem}[thm]{Lemma}
\newtheorem*{lem*}{Lemma}
\newtheorem{cor}{Corollary}
\newtheorem{conj}{Conjecture}

\renewenvironment{proof}%
{\begin{quote} \emph{Proof:} }%
{\rule{0pt}{0pt} \newline \rule{0pt}{15pt} \hfill Q.E.D. \end{quote}}


\addtolength{\abovedisplayskip}{0pt}
\addtolength{\belowdisplayskip}{24pt}
\addtolength{\abovedisplayshortskip}{0pt}
\addtolength{\belowdisplayshortskip}{48pt}


\begin{document}
\thispagestyle{empty}

\centerline{\Large Activity 25 -- Introduction to Proof}
\centerline{\large mathematical induction}

\bigskip
\Large

Outline of proof by mathematical induction:

We want to prove a statement that is universally quantified over the
natural numbers (or something similar):

\[
\forall n \in \Naturals, \; P_n.
\]

The proof consists of 2 parts:
\begin{enumerate}
\item Prove that $P_0$ is true.
\item Prove that for some (particular but arbitrarily chosen) $k \in \Naturals $
$ P_k \implies P_{k+1} $.
\end{enumerate}

Step 1 is known as the {\em base case} (a.k.a. {\em basis}).

Step 2 is known as the {\em inductive step}.

The parts of the conditional sentence in step 2 are called the {\em inductive
hypothesis} and the {\em inductive conclusion} (respectively).

\newpage


\begin{enumerate}
\item Suppose we are trying to prove the following statement:
\[ \forall n \in \Naturals, \sum_{i=0}^{n} 2i + 1 = (n + 1)^2. \]


Match the expressions with the terminology.

\begin{tabular}{lcr}
\rule{0pt}{24pt} & \rule{100pt}{0pt} & inductive step\\
\rule{0pt}{40pt}$\displaystyle \sum_{i=0}^0 2i + 1 = (0 + 1)^2$ & & \\
\rule{0pt}{24pt}& & inductive conclusion\\
\rule{0pt}{40pt}$\displaystyle \sum_{i=0}^{k+1} 2i + 1 = (k + 2)^2$ & & \\
\rule{0pt}{24pt}& & inductive hypothesis\\
\rule{0pt}{40pt}$\displaystyle \sum_{i=0}^{k} 2i + 1 = (k + 1)^2$ & & \\
\rule{0pt}{24pt} & & base case\\
\end{tabular}

\vspace{.4in}

\item Referring to the previous problem, what would you call the part of
the proof where one showed that

\[ \forall k \in \Naturals, \left( \sum_{i=0}^{k} 2i + 1 = (k + 1)^2 \right) \quad \implies \quad \left( \sum_{i=0}^{k+1} 2i + 1 = (k + 2)^2 \right). \]

\vfill

\newpage

\item There is a category of mathematical sentences known as ``postage
stamp problems'' that can be proved using induction.  Consider the
following:

\begin{quote}
Any amount of postage greater than 7\cents can be created
using some number of 3\cents and some number of 5\cents stamps.
\end{quote}

\begin{enumerate}
\item[(a)] Write out the ways to use 3’s and 5’s to get all the possible
postages between 10 and 20.

\vfill

\item[(b)] What would the base case in an inductive proof of this state-
ment be?

\vfill

\end{enumerate}

\item Consider the statement

\[ \forall n \in \Naturals, \sum_{j=0}^{n} j^3 \; = \; \left( \frac{n(n + 1)}{2} \right)^2. \]

Verify the base case for proving this statement by induction.

\vfill


\newpage

\item (Exercise 2 in \S 5.1 in GIAM) What is wrong with the following
inductive proof of ``all horses are the same color.''?

(P.S.\ I hope it’s clear that this can’t be a true theorem, we’re
being asked to figure out what’s wrong with an argument that
appears to prove something that we know isn’t true.)

\begin{thm*}
 Let H be a set of n horses, all horses in H are the
same color.
\end{thm*}

\begin{proof}
We proceed by induction on n.

Basis: \newline
Suppose H is a set containing 1 horse.  Clearly
this horse is the same color as itself.

Inductive step: \newline
Given a set of k+1 horses H we can con-
struct two sets of k horses.  Suppose $H = \{h_1 , h_2 , h_3 , {. . .} h_{k+1} \}$.
Define $H_a = \{h_1 , h_2 , h_3 , {. . .} h_{k} \}$ (i.e.\ $H_a$ contains just the
first $k$ horses) and $H_b = \{h_2 , h_3 , h_4 , {. . .} h_{k+1} \}$ (i.e.\ $H_b$ con-
tains the last $k$ horses).  By the inductive hypothesis both
these sets contain horses that are “all the same color.”
Also, all the horses from $h_2$ to $h_k$ are in both sets so both
$H_a$ and $H_b$ contain only horses of this (same) color.  Finally, we conclude that all the horses in $H$ are the same
color.
\end{proof}


Hint: Look carefully at the early stages.

\vfill

\newpage

\item It’s relatively common that the base case in an inductive proof is
true for vacuous reasons. (This can feel relatively unsettling, so
I usually do two things: (1) check that the smallest non-vacuous
statement is also true, and (2) pay careful attention in the inductive
step to seeing that the vacuous case implies the first non-vacuous
case.) A common situation involves sums and products that have
no entries.  Suppose $EMPTY$ represents a sum having no terms
and $SOME$ is some other sum.  What is $SOME + EMPTY$ ?
Explain why an empty sum is equal to 0.

\vfill

\item This exercise is about understanding an empty product.  Like an
empty sum, we define an empty product in terms of how it behaves.
We think of a product as a list of numbers, and multiplying two
such lists is really just concatenating the lists.  For example, if
$a = 3 \cdot 5 \cdot 7$ and $b = 2 \cdot 4 \cdot 6$, then $ab = 3 \cdot 5 \cdot 7 \cdot 2 \cdot 4 \cdot 6$.  What if one
of those lists had no members?
Let $PROD_n$ be the product of the numbers from $1$ to $n$. (In other
words, $PROD_n$ is the same thing as $n!$).  How many multiplicands
are in $PROD_n$ ? Is it clear that $PROD_0$ is an empty product?
Consider $PROD_0 \cdot ANY$ , where $ANY$ is any other product.  What
must be the value of an empty product?

\vfill

\newpage

\item A {\em taxicab path} is a piece-wise linear path, with pieces that are either
horizontal or vertical, that run between the points in the plane
whose coordinates are integers.  Suppose we wished to count all of
the taxicab paths from $(0, 0)$ to $(n, n)$, where $n \in \Naturals$.  It’s necessary
to add an additional restriction or the answer will always be infinity.
A {\em reasonable taxicab path} always moves towards the goal, so it
will consist of line segments that go in the positive x direction or
positive y direction only.  Play with this question and try to discover
a formula for the number of reasonable taxicab paths from $(0, 0)$
to $(n, n)$.  What does your formula say about the base case: a
reasonable taxicab path from $(0, 0)$ to $(0, 0)$?

\vfill

\end{enumerate}


\end{document}
