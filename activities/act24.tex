\documentclass{amsart}
\usepackage{amssymb}
\renewcommand{\baselinestretch}{1.5}
\addtolength{\textwidth}{.2in}
\addtolength{\topmargin}{-.5in}
\addtolength{\textheight}{1in}

\usepackage{ifthen}
\usepackage{graphicx}
\usepackage{xcolor}

\newcommand{\versionNum}{$3.2$\ }

\newboolean{InTextBook}
\setboolean{InTextBook}{false}
\newboolean{InWorkBook}
\setboolean{InWorkBook}{false}
\newboolean{InHints}
\setboolean{InHints}{false}

%When this boolean is true (beginning in Section 5.1) we will use the convention
% that $0 \in \Naturals$.  If it is false we will continue to count $1$ as the smallest
%natural number (thus making Giuseppe Peano spin in his grave...)
 
\newboolean{ZeroInNaturals}

%This boolean is used to distinguish the version where we use $\sim$ rather than $\lnot$

\newboolean{LNotIsSim}

%The values of the last two booleans are set in ``switches.tex''

%\input{switches}

\let\savedlnot\lnot
\ifthenelse{\boolean{LNotIsSim}}{\renewcommand{\lnot}{\sim} }{}

%This command puts different amounts of space depending on whether we are
% in the text, the workbook or the hints & solutions manual. 
\newcommand{\twsvspace}[3]{%
 \ifthenelse{\boolean{InTextBook} }{\vspace{#1}}{%
  \ifthenelse{\boolean{InWorkBook} }{\vspace{#2}}{%
   \ifthenelse{\boolean{InHints} }{\vspace{#3}}{} %
   }%
  }%
 }


\newcommand{\wbvfill}{\ifthenelse{\boolean{InWorkBook}}{\vfill}{}}
\newcommand{\wbitemsep}{\ifthenelse{\boolean{InWorkBook} }{\rule[-24pt]{0pt}{60pt}}{}}
\newcommand{\textbookpagebreak}{\ifthenelse{\boolean{InTextBook}}{\newpage}{}}
\newcommand{\workbookpagebreak}{\ifthenelse{\boolean{InWorkBook}}{\newpage}{}}
\newcommand{\hintspagebreak}{\ifthenelse{\boolean{InHints}}{\newpage}{}}

\newcommand{\hint}[1]{\ifthenelse{\boolean{InHints}}{ {\par \hspace{12pt} \color[rgb]{0,0,1} #1 } }{}}
\newcommand{\inlinehint}[1]{\ifthenelse{\boolean{InHints}}{ { \color[rgb]{0,0,1} #1 } }{}}

%\newlength{\cwidth}
%\newcommand{\cents}{\settowidth{\cwidth}{c}%
%\divide\cwidth by2
%\advance\cwidth by-.1pt
%c\kern-\cwidth
%\vrule width .1pt depth.2ex height1.2ex
%\kern 3\cwidth}
\newcommand{\cents}{\textcent\kern 5pt}

\newcommand{\sageprompt}{ {\tt sage$>$} }
\newcommand{\tab}{\rule{20pt}{0pt}}
\newcommand{\blnk}{\rule{1.5pt}{0pt}\rule{.4pt}{1.2pt}\rule{9pt}{.4pt}\rule{.4pt}{1.2pt}\rule{1.5pt}{0pt}}
\newcommand{\suchthat}{\; \rule[-3pt]{.5pt}{13pt} \;}
\newcommand{\divides}{\!\mid\!}
\newcommand{\tdiv}{\; \mbox{div} \;}
\newcommand{\restrict}[2]{#1 \,\rule[-4pt]{.25pt}{14pt}_{\,#2}}
\newcommand{\lcm}[2]{\mbox{lcm} (#1, #2)}
\renewcommand{\gcd}[2]{\mbox{gcd} (#1, #2)}
\newcommand{\Naturals}{{\mathbb N}}
\newcommand{\Integers}{{\mathbb Z}}
\newcommand{\Znoneg}{{\mathbb Z}^{\mbox{\tiny noneg}}}
\ifthenelse{\boolean{ZeroInNaturals}}{%
  \newcommand{\Zplus}{{\mathbb Z}^+} }{%
  \newcommand{\Zplus}{{\mathbb N}} }
\newcommand{\Enoneg}{{\mathbb E}^{\mbox{\tiny noneg}}}
\newcommand{\Qnoneg}{{\mathbb Q}^{\mbox{\tiny noneg}}}
\newcommand{\Rnoneg}{{\mathbb R}^{\mbox{\tiny noneg}}}
\newcommand{\Rationals}{{\mathbb Q}}
\newcommand{\Reals}{{\mathbb R}}
\newcommand{\Complexes}{{\mathbb C}}
%\newcommand{\F2}{{\mathbb F}_{2}}
\newcommand{\relQ}{\mbox{\textsf Q}}
\newcommand{\relR}{\mbox{\textsf R}}
\newcommand{\nrelR}{\mbox{\raisebox{1pt}{$\not$}\rule{1pt}{0pt}{\textsf R}}}
\newcommand{\relS}{\mbox{\textsf S}}
\newcommand{\relA}{\mbox{\textsf A}}
\newcommand{\Dom}[1]{\mbox{Dom}(#1)}
\newcommand{\Cod}[1]{\mbox{Cod}(#1)}
\newcommand{\Rng}[1]{\mbox{Rng}(#1)}

\DeclareMathOperator\caret{\raisebox{1ex}{$\scriptstyle\wedge$}}

\newtheorem*{defi}{Definition}
\newtheorem*{exer}{Exercise}
\newtheorem{thm}{Theorem}[section]
\newtheorem*{thm*}{Theorem}
\newtheorem{lem}[thm]{Lemma}
\newtheorem*{lem*}{Lemma}
\newtheorem{cor}{Corollary}
\newtheorem{conj}{Conjecture}

\renewenvironment{proof}%
{\begin{quote} \emph{Proof:} }%
{\rule{0pt}{0pt} \newline \rule{0pt}{15pt} \hfill Q.E.D. \end{quote}}


\addtolength{\abovedisplayskip}{0pt}
\addtolength{\belowdisplayskip}{24pt}
\addtolength{\abovedisplayshortskip}{0pt}
\addtolength{\belowdisplayshortskip}{48pt}


\begin{document}
\thispagestyle{empty}

\centerline{\Large Activity 24 -- Introduction to Proof}
\centerline{\large Russell's paradox}

\bigskip
\Large

{\bf \large Set Theory review questions}
\begin{enumerate}

\item Use the basic set-theoretic equalities to develop a two-column proof
of the following.
\[
(A \cap C) \cap (A  \cup  B) = A \cap C
\]

\vfill

\item There are three individuals named Abraham, Balthazar and Cincin-
natus.  One of them is a knight (who always makes truthful state-
ments), one is a knave (who always makes false statements) and
one is a kneither (whose statements can be either true or false).
\medskip

\begin{quotation}
Abraham says “Cincinnatus is a knave.”\newline
Balthazar says “Abraham is a knight.” \newline
Cincinnatus says “I am a kneither.” \newline
\end{quotation}

State – {\bf with proof} – who is the knight, who is the knave and
who is the kneither?

\vfill

\newpage

\item Let A be the set $\{\emptyset, 1, 2, \{3, 4, 5\}\}$.  Mark the following statements \newline
True or False.

\begin{tabular}{lcl}
\rule{0pt}{84pt} a)  $ \emptyset \subseteq A $ & \rule{1.5in}{0in} & b)  $\emptyset \in A$\\
\rule{0pt}{84pt} c) $ \{1\}  \subseteq  A $ & & d) $\{3, 4, 5\}  \subseteq  A$\\
\rule{0pt}{84pt} e) $ \{ \emptyset \}  \subseteq  A $  & & f) $\{1, 2\}  \in  A$\\
\rule{0pt}{84pt} g) $ \{3, 4, 5\}  \in  A $ & & h) $\{1, 2\}  \subseteq  A$\\
\end{tabular}

\rule{0pt}{84pt}

\item What interval would the following infinite union be equal to?

\[ \bigcup_{n=2}^\infty \; \left[ 1/n, 1 + 1/n \right] \]

\vfill

\newpage

\item Use a Venn diagram to solve the following:

Irving’s Used Cars currently has 71 vehicles on hand.  There are
29 whose engines won’t start.  There are 21 that need bodywork.
And there are 17 that need deep cleaning because they smell very
bad.  Three of the smelly cars also need bodywork and engine
repairs.  There are a total of 11 cars that need both engine and
body repairs (including the three smelly ones).  There is 1 car that
stinks and has a bad engine but it’s body is in good shape, and
there is 1 car that stinks and has lots of rust but its engine sounds
great.

How many cars are currently suitable for sale?

\vfill

\newpage

{\bf \large Questions regarding Russell's paradox}
\bigskip

\item Verify that $(A \Longrightarrow \lnot A) \wedge (\lnot A \Longrightarrow A)$ is a logical contradiction
by filling out a truth table.

\vfill

\item Would it be possible to write a book that catalogued all those books
(and only those books) that do not refer to themselves?

\vfill

\newpage

\item In a Star Trek (TOS) episode an evil robot has taken over their
spaceship.  Captain Kirk and Mr.\ Spock approach it and say the
following lines

Kirk: You know, everything Mr.\ Spock says is a lie.

Spock: Actually, everything Captain Kirk says is the truth.

Why does the robot subsequently explode?

\vfill

\item Is there any problem with defining n to be “the smallest positive
integer that cannot be described in fewer than fourteen words” ?

Describe the paradox.

\vfill

\newpage

\item Alan Turing, the famous british mathematician and computer sci-
entist, used an argument similar to the one in Russell’s paradox to
show the impossibility of solving the halting problem. \A solution
to the halting problem would be a program that could be used to
determine whether a given program (with its associated input) will
enter into an infinite loop or will halt in a finite amount of time.
Assume that {\tt Check(P,I)} is such a program.  The {\tt Check} program
takes a program P and its input I as its input and either outputs
True (if that program applied to that input will halt) or False (if
that program applied to that input will loop forever).  Turing’s
clever argument involved thinking about applying {\tt Check} where P
and I are both the same – in other words ask whether program P,
given its own code listing as input would halt or not.
Turing then defined a new program, let’s call it Strange, as
follows:
\medskip

\begin{verbatim}
Strange(x):
    if Check(x,x):
        while True:
            print "Aaargh! I’m in an infinite loop!"
    else
        print "Cool, I’m done."
        return
\end{verbatim}
\medskip


Try to figure out what will happen if we run {\tt Strange(Strange)}.


\end{enumerate}


\end{document}
