\documentclass{amsart}

\usepackage{hyperref}
\usepackage{color}
\usepackage{graphicx}

\hypersetup{colorlinks=true}

\addtolength{\topmargin}{-.5 in}
\addtolength{\textheight}{.5 in}
\addtolength{\oddsidemargin}{-.5 in}
\addtolength{\evensidemargin}{-.5 in}
\addtolength{\textwidth}{1 in}

\newcommand{\arcsec}{ {\rm arcsec}}
\newcommand{\arccsc}{ {\rm arccsc}}
\newcommand{\arccot}{ {\rm arccot}}
\newcommand{\diff}{\frac{\mbox{d}}{\mbox{d}x}\,}
\newcommand{\dx}{\;\mbox{d}x}
\newcommand{\dy}{\;\mbox{d}y}
\newcommand{\dz}{\;\mbox{d}z}

\newcommand{\dr}{\;\mbox{d}r}
\newcommand{\ds}{\;\mbox{d}s}
\newcommand{\dt}{\;\mbox{d}t}

\newcommand{\dtheta}{\;\mbox{d}\theta}
\newcommand{\dphi}{\;\mbox{d}\phi}
\newcommand{\drho}{\;\mbox{d}\rho}
\newcommand{\dA}{\;\mbox{d}A}
\newcommand{\dV}{\;\mbox{d}V}

\newcommand{\Integers}{ {\mathbb Z} }
\newcommand{\Rationals}{ {\mathbb Q} }
\newcommand{\Reals}{ {\mathbb R} }

\newcommand{\vs}{\rule[-24pt]{0pt}{60pt}}

\pagestyle{empty}

\begin{document}
\thispagestyle{empty}

\centerline{\Large \bf Activity -- MAT 252 -- Spring 2020}
\bigskip
\centerline{\large \bf November 18, 2020}

\Large


The flux-divergence form of Green's theorem tells us that an integral of 
normal flow (a.k.a. flux) taken around the boundary
of a region,

\[ \int_C \vec{F} \cdot \vec{n} \; \mbox{d}s, \]

is equal to the integral of the divergence of the vector field taken 
over the region

\[ \int\int_R f_x + g_y  \; \dA , \]

\noindent where $f$ and $g$ are the components of the vector field $\vec{F}$.

\vfill

The flow-curl form of Green's theorem tells us that an integral of 
tangential flow (usually just called flow) taken around the boundary
of a region,

\[ \int_C \vec{F} \cdot \vec{T} \; \mbox{d}s, \]

is equal to the integral of the $k$-component of the curl of the vector field taken 
over the region

\[ \int\int_R g_x - f_y  \; \dA , \]

\noindent where $f$ and $g$ are the components of the vector field $\vec{F}$.

\vfill

\newpage

\begin{enumerate}

\item Calculate the flow integral ($\int F \cdot T \ds$) of the
vector field $\vec{F} = \langle y, -x \rangle$ around the unit circle
$\langle \cos{(t)}, \sin{(t)} \rangle$ for $0 \leq t \leq 2\pi$.

\vfill

\vfill

\item What double integral should give us the same result?

\vfill

\newpage

\item Calculate the flux integral ($\int F \cdot n \mbox{d}s$) of the 
vector field $\langle x, y \rangle$ over the square path whose vertices
are $(1,1)$, $(-1,1)$, $(-1,-1)$ and $(1,-1)$. 

\vfill

\vfill

\item What double integral should give us the same result?

\vfill

\newpage

\item Use Green's theorem to find the flux integral 

\[ \int_C \vec{F} \cdot \vec{n} \mbox{d}s \]

\noindent where $F$ is the vector field $\langle x, y \rangle$ and $C$
is the closed path made up by following the curve $y=x^2-1$ from
$(-1,0)$ to $(1,0)$ and then following the curve $y = 1 - x^2$ back to
$(-1,0)$.


\newpage

\item Use Green's theorem to find the flow integral 

\[ \oint_C \vec{F} \cdot \vec{T} \ds \]

\noindent where $F$ is the vector field $\langle x^2+y, 2x-y^2 \rangle$ and $C$
is the closed path made up by following the curve $y=x^3$ from
$(0,0)$ to $(1,1)$ and then following the curve $y = x^2$ back to
$(0,0)$. 


\newpage


\item Use Green's theorem to find the path integral 

\[ \int_C \vec{F} \cdot \vec{n} \; \mbox{d}s, \]

where $\vec{F} = \langle x, y \rangle$ and $C$ is the unit circle 
given parametrically by $\langle \cos{(t)}, \sin{(t)} \rangle$ for
$0 \leq t \leq 2\pi$.  (Note that the outward normal vector is equal to the
position vector on this curve.)

\vfill

\item Suppose that the vector field $\vec{F} \, = \, \langle f, g \rangle$ satisfies $g_x \, = \, f_y$.  What value does this imply for all line integrals $\int \vec{F} \mbox{d}\vec{r}$ where the integral is taken over a simple closed curve?  Explain why this fact tells us that such line integrals are path independent.

\vfill

\newpage

\item A {\bf stream function} for a vector field $\vec{F} \, = \, \langle f, g \rangle$ is a function of two variables whose level curves are so-called {\bf streamlines} -- curves that are aligned with the flow of the vector field.

In general a stream function $\psi(x,y)$ satisfies $\psi_x = -g$ and $\psi_y = f$.  

Find a stream function for the vector field $\vec{F} \, = \, \langle x^2, -2xy \rangle$.

\vfill

\item Both potential functions and stream functions satisfy an important partial differential equation known as Laplace's equation:

\[ \phi_{xx} + \phi_{yy} = 0 \quad \mbox{and} \quad  \psi_{xx} + \psi_{yy} = 0.  \]

Verify Laplace's equation for the stream function in the previous problem.

\vfill

\end{enumerate}

\end{document}

