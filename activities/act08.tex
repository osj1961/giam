\documentclass{amsart}
%\usepackage{}
\renewcommand{\baselinestretch}{1.5}
\addtolength{\textwidth}{.2in}
\newcommand{\versionNum}{$3.2$\ }

\newboolean{InTextBook}
\setboolean{InTextBook}{false}
\newboolean{InWorkBook}
\setboolean{InWorkBook}{false}
\newboolean{InHints}
\setboolean{InHints}{false}

%When this boolean is true (beginning in Section 5.1) we will use the convention
% that $0 \in \Naturals$.  If it is false we will continue to count $1$ as the smallest
%natural number (thus making Giuseppe Peano spin in his grave...)
 
\newboolean{ZeroInNaturals}

%This boolean is used to distinguish the version where we use $\sim$ rather than $\lnot$

\newboolean{LNotIsSim}

%The values of the last two booleans are set in ``switches.tex''

%\input{switches}

\let\savedlnot\lnot
\ifthenelse{\boolean{LNotIsSim}}{\renewcommand{\lnot}{\sim} }{}

%This command puts different amounts of space depending on whether we are
% in the text, the workbook or the hints & solutions manual. 
\newcommand{\twsvspace}[3]{%
 \ifthenelse{\boolean{InTextBook} }{\vspace{#1}}{%
  \ifthenelse{\boolean{InWorkBook} }{\vspace{#2}}{%
   \ifthenelse{\boolean{InHints} }{\vspace{#3}}{} %
   }%
  }%
 }


\newcommand{\wbvfill}{\ifthenelse{\boolean{InWorkBook}}{\vfill}{}}
\newcommand{\wbitemsep}{\ifthenelse{\boolean{InWorkBook} }{\rule[-24pt]{0pt}{60pt}}{}}
\newcommand{\textbookpagebreak}{\ifthenelse{\boolean{InTextBook}}{\newpage}{}}
\newcommand{\workbookpagebreak}{\ifthenelse{\boolean{InWorkBook}}{\newpage}{}}
\newcommand{\hintspagebreak}{\ifthenelse{\boolean{InHints}}{\newpage}{}}

\newcommand{\hint}[1]{\ifthenelse{\boolean{InHints}}{ {\par \hspace{12pt} \color[rgb]{0,0,1} #1 } }{}}
\newcommand{\inlinehint}[1]{\ifthenelse{\boolean{InHints}}{ { \color[rgb]{0,0,1} #1 } }{}}

%\newlength{\cwidth}
%\newcommand{\cents}{\settowidth{\cwidth}{c}%
%\divide\cwidth by2
%\advance\cwidth by-.1pt
%c\kern-\cwidth
%\vrule width .1pt depth.2ex height1.2ex
%\kern 3\cwidth}
\newcommand{\cents}{\textcent\kern 5pt}

\newcommand{\sageprompt}{ {\tt sage$>$} }
\newcommand{\tab}{\rule{20pt}{0pt}}
\newcommand{\blnk}{\rule{1.5pt}{0pt}\rule{.4pt}{1.2pt}\rule{9pt}{.4pt}\rule{.4pt}{1.2pt}\rule{1.5pt}{0pt}}
\newcommand{\suchthat}{\; \rule[-3pt]{.5pt}{13pt} \;}
\newcommand{\divides}{\!\mid\!}
\newcommand{\tdiv}{\; \mbox{div} \;}
\newcommand{\restrict}[2]{#1 \,\rule[-4pt]{.25pt}{14pt}_{\,#2}}
\newcommand{\lcm}[2]{\mbox{lcm} (#1, #2)}
\renewcommand{\gcd}[2]{\mbox{gcd} (#1, #2)}
\newcommand{\Naturals}{{\mathbb N}}
\newcommand{\Integers}{{\mathbb Z}}
\newcommand{\Znoneg}{{\mathbb Z}^{\mbox{\tiny noneg}}}
\ifthenelse{\boolean{ZeroInNaturals}}{%
  \newcommand{\Zplus}{{\mathbb Z}^+} }{%
  \newcommand{\Zplus}{{\mathbb N}} }
\newcommand{\Enoneg}{{\mathbb E}^{\mbox{\tiny noneg}}}
\newcommand{\Qnoneg}{{\mathbb Q}^{\mbox{\tiny noneg}}}
\newcommand{\Rnoneg}{{\mathbb R}^{\mbox{\tiny noneg}}}
\newcommand{\Rationals}{{\mathbb Q}}
\newcommand{\Reals}{{\mathbb R}}
\newcommand{\Complexes}{{\mathbb C}}
%\newcommand{\F2}{{\mathbb F}_{2}}
\newcommand{\relQ}{\mbox{\textsf Q}}
\newcommand{\relR}{\mbox{\textsf R}}
\newcommand{\nrelR}{\mbox{\raisebox{1pt}{$\not$}\rule{1pt}{0pt}{\textsf R}}}
\newcommand{\relS}{\mbox{\textsf S}}
\newcommand{\relA}{\mbox{\textsf A}}
\newcommand{\Dom}[1]{\mbox{Dom}(#1)}
\newcommand{\Cod}[1]{\mbox{Cod}(#1)}
\newcommand{\Rng}[1]{\mbox{Rng}(#1)}

\DeclareMathOperator\caret{\raisebox{1ex}{$\scriptstyle\wedge$}}

\newtheorem*{defi}{Definition}
\newtheorem*{exer}{Exercise}
\newtheorem{thm}{Theorem}[section]
\newtheorem*{thm*}{Theorem}
\newtheorem{lem}[thm]{Lemma}
\newtheorem*{lem*}{Lemma}
\newtheorem{cor}{Corollary}
\newtheorem{conj}{Conjecture}

\renewenvironment{proof}%
{\begin{quote} \emph{Proof:} }%
{\rule{0pt}{0pt} \newline \rule{0pt}{15pt} \hfill Q.E.D. \end{quote}}


\begin{document}
\thispagestyle{empty}

\centerline{\Large Activity 8 -- Introduction to Proof}
\centerline{\large logic}

\bigskip
\Large


\begin{enumerate}
\item Which of the following sentences are statements?  For those which are not, is there a way to modify them so that they are?

\begin{enumerate}
\item The domain of the divisibility relation is $\Integers$.
\item $x$ is a prime number.
\item The range of the divisibility relation is $\Reals$.
\item This sentence is false.
\end{enumerate}

\vspace{.2in}

\item Several compound sentences can be constructed using either the $\land$ or the $\lor$ operator together with $\lnot$'s on the predicate variables.  (For example, $\lnot A \lor B$ or $\lnot A \land \lnot B$.)  Assuming that $A$ and $B$ appear in that order, there are 8 such sentences. 
List them.

\vfill

\newpage

\item A shortcut to figuring out the truth tables of sentences like the ones from the previous problem is this:  ``Or'' statements are {\em false} in exactly one row,  ``And'' statements are 
{\em true} in exactly one row.  

We take the ordering of the rows of a 2-variable truth table to be:

\begin{center}
\begin{tabular}{r|c|c}
 & \rule{12pt}{0pt} $A$ \rule{12pt}{0pt} & \rule{12pt}{0pt} $B$ \rule{12pt}{0pt} \\ \hline
row 1: \rule{12pt}{0pt} & T & T \\
row 2: \rule{12pt}{0pt} & T & $\phi$ \\
row 3: \rule{12pt}{0pt} &  $\phi$ & T \\
row 4: \rule{12pt}{0pt} &  $\phi$ &  $\phi$ \\
\end{tabular}
\end{center}

\vspace{.3in}

With the above convention what row number(s) are the following sentences true in?

\begin{enumerate}
\item $A \land B$
\item $A \land \lnot B$
\item $\lnot A \land B$
\end{enumerate}

\vspace{.3in}

What row noumbers are the following false in?

\begin{enumerate}
\item $\lnot A \lor B$
\item $A \lor \lnot B$
\item $\lnot A \lor \lnot B$
\end{enumerate}

\newpage

\item Now we should be ready to create all 8 truth tables for the statements from problem 2.  I've pre-filled the skeletons of the tables to ease your burden.

\vspace{.3in}

\hspace{-1 in}\begin{tabular}{cccc} 
\begin{tabular}{c|c|c}
 $A$ & $B$  &  $A\land B$ \\ \hline
T & T & \\
T & $\phi$ & \\
$\phi$ & T & \\
$\phi$ &  $\phi$ & \\
\end{tabular}
\rule{6pt}{0pt} & \rule{6pt}{0pt}
\begin{tabular}{c|c|c}
 $A$  & $B$  & $A\land \lnot B$  \\ \hline
T & T & \\
T & $\phi$ & \\
$\phi$ & T & \\
$\phi$ &  $\phi$ & \\
\end{tabular}
\rule{6pt}{0pt} & \rule{6pt}{0pt}
\begin{tabular}{c|c|c}
 $A$ &  $B$ &  $\lnot A \land B$\\ \hline
T & T & \\
T & $\phi$ & \\
$\phi$ & T & \\
$\phi$ &  $\phi$ & \\
\end{tabular}
\rule{6pt}{0pt} & \rule{6pt}{0pt}
\begin{tabular}{c|c|c}
$A$ &  $B$  &  $\lnot A \land \lnot B$ \\ \hline
T & T & \\
T & $\phi$ & \\
$\phi$ & T & \\
$\phi$ &  $\phi$ & \\
\end{tabular}
\\
\rule{0pt}{36pt} & & & \\
\begin{tabular}{c|c|c}
 $A$ & $B$  &  $A\lor B$ \\ \hline
T & T & \\
T & $\phi$ & \\
$\phi$ & T & \\
$\phi$ &  $\phi$ & \\
\end{tabular}
\rule{6pt}{0pt} & \rule{6pt}{0pt}
\begin{tabular}{c|c|c}
 $A$  & $B$  & $A\lor \lnot B$  \\ \hline
T & T & \\
T & $\phi$ & \\
$\phi$ & T & \\
$\phi$ &  $\phi$ & \\
\end{tabular}
\rule{6pt}{0pt} & \rule{6pt}{0pt}
\begin{tabular}{c|c|c}
 $A$ &  $B$ &  $\lnot A \lor B$\\ \hline
T & T & \\
T & $\phi$ & \\
$\phi$ & T & \\
$\phi$ &  $\phi$ & \\
\end{tabular}
\rule{6pt}{0pt} & \rule{6pt}{0pt}
\begin{tabular}{c|c|c}
$A$ &  $B$  &  $\lnot A \lor \lnot B$ \\ \hline
T & T & \\
T & $\phi$ & \\
$\phi$ & T & \\
$\phi$ &  $\phi$ & \\
\end{tabular}
\\
\end{tabular}

\vspace{.5in}

\item Create a truth table for $A \land (A \lor B)$

\vfill

\newpage

\item Create a truth table for $A \lor (A \land B)$

\vfill

\item Comparing the last two problems, what do you notice?

\vspace{1 in}

\item Draw the digital logic diagram for $A \land (A \lor B)$ and $A \lor (A \land B)$.  How could they be simplified?

\vfill

\newpage

\item In the video we mentioned the {\em exclusive or} operator which is true when exactly one of its inputs is true.  Usually, exclusive or is notated thusly: $A \oplus B$.

Using the notion of {\em disjunctive normal form}\footnote{Recall that disjunctive normal form consists of the ``or'' of a bunch of ``and'' statements that recognize the rows we want to be true.} determine an expression that only uses $\land$, $\lor$ and $\lnot$ that is equivalent to $A \oplus B$.

\vfill


\item Construct a digital logic diagram -- using the disjunctive normal form concept -- for the following truth table.  Challenge: A much simpler logic diagram is possible.  Can you find it?

\vspace{.2in}

\begin{tabular}{c|c|c||c}
 $A$ &  $B$ &  $C$ & ???\\ \hline
T & T & T & $\phi$ \\
T & T & $\phi$ & T \\
T & $\phi$ & T & $\phi$ \\
T & $\phi$ &  $\phi$ & T\\
$\phi$ & T & T & $\phi$\\
$\phi$ & T & $\phi$ & $\phi$\\
$\phi$ & $\phi$ & T & $\phi$\\
$\phi$ & $\phi$ &  $\phi$ & $\phi$\\
\end{tabular}


\vfill

\end{enumerate}


\end{document}
