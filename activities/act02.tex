\documentclass{amsart}
%\usepackage{}
\renewcommand{\baselinestretch}{1.5}

\newcommand{\versionNum}{$3.2$\ }

\newboolean{InTextBook}
\setboolean{InTextBook}{false}
\newboolean{InWorkBook}
\setboolean{InWorkBook}{false}
\newboolean{InHints}
\setboolean{InHints}{false}

%When this boolean is true (beginning in Section 5.1) we will use the convention
% that $0 \in \Naturals$.  If it is false we will continue to count $1$ as the smallest
%natural number (thus making Giuseppe Peano spin in his grave...)
 
\newboolean{ZeroInNaturals}

%This boolean is used to distinguish the version where we use $\sim$ rather than $\lnot$

\newboolean{LNotIsSim}

%The values of the last two booleans are set in ``switches.tex''

%\input{switches}

\let\savedlnot\lnot
\ifthenelse{\boolean{LNotIsSim}}{\renewcommand{\lnot}{\sim} }{}

%This command puts different amounts of space depending on whether we are
% in the text, the workbook or the hints & solutions manual. 
\newcommand{\twsvspace}[3]{%
 \ifthenelse{\boolean{InTextBook} }{\vspace{#1}}{%
  \ifthenelse{\boolean{InWorkBook} }{\vspace{#2}}{%
   \ifthenelse{\boolean{InHints} }{\vspace{#3}}{} %
   }%
  }%
 }


\newcommand{\wbvfill}{\ifthenelse{\boolean{InWorkBook}}{\vfill}{}}
\newcommand{\wbitemsep}{\ifthenelse{\boolean{InWorkBook} }{\rule[-24pt]{0pt}{60pt}}{}}
\newcommand{\textbookpagebreak}{\ifthenelse{\boolean{InTextBook}}{\newpage}{}}
\newcommand{\workbookpagebreak}{\ifthenelse{\boolean{InWorkBook}}{\newpage}{}}
\newcommand{\hintspagebreak}{\ifthenelse{\boolean{InHints}}{\newpage}{}}

\newcommand{\hint}[1]{\ifthenelse{\boolean{InHints}}{ {\par \hspace{12pt} \color[rgb]{0,0,1} #1 } }{}}
\newcommand{\inlinehint}[1]{\ifthenelse{\boolean{InHints}}{ { \color[rgb]{0,0,1} #1 } }{}}

%\newlength{\cwidth}
%\newcommand{\cents}{\settowidth{\cwidth}{c}%
%\divide\cwidth by2
%\advance\cwidth by-.1pt
%c\kern-\cwidth
%\vrule width .1pt depth.2ex height1.2ex
%\kern 3\cwidth}
\newcommand{\cents}{\textcent\kern 5pt}

\newcommand{\sageprompt}{ {\tt sage$>$} }
\newcommand{\tab}{\rule{20pt}{0pt}}
\newcommand{\blnk}{\rule{1.5pt}{0pt}\rule{.4pt}{1.2pt}\rule{9pt}{.4pt}\rule{.4pt}{1.2pt}\rule{1.5pt}{0pt}}
\newcommand{\suchthat}{\; \rule[-3pt]{.5pt}{13pt} \;}
\newcommand{\divides}{\!\mid\!}
\newcommand{\tdiv}{\; \mbox{div} \;}
\newcommand{\restrict}[2]{#1 \,\rule[-4pt]{.25pt}{14pt}_{\,#2}}
\newcommand{\lcm}[2]{\mbox{lcm} (#1, #2)}
\renewcommand{\gcd}[2]{\mbox{gcd} (#1, #2)}
\newcommand{\Naturals}{{\mathbb N}}
\newcommand{\Integers}{{\mathbb Z}}
\newcommand{\Znoneg}{{\mathbb Z}^{\mbox{\tiny noneg}}}
\ifthenelse{\boolean{ZeroInNaturals}}{%
  \newcommand{\Zplus}{{\mathbb Z}^+} }{%
  \newcommand{\Zplus}{{\mathbb N}} }
\newcommand{\Enoneg}{{\mathbb E}^{\mbox{\tiny noneg}}}
\newcommand{\Qnoneg}{{\mathbb Q}^{\mbox{\tiny noneg}}}
\newcommand{\Rnoneg}{{\mathbb R}^{\mbox{\tiny noneg}}}
\newcommand{\Rationals}{{\mathbb Q}}
\newcommand{\Reals}{{\mathbb R}}
\newcommand{\Complexes}{{\mathbb C}}
%\newcommand{\F2}{{\mathbb F}_{2}}
\newcommand{\relQ}{\mbox{\textsf Q}}
\newcommand{\relR}{\mbox{\textsf R}}
\newcommand{\nrelR}{\mbox{\raisebox{1pt}{$\not$}\rule{1pt}{0pt}{\textsf R}}}
\newcommand{\relS}{\mbox{\textsf S}}
\newcommand{\relA}{\mbox{\textsf A}}
\newcommand{\Dom}[1]{\mbox{Dom}(#1)}
\newcommand{\Cod}[1]{\mbox{Cod}(#1)}
\newcommand{\Rng}[1]{\mbox{Rng}(#1)}

\DeclareMathOperator\caret{\raisebox{1ex}{$\scriptstyle\wedge$}}

\newtheorem*{defi}{Definition}
\newtheorem*{exer}{Exercise}
\newtheorem{thm}{Theorem}[section]
\newtheorem*{thm*}{Theorem}
\newtheorem{lem}[thm]{Lemma}
\newtheorem*{lem*}{Lemma}
\newtheorem{cor}{Corollary}
\newtheorem{conj}{Conjecture}

\renewenvironment{proof}%
{\begin{quote} \emph{Proof:} }%
{\rule{0pt}{0pt} \newline \rule{0pt}{15pt} \hfill Q.E.D. \end{quote}}


\begin{document}
\thispagestyle{empty}

\centerline{\Large Activity 2 -- Introduction to Proof}
\centerline{\large Definitions and Primes}

\bigskip
\Large


\begin{enumerate}
\item The odd numbers could be defined by:

An integer $n$ is {\em odd} when there is another integer $k$ such that $n \, = \, 2k+1$.

But it would also be fine to use:

An integer $n$ is {\em odd} when there is another integer $k$ such that $n \, = \, 2k-1$.

Explain why these definitions are equivalent.  Can you produce yet another definition for ``odd''?

\vfill

\item To show that a number $x$ is prime (using the original definition) we need to examine all the potential divisors of $x$ and show that none of them (except $1$ and $x$) actually divide evenly into $x$.  This process is known as trial division.  On the surface it seems we'd need to do trial division for every number from $2$ to $x-1$.  Can we shorten that list?

\vfill 

\newpage

\item Carry out the Sieve of Eratosthenes only using the primes you find in the first row of the table.  If the table were larger, what would the smallest un-sieved composite number be? 

\begin{tabular}{cccccccccc}
\rule{24pt}{0pt} & \rule{24pt}{0pt} & \rule{24pt}{0pt} &
\rule{24pt}{0pt} & \rule{24pt}{0pt} & \rule{24pt}{0pt} & 
\rule{24pt}{0pt} & \rule{24pt}{0pt} & \rule{24pt}{0pt} &
\rule{24pt}{0pt} \\
\rule[-5pt]{0pt}{24pt} 1 & 2 & 3 & 4 & 5 & 6 & 7 & 8 & 9 & 10 \\
\rule[-5pt]{0pt}{24pt} 11 & 12 & 13 & 14 & 15 & 16 & 17 & 18 & 19 & 20 \\
\rule[-5pt]{0pt}{24pt} 21 & 22 & 23 & 24 & 25 & 26 & 27 & 28 & 29 & 30 \\
\rule[-5pt]{0pt}{24pt} 31 & 32 & 33 & 34 & 35 & 36 & 37 & 38 & 39 & 40 \\
\rule[-5pt]{0pt}{24pt} 41 & 42 & 43 & 44 & 45 & 46 & 47 & 48 & 49 & 50 \\
\rule[-5pt]{0pt}{24pt} 51 & 52 & 53 & 54 & 55 & 56 & 57 & 58 & 59 & 60 \\ 
\rule[-5pt]{0pt}{24pt} 61 & 62 & 63 & 64 & 65 & 66 & 67 & 68 & 69 & 70 \\
\rule[-5pt]{0pt}{24pt} 71 & 72 & 73 & 74 & 75 & 76 & 77 & 78 & 79 & 80 \\
\rule[-5pt]{0pt}{24pt} 81 & 82 & 83 & 84 & 85 & 86 & 87 & 88 & 89 & 90 \\
\rule[-5pt]{0pt}{24pt} 91 & 92 & 93 & 94 & 95 & 96 & 97 & 98 & 99 & 100
\end{tabular}

\vspace{.5in}

\item Suppose you wanted to discover all of the primes $\leq 1000$.  What primes would you need to use in the sieve?

\vfill

\item Figure out what's going on in the prime ``table'' on page 16 of GIAM.  Hint: why are the column headings labeled T and the row headings labeled H?

\vfill

\newpage

\item Use the prime table to find the twin primes between 4900 and 5000. 

\vfill

\item In a sage cell type \hspace{12pt} {\tt range?} \hspace{12pt}.  (This is how you get help on a sage function.)  Now use a for loop and the {\tt is\_prime} function to verify your answer from the previous question.

\vfill

\item A Fermat number is a number of the form $2^{2^n}+1$.  The famous mathematician Pierre de Fermat conjectured that they were all prime.  Explore this conjecture, at first by hand, and when the numbers get too big switch to sage.

\vfill


\end{enumerate}

\end{document}
