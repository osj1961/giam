\documentclass{amsart}
\usepackage{amssymb}
\renewcommand{\baselinestretch}{1.5}
\addtolength{\textwidth}{.2in}
\addtolength{\topmargin}{-.5in}
\addtolength{\textheight}{1in}

\usepackage{ifthen}

\newcommand{\versionNum}{$3.2$\ }

\newboolean{InTextBook}
\setboolean{InTextBook}{false}
\newboolean{InWorkBook}
\setboolean{InWorkBook}{false}
\newboolean{InHints}
\setboolean{InHints}{false}

%When this boolean is true (beginning in Section 5.1) we will use the convention
% that $0 \in \Naturals$.  If it is false we will continue to count $1$ as the smallest
%natural number (thus making Giuseppe Peano spin in his grave...)
 
\newboolean{ZeroInNaturals}

%This boolean is used to distinguish the version where we use $\sim$ rather than $\lnot$

\newboolean{LNotIsSim}

%The values of the last two booleans are set in ``switches.tex''

%\input{switches}

\let\savedlnot\lnot
\ifthenelse{\boolean{LNotIsSim}}{\renewcommand{\lnot}{\sim} }{}

%This command puts different amounts of space depending on whether we are
% in the text, the workbook or the hints & solutions manual. 
\newcommand{\twsvspace}[3]{%
 \ifthenelse{\boolean{InTextBook} }{\vspace{#1}}{%
  \ifthenelse{\boolean{InWorkBook} }{\vspace{#2}}{%
   \ifthenelse{\boolean{InHints} }{\vspace{#3}}{} %
   }%
  }%
 }


\newcommand{\wbvfill}{\ifthenelse{\boolean{InWorkBook}}{\vfill}{}}
\newcommand{\wbitemsep}{\ifthenelse{\boolean{InWorkBook} }{\rule[-24pt]{0pt}{60pt}}{}}
\newcommand{\textbookpagebreak}{\ifthenelse{\boolean{InTextBook}}{\newpage}{}}
\newcommand{\workbookpagebreak}{\ifthenelse{\boolean{InWorkBook}}{\newpage}{}}
\newcommand{\hintspagebreak}{\ifthenelse{\boolean{InHints}}{\newpage}{}}

\newcommand{\hint}[1]{\ifthenelse{\boolean{InHints}}{ {\par \hspace{12pt} \color[rgb]{0,0,1} #1 } }{}}
\newcommand{\inlinehint}[1]{\ifthenelse{\boolean{InHints}}{ { \color[rgb]{0,0,1} #1 } }{}}

%\newlength{\cwidth}
%\newcommand{\cents}{\settowidth{\cwidth}{c}%
%\divide\cwidth by2
%\advance\cwidth by-.1pt
%c\kern-\cwidth
%\vrule width .1pt depth.2ex height1.2ex
%\kern 3\cwidth}
\newcommand{\cents}{\textcent\kern 5pt}

\newcommand{\sageprompt}{ {\tt sage$>$} }
\newcommand{\tab}{\rule{20pt}{0pt}}
\newcommand{\blnk}{\rule{1.5pt}{0pt}\rule{.4pt}{1.2pt}\rule{9pt}{.4pt}\rule{.4pt}{1.2pt}\rule{1.5pt}{0pt}}
\newcommand{\suchthat}{\; \rule[-3pt]{.5pt}{13pt} \;}
\newcommand{\divides}{\!\mid\!}
\newcommand{\tdiv}{\; \mbox{div} \;}
\newcommand{\restrict}[2]{#1 \,\rule[-4pt]{.25pt}{14pt}_{\,#2}}
\newcommand{\lcm}[2]{\mbox{lcm} (#1, #2)}
\renewcommand{\gcd}[2]{\mbox{gcd} (#1, #2)}
\newcommand{\Naturals}{{\mathbb N}}
\newcommand{\Integers}{{\mathbb Z}}
\newcommand{\Znoneg}{{\mathbb Z}^{\mbox{\tiny noneg}}}
\ifthenelse{\boolean{ZeroInNaturals}}{%
  \newcommand{\Zplus}{{\mathbb Z}^+} }{%
  \newcommand{\Zplus}{{\mathbb N}} }
\newcommand{\Enoneg}{{\mathbb E}^{\mbox{\tiny noneg}}}
\newcommand{\Qnoneg}{{\mathbb Q}^{\mbox{\tiny noneg}}}
\newcommand{\Rnoneg}{{\mathbb R}^{\mbox{\tiny noneg}}}
\newcommand{\Rationals}{{\mathbb Q}}
\newcommand{\Reals}{{\mathbb R}}
\newcommand{\Complexes}{{\mathbb C}}
%\newcommand{\F2}{{\mathbb F}_{2}}
\newcommand{\relQ}{\mbox{\textsf Q}}
\newcommand{\relR}{\mbox{\textsf R}}
\newcommand{\nrelR}{\mbox{\raisebox{1pt}{$\not$}\rule{1pt}{0pt}{\textsf R}}}
\newcommand{\relS}{\mbox{\textsf S}}
\newcommand{\relA}{\mbox{\textsf A}}
\newcommand{\Dom}[1]{\mbox{Dom}(#1)}
\newcommand{\Cod}[1]{\mbox{Cod}(#1)}
\newcommand{\Rng}[1]{\mbox{Rng}(#1)}

\DeclareMathOperator\caret{\raisebox{1ex}{$\scriptstyle\wedge$}}

\newtheorem*{defi}{Definition}
\newtheorem*{exer}{Exercise}
\newtheorem{thm}{Theorem}[section]
\newtheorem*{thm*}{Theorem}
\newtheorem{lem}[thm]{Lemma}
\newtheorem*{lem*}{Lemma}
\newtheorem{cor}{Corollary}
\newtheorem{conj}{Conjecture}

\renewenvironment{proof}%
{\begin{quote} \emph{Proof:} }%
{\rule{0pt}{0pt} \newline \rule{0pt}{15pt} \hfill Q.E.D. \end{quote}}


\addtolength{\abovedisplayskip}{0pt}
\addtolength{\belowdisplayskip}{24pt}
\addtolength{\abovedisplayshortskip}{0pt}
\addtolength{\belowdisplayshortskip}{44pt}


\begin{document}
\thispagestyle{empty}

\centerline{\Large Activity 16 -- Introduction to Proof}
\centerline{\large indirect arguments}

\bigskip
\Large


\begin{enumerate}

\item Prior to starting a proof by contradiction you must first determine the logical negation of the statement you are hoping to prove.  Find negations of the following.
\begin{enumerate}
\item \rule[-6pt]{0pt}{24pt} There are infinitely-many natural number $p$ such that $p$ is prime and $p+2$ is prime.
\item \rule[-6pt]{0pt}{24pt} $\displaystyle \forall n \in \Naturals, \; n^3 \; \mbox{is even} \; \implies \; n \; \mbox{is even} $
\item \rule[-6pt]{0pt}{24pt} $\displaystyle \forall n \in \Naturals, \; ( n \geq 2 ) \; \implies (\exists p,q \in \mbox{Primes}, \; p+q \, = \, 2n$
\end{enumerate}

\vfill

\item Proofs by contradiction and by contraposition are considered equivalent because, if you have a proof of one sort, you can recast it as the other sort.  If we are trying to prove a universal conditional sentence $\forall x, P(x) \implies Q(x)$,
a proof by contradiction consists in showing that $\exists x, (P(x) \land \lnot Q(x)) \implies c$, while a proof by contraposition consists of showing that  $\forall x, \lnot Q(x) \implies \lnot P(x)$.

Use a truth table to determine whether $(P \land \lnot Q) \implies c$ and $\lnot Q \implies \lnot P$ are equivalent.

\vfill

\vfill

\newpage

\item Devise a proof by contraposition and a proof by contradiction for item (b) in problem 1.

\vfill

\item Show, by contradiction, that there is no smallest positive real number.

\vfill

\newpage

\item Show that $\displaystyle \forall x, y \in \Reals, (x \in \Rationals \; \land \; xy \notin \Rationals) \; \implies \; y \notin \Rationals$.

\vfill

\item Prove that $\displaystyle \forall x, y \in \Integers, \; (x >1 ) \; \implies \; ( x \nmid y \; \lor \; x \nmid y+1 )$

\vfill

\newpage

\item Let's do one that's not an indirect argument.  (Just to mix things up a little.) The squares of natural numbers can only have certain values mod 4.  In particular, $\forall x \in \Naturals, \; x^2 \mod{4} = 0 \; \mbox{or} \; x^2 \mod{4} = 1$.  You can prove this by looking at two possibilities: either $x$ is even, or $x$ is odd.

\vfill

\item A \emph{Pythagorean triple} is a set of three
natural numbers, $a$, $b$ and $c$, that satisfy the Pythagorean theorem: $a^2 + b^2 = c^2$.   Prove that, in a
Pythagorean triple, at least one of $a$ and $b$ is even. (Use contradiction -- it will involve the statement we just proved in problem 7.

\vfill

\end{enumerate}

\end{document}
