\documentclass{amsart}
%\usepackage{}
\renewcommand{\baselinestretch}{1.5}
\addtolength{\textwidth}{.2in}

\newlength{\cwidth}
\newcommand{\cents}{\settowidth{\cwidth}{c}%
\divide\cwidth by2
\advance\cwidth by-.1pt
c\kern-\cwidth
\vrule width .1pt depth.2ex height1.2ex
\kern\cwidth}

\newcommand{\sageprompt}{ {\tt sage$>$} }
\newcommand{\tab}{\rule{20pt}{0pt}}
\newcommand{\blnk}{\rule{1.5pt}{0pt}\rule{.4pt}{1.2pt}\rule{9pt}{.4pt}\rule{.4pt}{1.2pt}\rule{1.5pt}{0pt}}
\newcommand{\suchthat}{\; \rule[-3pt]{.25pt}{13pt} \;}
\newcommand{\divides}{\!\mid\!}
\newcommand{\tdiv}{\; \mbox{div} \;}
\newcommand{\restrict}[2]{#1 \,\rule[-4pt]{.125pt}{14pt}_{\,#2}}
\newcommand{\lcm}[2]{\mbox{lcm} (#1, #2)}
\renewcommand{\gcd}[2]{\mbox{gcd} (#1, #2)}
\newcommand{\Naturals}{{\mathbb N}}
\newcommand{\Integers}{{\mathbb Z}}
\newcommand{\Znoneg}{{\mathbb Z}^{\mbox{\tiny noneg}}}
\newcommand{\Enoneg}{{\mathbb E}^{\mbox{\tiny noneg}}}
\newcommand{\Qnoneg}{{\mathbb Q}^{\mbox{\tiny noneg}}}
\newcommand{\Rnoneg}{{\mathbb R}^{\mbox{\tiny noneg}}}
\newcommand{\Rationals}{{\mathbb Q}}
\newcommand{\Reals}{{\mathbb R}}
\newcommand{\Complexes}{{\mathbb C}}
%\newcommand{\F2}{{\mathbb F}_{2}}
\newcommand{\relQ}{\mbox{\textsf Q}}
\newcommand{\relR}{\mbox{\textsf R}}
\newcommand{\nrelR}{\mbox{\raisebox{1pt}{$\not$}\rule{1pt}{0pt}{\textsf R}}}
\newcommand{\relS}{\mbox{\textsf S}}
\newcommand{\relA}{\mbox{\textsf A}}
\newcommand{\Dom}[1]{\mbox{Dom}(#1)}
\newcommand{\Cod}[1]{\mbox{Cod}(#1)}
\newcommand{\Rng}[1]{\mbox{Rng}(#1)}

\DeclareMathOperator\caret{\raisebox{1ex}{$\scriptstyle\wedge$}}

\newtheorem*{defi}{Definition}
\newtheorem*{exer}{Exercise}
\newtheorem{thm}{Theorem}[section]
\newtheorem*{thm*}{Theorem}
\newtheorem{lem}[thm]{Lemma}
\newtheorem{cor}{Corollary}
\newtheorem{conj}{Conjecture}

\renewenvironment{proof}%
{\begin{quote} \emph{Proof:} }%
{\rule{0pt}{0pt} \newline \rule{0pt}{15pt} \hfill Q.E.D. \end{quote}}


\begin{document}
\thispagestyle{empty}

\centerline{\Large Activity 6 -- Introduction to Proof}
\centerline{\large irrational numbers}

\bigskip
\Large


\begin{enumerate}
\item Which of the following pairs of numbers are relatively prime? Do as many as you can ``by hand,'' but feel free to use sage for the tougher ones.

\vspace{.1in}

\begin{tabular}{cccc}
 & \rule{72pt}{0pt} & \rule{72pt}{0pt} & \rule{72pt}{0pt} \\
\rule[-15pt]{0pt}{44pt} & $52$ \& $143$ &  $7$ \& $112$ & $2457$ \& $1265$\\
\rule[-15pt]{0pt}{44pt} & $59$ \& $69$ & $10000$ \& $101$ & $5183$ \& $4189$ \\
\rule[-15pt]{0pt}{44pt} & $256$ \& $243$ & $13706$ \& $14861$ & $999$ \& $1000$ \\
\end{tabular}

\rule{0pt}{0pt}

\vspace{.1in}

\rule{0pt}{0pt}

\item True or False: A number and its immediate successor (i.e. $n$ and $n+1$) are always relatively prime. Explain your answer. (Hint: it may be better to think about $n \mod d$ rather than $d\divides n$ for this.)

\vfill

\item True or False: A number and its successor's successor (i.e. $n$ and $n+2$) are never relatively prime. Explain your answer.

\vfill

\item Play around (with sage or your calculator) and find a criterion for when the squareroot of an integer is irrational.

\vfill

\newpage

\item Recall the lemma we used as part of the proof of the irrationality of $\sqrt{2}$:

\[ \forall x \in \Integers, \quad \mbox{if} \; x^2 \; \mbox{is even, then } \; x \; \mbox{is even}. \]

This can be restated using the divisibility symbol as:

\[ \forall x \in \Integers, \quad \mbox{if} \; 2 \divides x^2 \; \mbox{then } \; 2\divides x. \]

State the lemma that we'd need for making a similar proof (to that of Hippassus) which shows that  $\sqrt{3}$ is irrational.

\vfill

\item A similar change (substituting 4 for 2) gives:

\[ \forall x \in \Integers, \quad \mbox{if} \; 4 \divides x^2 \; \mbox{then } \; 4\divides x. \]

\noindent but this lemma is false!  Provide a counterexample and explain what this has to do with your answer to \# 4.

\vfill

\newpage

\item A variant of Hippasus' proof involves slightly weakening the ``relatively prime'' restriction on the numerator and denomenator of a fraction.  The variant would start with:

\begin{quote} 
{\em Suppose to the contrary that $\displaystyle \sqrt{2} = \frac{a}{b}$ where $a$ and $b$ are integers which are not both even.}
\end{quote}

How would you end this version of the proof? The middle part of the argument would stay the same. You just need to finish the sentence that begins with ``Finally, we have arrived at the desired absurdity because\textellipsis''

\vfill

\item Try writing a proof that $\sqrt{3}$ is irrational using the lemma 

\[ \forall x \in \Integers, \; \mbox{if} \; 3 \divides x^2 \; \mbox{then } \; 3 \divides x. \]

\noindent and the variant approach from the previous problem.

\vfill



\end{enumerate}

\end{document}
