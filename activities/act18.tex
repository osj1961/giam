\documentclass{amsart}
\usepackage{amssymb}
\renewcommand{\baselinestretch}{1.5}
\addtolength{\textwidth}{.2in}
\addtolength{\topmargin}{-.5in}
\addtolength{\textheight}{1in}

\usepackage{ifthen}


\newlength{\cwidth}
\newcommand{\cents}{\settowidth{\cwidth}{c}%
\divide\cwidth by2
\advance\cwidth by-.1pt
c\kern-\cwidth
\vrule width .1pt depth.2ex height1.2ex
\kern\cwidth}

\newcommand{\sageprompt}{ {\tt sage$>$} }
\newcommand{\tab}{\rule{20pt}{0pt}}
\newcommand{\blnk}{\rule{1.5pt}{0pt}\rule{.4pt}{1.2pt}\rule{9pt}{.4pt}\rule{.4pt}{1.2pt}\rule{1.5pt}{0pt}}
\newcommand{\suchthat}{\; \rule[-3pt]{.25pt}{13pt} \;}
\newcommand{\divides}{\!\mid\!}
\newcommand{\tdiv}{\; \mbox{div} \;}
\newcommand{\restrict}[2]{#1 \,\rule[-4pt]{.125pt}{14pt}_{\,#2}}
\newcommand{\lcm}[2]{\mbox{lcm} (#1, #2)}
\renewcommand{\gcd}[2]{\mbox{gcd} (#1, #2)}
\newcommand{\Naturals}{{\mathbb N}}
\newcommand{\Integers}{{\mathbb Z}}
\newcommand{\Znoneg}{{\mathbb Z}^{\mbox{\tiny noneg}}}
\newcommand{\Enoneg}{{\mathbb E}^{\mbox{\tiny noneg}}}
\newcommand{\Qnoneg}{{\mathbb Q}^{\mbox{\tiny noneg}}}
\newcommand{\Rnoneg}{{\mathbb R}^{\mbox{\tiny noneg}}}
\newcommand{\Rationals}{{\mathbb Q}}
\newcommand{\Reals}{{\mathbb R}}
\newcommand{\Complexes}{{\mathbb C}}
%\newcommand{\F2}{{\mathbb F}_{2}}
\newcommand{\relQ}{\mbox{\textsf Q}}
\newcommand{\relR}{\mbox{\textsf R}}
\newcommand{\nrelR}{\mbox{\raisebox{1pt}{$\not$}\rule{1pt}{0pt}{\textsf R}}}
\newcommand{\relS}{\mbox{\textsf S}}
\newcommand{\relA}{\mbox{\textsf A}}
\newcommand{\Dom}[1]{\mbox{Dom}(#1)}
\newcommand{\Cod}[1]{\mbox{Cod}(#1)}
\newcommand{\Rng}[1]{\mbox{Rng}(#1)}

\DeclareMathOperator\caret{\raisebox{1ex}{$\scriptstyle\wedge$}}

\newtheorem*{defi}{Definition}
\newtheorem*{exer}{Exercise}
\newtheorem{thm}{Theorem}[section]
\newtheorem*{thm*}{Theorem}
\newtheorem{lem}[thm]{Lemma}
\newtheorem{cor}{Corollary}
\newtheorem{conj}{Conjecture}

\renewenvironment{proof}%
{\begin{quote} \emph{Proof:} }%
{\rule{0pt}{0pt} \newline \rule{0pt}{15pt} \hfill Q.E.D. \end{quote}}


\addtolength{\abovedisplayskip}{0pt}
\addtolength{\belowdisplayskip}{24pt}
\addtolength{\abovedisplayshortskip}{0pt}
\addtolength{\belowdisplayshortskip}{44pt}


\begin{document}
\thispagestyle{empty}

\centerline{\Large Activity 18 -- Introduction to Proof}
\centerline{\large cases and exhaustion}

\bigskip
\Large


\begin{enumerate}

\item Redo the proof that $\forall x \in \Reals, \; x^2 \geq 0$, using the trichotomy property and proof by cases. 

\vfill

\item We discussed two different ways to talk about the cases, when the cases come from the quotient remainder theorem.

For example,

\begin{itemize}
\item[\bf Case 2:] ($x$ is of the form $4q+1$)
\end{itemize}
\noindent vs.
\begin{itemize}
\item[\bf Case 2:] ($x \mod 4 \; = \; 1$ )
\end{itemize}

Use the 2nd approach to redo the proof by cases that squares are either $0$ or $1 \mod 4$.

\vfill

\newpage

\item In the four four's puzzle if we allow factorials ($!$) and squareroots ($\sqrt{\phantom{x}}$) it is possible to find expressions for many more whole numbers.  $10$ is pretty doable, $11$ is quite hard!  How many more can you do?

\vfill

\item Do an exhaustive proof that for all pairs of primes $p$ and $q$, with $p < q \leq 13$, 
\[ p^q \mod{q} \; = \; p  \]

\vfill

\newpage

\item If we're working in $\!\!\mod{p}$ arithmetic, where $p$ is a prime, about half of the potential residues $\{0,1,2, \ldots p-1\}$ occur as the values of squares.  For example, in $\mod{5}$ arithmetic, squares are always in $\{0,1,4\}$.  The modular values that can be squares are called {\em quadratic residues}.  Determine what are the quadratic residues in $\mod{7}$ arithmetic, and prove your answer with a proof by cases.

\vfill

\item (GIAM \S 3.5 \# 7) Lagrange's theorem on representation of integers as sums of squares
says that every positive integer can be expressed as the sum of at most 
$4$ squares.  For example, $79 = 7^2 + 5^2 + 2^2 + 1^2$.  Show (exhaustively) 
that $15$ can not be represented using fewer than $4$ squares.

\vfill 

\newpage

\item How many numbers less than 30 cannot be represented using fewer than $4$ squares?

\vfill

\item The graph $C_4$ has 4 nodes that are connected in a circular fashion.  What is its' pebbling number?

\vfill

\end{enumerate}

\end{document}
