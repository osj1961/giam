\documentclass{amsart}
\usepackage{amssymb}
\renewcommand{\baselinestretch}{1.5}
\addtolength{\textwidth}{.2in}
\addtolength{\topmargin}{-.5in}
\addtolength{\textheight}{1in}

\usepackage{ifthen}

\newcommand{\versionNum}{$3.2$\ }

\newboolean{InTextBook}
\setboolean{InTextBook}{false}
\newboolean{InWorkBook}
\setboolean{InWorkBook}{false}
\newboolean{InHints}
\setboolean{InHints}{false}

%When this boolean is true (beginning in Section 5.1) we will use the convention
% that $0 \in \Naturals$.  If it is false we will continue to count $1$ as the smallest
%natural number (thus making Giuseppe Peano spin in his grave...)
 
\newboolean{ZeroInNaturals}

%This boolean is used to distinguish the version where we use $\sim$ rather than $\lnot$

\newboolean{LNotIsSim}

%The values of the last two booleans are set in ``switches.tex''

%\input{switches}

\let\savedlnot\lnot
\ifthenelse{\boolean{LNotIsSim}}{\renewcommand{\lnot}{\sim} }{}

%This command puts different amounts of space depending on whether we are
% in the text, the workbook or the hints & solutions manual. 
\newcommand{\twsvspace}[3]{%
 \ifthenelse{\boolean{InTextBook} }{\vspace{#1}}{%
  \ifthenelse{\boolean{InWorkBook} }{\vspace{#2}}{%
   \ifthenelse{\boolean{InHints} }{\vspace{#3}}{} %
   }%
  }%
 }


\newcommand{\wbvfill}{\ifthenelse{\boolean{InWorkBook}}{\vfill}{}}
\newcommand{\wbitemsep}{\ifthenelse{\boolean{InWorkBook} }{\rule[-24pt]{0pt}{60pt}}{}}
\newcommand{\textbookpagebreak}{\ifthenelse{\boolean{InTextBook}}{\newpage}{}}
\newcommand{\workbookpagebreak}{\ifthenelse{\boolean{InWorkBook}}{\newpage}{}}
\newcommand{\hintspagebreak}{\ifthenelse{\boolean{InHints}}{\newpage}{}}

\newcommand{\hint}[1]{\ifthenelse{\boolean{InHints}}{ {\par \hspace{12pt} \color[rgb]{0,0,1} #1 } }{}}
\newcommand{\inlinehint}[1]{\ifthenelse{\boolean{InHints}}{ { \color[rgb]{0,0,1} #1 } }{}}

%\newlength{\cwidth}
%\newcommand{\cents}{\settowidth{\cwidth}{c}%
%\divide\cwidth by2
%\advance\cwidth by-.1pt
%c\kern-\cwidth
%\vrule width .1pt depth.2ex height1.2ex
%\kern 3\cwidth}
\newcommand{\cents}{\textcent\kern 5pt}

\newcommand{\sageprompt}{ {\tt sage$>$} }
\newcommand{\tab}{\rule{20pt}{0pt}}
\newcommand{\blnk}{\rule{1.5pt}{0pt}\rule{.4pt}{1.2pt}\rule{9pt}{.4pt}\rule{.4pt}{1.2pt}\rule{1.5pt}{0pt}}
\newcommand{\suchthat}{\; \rule[-3pt]{.5pt}{13pt} \;}
\newcommand{\divides}{\!\mid\!}
\newcommand{\tdiv}{\; \mbox{div} \;}
\newcommand{\restrict}[2]{#1 \,\rule[-4pt]{.25pt}{14pt}_{\,#2}}
\newcommand{\lcm}[2]{\mbox{lcm} (#1, #2)}
\renewcommand{\gcd}[2]{\mbox{gcd} (#1, #2)}
\newcommand{\Naturals}{{\mathbb N}}
\newcommand{\Integers}{{\mathbb Z}}
\newcommand{\Znoneg}{{\mathbb Z}^{\mbox{\tiny noneg}}}
\ifthenelse{\boolean{ZeroInNaturals}}{%
  \newcommand{\Zplus}{{\mathbb Z}^+} }{%
  \newcommand{\Zplus}{{\mathbb N}} }
\newcommand{\Enoneg}{{\mathbb E}^{\mbox{\tiny noneg}}}
\newcommand{\Qnoneg}{{\mathbb Q}^{\mbox{\tiny noneg}}}
\newcommand{\Rnoneg}{{\mathbb R}^{\mbox{\tiny noneg}}}
\newcommand{\Rationals}{{\mathbb Q}}
\newcommand{\Reals}{{\mathbb R}}
\newcommand{\Complexes}{{\mathbb C}}
%\newcommand{\F2}{{\mathbb F}_{2}}
\newcommand{\relQ}{\mbox{\textsf Q}}
\newcommand{\relR}{\mbox{\textsf R}}
\newcommand{\nrelR}{\mbox{\raisebox{1pt}{$\not$}\rule{1pt}{0pt}{\textsf R}}}
\newcommand{\relS}{\mbox{\textsf S}}
\newcommand{\relA}{\mbox{\textsf A}}
\newcommand{\Dom}[1]{\mbox{Dom}(#1)}
\newcommand{\Cod}[1]{\mbox{Cod}(#1)}
\newcommand{\Rng}[1]{\mbox{Rng}(#1)}

\DeclareMathOperator\caret{\raisebox{1ex}{$\scriptstyle\wedge$}}

\newtheorem*{defi}{Definition}
\newtheorem*{exer}{Exercise}
\newtheorem{thm}{Theorem}[section]
\newtheorem*{thm*}{Theorem}
\newtheorem{lem}[thm]{Lemma}
\newtheorem*{lem*}{Lemma}
\newtheorem{cor}{Corollary}
\newtheorem{conj}{Conjecture}

\renewenvironment{proof}%
{\begin{quote} \emph{Proof:} }%
{\rule{0pt}{0pt} \newline \rule{0pt}{15pt} \hfill Q.E.D. \end{quote}}


\addtolength{\abovedisplayskip}{0pt}
\addtolength{\belowdisplayskip}{24pt}
\addtolength{\abovedisplayshortskip}{0pt}
\addtolength{\belowdisplayshortskip}{44pt}


\begin{document}
\thispagestyle{empty}

\centerline{\Large Activity 20 -- Introduction to Proof}
\centerline{\large intro to set theory}

\bigskip
\Large


\begin{enumerate}

\item What is the truth set of $P(x) = $ ``$x$ is divisible by $3$ and $x$ is a square'' ?

\vfill

\item What logical open sentence corresponds to the set 

\[ A = \{ 1, 5, 9, 13, 17, 21, \ldots \} ? \]

\vfill

\item What are the cardinalities (i.e.\ sizes, i.e.\ how many elements?) of the following sets?

\[  \{1,2,3,4\} \qquad \{1,2,\{3,4\}\} \qquad \{\{1,2\},\{3,4\}\} \qquad \{\{1,2,3,4\}\}  \]

\vfill

\newpage

\item Which of the following are equal?

\hspace{-1.25in}\begin{tabular}{llllllll}
 & \rule{1.1in}{0pt} &  & \rule{1.1in}{0pt} &  & \rule{1.1in}{0pt} & & \rule{1.1in}{0pt} \\
\rule[-6pt]{0pt}{24pt} (i) & $\{1,2\}$ & (ii) & $\{2\}$ & (iii) & $\{1,2,3,4\}$ & (iv) & $\{1,2,\{3,4\}\}$ \\
\rule[-6pt]{0pt}{24pt} (v) & $\{2,2,2\}$ & (vi) & $\{1,3,2,4\}$ & (vii) & $\{2,\{2,2\}\}$ & (viii) & $\{1,2,1,2\}$\\
\rule[-6pt]{0pt}{24pt} (ix) & $\{2,1\}$ & (x) & $\{1,2,\{4,3\}\}$ & (xi) & $\{4,3,2,1,2,3,4\}$ & (xii) & $\{\{2\},2\}$ \\
\rule[-6pt]{0pt}{24pt} (xiii) & $\{\{4,3\},2,1\}$ & (xiv) & $\{1,2,1\}$ & (xv) & $\{2,2\}$ & (xvi) & $\{2,1,\{3,4\}\}$\\
\end{tabular}

\vfill



\item In Logic we defined two special statements, $c$ and $t$, contradiction and tautology.  Supposing we are working in an unspecified universal set $U$, what are the Set Theory equivalents to $c$ and $t$?

\vfill

\item Complete the suggested exercise in the text by writing out the power sets of $\{1,2,3\}$, $\{1,2\}$, $\{1\}$ and $\emptyset$.  Conjecture a formula for the cardinality of ${\mathcal P}(\{1,2,3,\ldots n\})$.

\vfill

\newpage

\item The power set of $\{1, 2, 3, 4, 5\}$ will consist of sets having cardinalities between $0$ (the empty set) and $5$ (the entire set).  For each cardinality, give an example of a subset of $\{1, 2, 3, 4, 5\}$ having that cardinality and state how many subsets of $\{1, 2, 3, 4, 5\}$ will have that cardinality.

\vfill

\item Some of the sets in a power set are contained in others.  Create a graph (the sort of diagram we used in the pebbling number problems) that has nodes labelled by the elements of the power set of $\{1,2\}$ and edges (i.e.\ connections) between nodes where one set is contained in the other.

\vfill

\newpage

\item Make a diagram similar to the one in the previous problem, but with nodes labelled by divisors of $15$, and edges where one number divides another. 

\vfill

\item Is there a sets/inclusion graph\footnote{as in problem 8}  that corresponds to the numbers/divisibility graph\footnote{as in problem 9} for the divisors of $12$?  If so, what is a set?  If not, explain why not.

\vfill

\end{enumerate}

\end{document}
