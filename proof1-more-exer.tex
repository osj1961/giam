\begin{enumerate}
\item Suppose you have a savings account which bears interest 
compounded monthly.  The July statement shows a balance of 
\$ 2104.87 and the September statement shows a balance \$ 2125.97.
What would be the balance on the (missing) August statement?
\item Recall that a quadratic equation $ax^2+bx+c=0$ has two real solutions
if and only if the discriminant $b^2-4ac$ is positive.  Prove that if 
$a$ and $c$ have different signs then the quadratic equation has two 
real solutions.
\item Prove that if $x^3-x^2$ is negative then $3x+4 < 7$.

\item Prove that for all integers $a,b,$ and $c$, if $a|b$ and $a|(b+c)$, then
$a|c$.

\item Show that if $x$ is a positive real number, then $x+\frac{1}{x} \geq 2$. 

\item Prove that for all real numbers $a,b,$ and $c$, if $ac<0$, then the quadratic
equation $ax^{2}+bx+c=0$ has two real solutions.\\
\textbf{Hint:} The quadratic equation $ax^{2}+bx+c=0$ has two
real solutions if and only if $b^{2}-4ac>0$ and $a\neq0$.

\item Show that $\binom{n}{k} \cdot \binom{k}{r} \; = \; \binom{n}{r} \cdot \binom{n-r}{k-r}$ (for all integers $r$, $k$ and $n$ with $r \leq k \leq n$). 

\item In proving the \index{product rule} \emph{product rule} in Calculus using the definition of the derivative, we might start our proof with:

\[
\frac{\mbox{d}}{\mbox{d}x} \left( f(x) \cdot g(x) \right)
\]

\[ = \lim_{h \longrightarrow 0} \frac{f(x+h) \cdot g(x+h) - f(x) \cdot g(x)}{h} \]

\noindent The last two lines of our proof should be:
\[
= \lim_{h \longrightarrow 0} \frac{f(x+h) - f(x)}{h} \cdot g(x) \; + \; f(x) \cdot \lim_{h \longrightarrow 0} \frac{g(x+h) - g(x)}{h}
\]

\[
= \frac{\mbox{d}}{\mbox{d}x}\left( f(x) \right) \cdot g(x) \; + \; f(x) \cdot \frac{\mbox{d}}{\mbox{d}x}\left( g(x) \right) 
\]

Fill in the rest of the proof.
\end{enumerate}