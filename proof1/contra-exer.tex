\begin{enumerate}
\item Prove that if the cube of an integer is odd, then that integer is odd.

\hint{The best hint for this problem is simply to write down the contrapositive statement.  It is trivial to prove!}

\wbvfill

\item Prove that whenever a prime $p$ does not divide the square of an integer, 
it also doesn't divide the original integer. 
($p \nmid x^2 \; \implies \; p \nmid x$)

\hint{The contrapositive is $(p \divides x) \; \implies \; (p \divides x^2)$.}

\wbvfill

\workbookpagebreak

\item Prove (by contradiction) that there is no largest integer.

\hint{Well, if there was a largest integer -- let's call it $L$ (for largest) -- then isn't $L+1$ an integer, and isn't it bigger?  That's the main idea.  A more formal proof might look like this:

\begin{proof} 
Suppose (by way of contradiction) that there is a largest integer $L$.   Then $L \in \Integers$ and $\forall z \in \Integers, L \geq z$.
Consider the quantity $L+1$.  Clearly $L+1$ is an integer (because it is the sum of two integers) and also
$L+1 > L$.   This is a contradiction so the original supposition is false.   Hence there is no largest integer.
\end{proof}
}

\wbvfill

\item Prove (by contradiction) that there is no smallest positive real number.

\hint{Assume there was a smallest positive real number -- might as well call it $s$ (for smallest) -- what can we do to produce an even smaller number? (But be careful that it needs to remain positive -- for instance $s-1$ won't work.)}

\wbvfill

\workbookpagebreak

\item Prove (by contradiction) that the sum of a rational and an irrational 
number is irrational.

\hint{Suppose that x is rational and y is irrational and their sum (let's call it z) is also rational.  Do some algebra to solve for y, and you will see that y (which is, by presumption, irrational) is also the difference of two rational numbers (and hence, rational -- a contradiction.)
}

\wbvfill

%\workbookpagebreak

\item Prove (by contraposition) that for all integers $x$ and $y$, if $x+y$ is odd, then $x\neq y$.

\hint{Well, the problem says to do this by contraposition, so let's write down the contrapositive:

\[ \forall x, y \in \Integers, \; x=y \, \implies \, x+y \; \mbox{is even}. \]

But proving that is obvious!
}

\wbvfill

\workbookpagebreak

\item Prove (by contraposition) that for all real numbers $a$ and $b$, if $ab$ is irrational, then $a$
is irrational or $b$ is irrational.

\hint{The contrapositive would be:

\[ \forall a,b \in \Reals, \; (a \in \Rationals \land b \in \Rationals) \, \implies ab \in \Rationals. \]

Wow! Haven't we proved that before?}

\wbvfill


%\workbookpagebreak

\item A \index{Pythagorean triple}\emph{Pythagorean triple} is a set of three
natural numbers, $a$, $b$ and $c$, such that $a^2 + b^2 = c^2$.  Prove that, in a
Pythagorean triple, at least one of $a$ and $b$ is even.  Use either a proof by
contradiction or a proof by contraposition.

\hint{If both $a$ and $b$ are odd then their squares will be 1 mod 4 -- so the sum of their squares
will be 2 mod 4.  But $c^2$ can only be 0 or 1 mod 4, which gives us a contradiction.}

\wbvfill

\workbookpagebreak

\item Suppose you have 2 pairs of positive real numbers whose products are 1.  That is, you have $(a,b)$ and $(c,d)$ in $\Reals^2$ satisfying $ab=cd=1$.  Prove that
$a < c$ implies that $b > d$.

 \hint{
 \begin{proof}
 Suppose by way of contradiction that $a,b,c,d \in \Reals$ satisfy $ab=cd=1$ and that $a<c$ and $b \leq d$.
 By multiplying the inequalities we get that $ab < cd$ which contradicts the assumption that both products
 are equal to 1 (and so must be equal to one another).
 \end{proof} 
  } 
  
  \wbvfill
  
  \workbookpagebreak
  
\end{enumerate}
